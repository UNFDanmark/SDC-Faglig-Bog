%% PREAMBLE LaTeX file, DO NOT TOUCH
%% Denne fil skal ikke ændres af andre end LaTeX guruerne. Du skal holde 
%% nallerne væk, og ændre i den fil der ligger i den undermappe af ``sections'' 
%% som du skal bruge.

%% Sæt dokumentet op til at være skrift-størrelse 12 og af typen 'book'.
\documentclass[12pt]{book}

%% Sæt dokumentet til dansk
\usepackage[danish]{babel}
%% Brug UTF8 encoding.
\usepackage[utf8]{inputenc}
%% Brug 8-bit encoding der har 256 glyphs, something something accenter.
\usepackage[T1]{fontenc} 

%% blindtext lader os generere place-holder ``lorem-ipsum'' tekst.
\usepackage{blindtext}

%% xcolor lader os navngive vores farver
\usepackage{xcolor}

%% tikz bliver brugt til at tegne forskellige figurer, som kan bruges til 
%% f.eks. forsiden
\usepackage{tikz}

%% imakeidx lader os lave et overblik over referencer til vigtige emner i vores 
%% bog. Se det som en slags indholdsfortegnelse, af den slags der er 
%% bagerst i fagbøger.
\usepackage{imakeidx}
\makeindex % Fortæller LaTeX at den skal generere filerne der bruges til index.

%% Dette er den gennemgående farve i dokumentet som vores ``settings'' filer 
%% benytter. Den er pt. sat til ``UNF-blå'' fra logoet.
\definecolor{ocre}{RGB}{19,67,149}

%% Dette indeholder al vores opsætning af code-highlighting.
\input{settings-code}

%% Dette indeholder opsætning og makroer af noter i margin.
\input{settings-margin-notes}

%% Denne fil, indeholder opsætning af ``boxes'' såsom ``exercise'' eller 
%% ``example''
\input{settings-content-boxes}

%% Dette er en grim fil, der definerer udseendet af dokumentet. Hvis man 
%% sletter denne fil, får dokumentet et standard ``LaTeX look''.
\input{settings-book-style}

%% hyperref sørger for at vi kan lave links og referencer i vores dokument.
%% Pga. et problem med timing mellem opsætning af hyperref i 
%% ``settings-book-style'' og den generelle inklusion af biblioteket her, 
%% bliver vi nødt til at indsætte den efter.
\usepackage{hyperref}