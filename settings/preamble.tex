%% PREAMBLE LaTeX file, DO NOT TOUCH
%% Denne fil skal ikke ændres af andre end LaTeX guruerne. Du skal holde
%% nallerne væk, og ændre i den fil der ligger i den undermappe af ``sections''
%% som du skal bruge.

%% Sæt dokumentet op til at være skrift-størrelse 12 og af typen 'book'.
\documentclass[12pt]{book}

%% Sæt dokumentet til dansk
\usepackage[danish]{babel}
%% Brug UTF8 encoding.
\usepackage[utf8]{inputenc}
%% Brug 8-bit encoding der har 256 glyphs, something something accenter.
\usepackage[T1]{fontenc}

%% blindtext lader os generere place-holder ``lorem-ipsum'' tekst.
\usepackage{blindtext}

%% xcolor lader os navngive vores farver
\usepackage{xcolor}

%% tikz bliver brugt til at tegne forskellige figurer, som kan bruges til
%% f.eks. forsiden
\usepackage{tikz}

%% imakeidx lader os lave et overblik over referencer til vigtige emner i vores
%% bog. Se det som en slags indholdsfortegnelse, af den slags der er
%% bagerst i fagbøger.
\usepackage{imakeidx}
\makeindex % Fortæller LaTeX at den skal generere filerne der bruges til index.

%% Dette er den gennemgående farve i dokumentet som vores ``settings'' filer
%% benytter. Den er pt. sat til ``UNF-blå'' fra logoet.
\definecolor{ocre}{RGB}{19,67,149}

\addto\captionsdanish{%
  %\renewcommand{\figurename}{\textit{Bild}}%
  %\renewcommand{\tablename}{\textit{Tabelle}}%
}
\addto\extrasdanish{%
  \def\figureautorefname{Figur}%
  \def\tableautorefname{Tabel}%
  \def\subsectionautorefname{Sektion}%
  \def\sectionautorefname{Sektion}%
  \def\subsubsectionautorefname{Sektion}%
  \def\equationautorefname{Ligning}%
}

%% Dette indeholder al vores opsætning af code-highlighting.
%%
% CODE HIGHLIGHTING LaTeX file, DO NOT TOUCH
% Denne fil skal ikke ændres af andre end LaTeX guruerne. Du skal holde 
% nallerne væk, og ændre i den fil der ligger i den undermappe af ``sections'' 
% som du skal bruge.

%% Primitiv syntax rendering
\usepackage{fancyvrb}

%% Mere fancy syntax rendering
\usepackage{listings}
%% lstautogobble er en custom package, der fjerner leading spaces i lstlisting 
%% miljøer.
\usepackage{lstautogobble}

\usepackage{xcolor}

\newif\ifsyntaxcolor
\syntaxcolortrue
\ifsyntaxcolor
\definecolor{code-back}{RGB}{249,245,249}
\definecolor{code-comment}{RGB}{57,93,161}
\definecolor{code-keyword}{RGB}{127,0,85}
\definecolor{code-number}{RGB}{51,51,51}
\definecolor{code-string}{RGB}{51,151,51}
\else
\definecolor{code-back}{rgb}{1,1,1}
\definecolor{code-comment}{rgb}{0.5,0.5,0.5}
\definecolor{code-keyword}{rgb}{0.5,0.5,0.5}
\definecolor{code-number}{rgb}{0.5,0.5,0.5}
\definecolor{code-string}{rgb}{0.5,0.5,0.5}
\fi

% macro to select a scaled-down version of Bera Mono (for instance)
\makeatletter
\newcommand\BeraMonottfamily{%
	\def\fvm@Scale{0.75}% scales the font down
	\fontfamily{fvm}\selectfont% selects the Bera Mono font
}
\makeatother

\lstdefinestyle{basic}{
	%% Syntax color
	backgroundcolor=\color{code-back},
	commentstyle=\color{code-comment},
	keywordstyle=\color{code-keyword},
	numberstyle=\color{code-number},
	stringstyle=\color{code-string},
	basicstyle=\BeraMonottfamily,
	%% Line numbers
	numbers=left,
	numbersep=5pt,
	frame=leftline,
	%% Place caption
	captionpos=b,
	%% Hide spaces in strings
	showstringspaces=false,
	%% Tabsize
	tabsize=4,
	autogobble
}

\lstnewenvironment{JavaCode}[2]
	{\lstset{language=Java, style=basic, caption={#1}, label=#2}\noindent}
	{}
	
\newcommand{\JavaInline}{\lstset{language=Java, style=basic}\lstinline}
	
\lstnewenvironment{XmlCode}[2]
	{\lstset{language=XML, style=basic, caption={#1}, label=#2}\noindent}
	{}
	
\newcommand{\XmlInline}{\lstset{language=Xml, style=basic}\lstinline}
	
\lstnewenvironment{LaTeXCode}[2]
	{\lstset{language=[LaTeX]TeX, style=basic, caption={#1}, label=#2}\noindent}
	{}

\newcommand{\LaTeXInline}{\lstset{language=[LaTeX]TeX, style=basic}\lstinline}

	

%% Dette indeholder opsætning og makroer af noter i margin.
%%
% MARGINNOTE LaTeX file, DO NOT TOUCH
% Denne fil skal ikke ændres af andre end LaTeX guruerne. Du skal holde 
% nallerne væk, og ændre i den fil der ligger i den undermappe af ``sections'' 
% som du skal bruge.

\usepackage{marginnote}

\usepackage{graphicx}
\usepackage{caption}
\usepackage{float}

\newcommand{\marginfigure}[2]{
	\marginnote{
		\begin{center}
			\includegraphics[width=\marginparwidth]{#1}
			\captionof{figure}{#2}
		\end{center}
	}[0cm]
}

\newcommand{\bottomfigure}[2]{
	\begin{figure*}[b]
		\centering
		\includegraphics[width=\textwidth,height=2cm,keepaspectratio]{#1}
		\caption{#2}
	\end{figure*}
}

%% Denne fil, indeholder opsætning af ``boxes'' såsom ``exercise'' eller
%% ``example''
%%
% Denne fil er baseret på den book-template der er beskrevet i licensen 
% herunder. Derfor er den ikke dokumenteret på dansk, og lever ikke op til mine 
% (Lukas Jørgensen's) kode-standarder. Den er ærlig talt grim og kompliceret. 
% Så pil ikke ved noget, medmindre du ved hvad du laver!
%


%%%%%%%%%%%%%%%%%%%%%%%%%%%%%%%%%%%%%%%%%
% The Legrand Orange Book
% Structural Definitions File
% Version 2.0 (9/2/15)
%
% Original author:
% Mathias Legrand (legrand.mathias@gmail.com) with modifications by:
% Vel (vel@latextemplates.com)
% 
% This file has been downloaded from:
% http://www.LaTeXTemplates.com
%
% License:
% CC BY-NC-SA 3.0 (http://creativecommons.org/licenses/by-nc-sa/3.0/)
%
%%%%%%%%%%%%%%%%%%%%%%%%%%%%%%%%%%%%%%%%%

%----------------------------------------------------------------------------------------
%	THEOREM STYLES
%----------------------------------------------------------------------------------------

\usepackage{amsmath,amsfonts,amssymb,amsthm} % For math equations, 
%theorems, 
%symbols, etc

\newcommand{\intoo}[2]{\mathopen{]}#1\,;#2\mathclose{[}}
\newcommand{\ud}{\mathop{\mathrm{{}d}}\mathopen{}}
\newcommand{\intff}[2]{\mathopen{[}#1\,;#2\mathclose{]}}
\newtheorem{notation}{Notation}[chapter]

%TODO: Jeg har udkommenteret \@ifnotempty fordi at den opførte sig dumt.

% Boxed/framed environments
\newtheoremstyle{ocrenumbox}% % Theorem style name
{0pt}% Space above
{0pt}% Space below
{\normalfont}% % Body font
{}% Indent amount
{\small\bf\sffamily\color{ocre}}% % Theorem head font
{\;}% Punctuation after theorem head
{0.25em}% Space after theorem head
{\small\sffamily\color{ocre}\thmname{#1}\nobreakspace\thmnumber{#2}%\@ifnotempty{#1}{}\@upn{#2}}%
	% Theorem text (e.g. Theorem 2.1)
	\thmnote{\nobreakspace\the\thm@notefont\sffamily\bfseries\color{black}---\nobreakspace#3.}}
% Optional theorem note
\renewcommand{\qedsymbol}{$\blacksquare$}% Optional qed square

\newtheoremstyle{blacknumex}% Theorem style name
{5pt}% Space above
{5pt}% Space below
{\normalfont}% Body font
{} % Indent amount
{\small\bf\sffamily}% Theorem head font
{\;}% Punctuation after theorem head
{0.25em}% Space after theorem head
{\small\sffamily{\tiny\ensuremath{\blacksquare}}\nobreakspace\thmname{#1}\nobreakspace\thmnumber{#2}%\@ifnotempty{#1}{}\@upn{#2}}%
	% Theorem text (e.g. Theorem 2.1)
	\thmnote{\nobreakspace\the\thm@notefont\sffamily\bfseries---\nobreakspace#3.}}%
% 
%Optional theorem note

% Defines the theorem text style for each type of theorem to one of the three 
%styles above
\newcounter{dummy} 
\numberwithin{dummy}{section}
\theoremstyle{ocrenumbox}
\newtheorem{exerciseT}{Øvelse}[chapter]
\theoremstyle{blacknumex}
\newtheorem{exampleT}{Eksempel}[chapter]

%----------------------------------------------------------------------------------------
%	DEFINITION OF COLORED BOXES
%----------------------------------------------------------------------------------------

\RequirePackage[framemethod=default]{mdframed} % Required for creating the 
%theorem, definition, exercise and corollary boxes

% Exercise box	  
\newmdenv[skipabove=7pt,
skipbelow=7pt,
rightline=false,
leftline=true,
topline=false,
bottomline=false,
backgroundcolor=ocre!10,
linecolor=ocre,
innerleftmargin=5pt,
innerrightmargin=5pt,
innertopmargin=5pt,
innerbottommargin=5pt,
leftmargin=0cm,
rightmargin=0cm,
linewidth=4pt]{eBox}	

% Creates an environment for each type of theorem and assigns it a theorem text 
%style from the "Theorem Styles" section above and a colored box from above
\newenvironment{exercise}{\begin{eBox}\begin{exerciseT}}{\hfill{\color{ocre}\tiny\ensuremath{\blacksquare}}\end{exerciseT}\end{eBox}}


\newenvironment{example}{\begin{exampleT}}{\hfill{\tiny\ensuremath{\blacksquare}}\end{exampleT}}

%----------------------------------------------------------------------------------------
%	REMARK ENVIRONMENT
%----------------------------------------------------------------------------------------

\newenvironment{remark}{\par\vspace{10pt}\small % Vertical white space above 
	%the remark and smaller font size
	\begin{list}{}{
			\leftmargin=35pt % Indentation on the left
			\rightmargin=25pt}\item\ignorespaces % Indentation on the right
		\makebox[-2.5pt]{\begin{tikzpicture}[overlay]
			\node[draw=ocre!60,line 
			width=1pt,circle,fill=ocre!25,font=\sffamily\bfseries,inner 
			sep=2pt,outer 
			sep=0pt] at (-15pt,0pt){\textcolor{ocre}{B}};\end{tikzpicture}} % 
			%Orange R in a 
		%circle
		\advance\baselineskip -1pt}{\end{list}\vskip5pt} % Tighter line spacing 
		%and 
%white space after remark

%% Denne fil, indeholder opsætning af figurer og tabeller
\input{settings/figures-and-tables}

%% Dette er en grim fil, der definerer udseendet af dokumentet. Hvis man
%% sletter denne fil, får dokumentet et standard ``LaTeX look''.
\input{settings/book-style}

% En lille fil som giver flotte \todo makroer.

%%%%%%%%%%%%%%%%%%%%%%%%%%%%%%%%%%%%%%%%%%%%%%%%%%%%%%%%%%%%%%%%%%%%%%%%%%%%%%%%
%
% Hov! Hej. Undskyld for rodet, det er bare sådan at jeg kan have pæne TODO
% kasser, og som er nemme at se blandt al den skide tekst.

\newtheoremstyle{redbox}% % Theorem style name
{0pt}% Space above
{0pt}% Space below
{\normalfont}% % Body font
{}% Indent amount
{\small\bf\sffamily\color{red}}% % Theorem head font
{\;}% Punctuation after theorem head
{0.25em}% Space after theorem head
{\small\sffamily\color{red}\thmname{#1}%\@ifnotempty{#1}{}\@upn{#2}}%
	% Theorem text (e.g. Theorem 2.1)
	\thmnote{\nobreakspace\the\thm@notefont\sffamily\bfseries\color{black}---\nobreakspace#3.}}

\theoremstyle{redbox}
\newtheorem{todoT}{TODO}

% Exercise box
\newmdenv[skipabove=7pt,
skipbelow=7pt,
rightline=false,
leftline=true,
topline=false,
bottomline=false,
backgroundcolor=red!10,
linecolor=red,
innerleftmargin=5pt,
innerrightmargin=5pt,
innertopmargin=5pt,
innerbottommargin=5pt,
leftmargin=0cm,
rightmargin=0cm,
linewidth=4pt]{todoBox}

\newcommand{\todo}[1]{
	\begin{todoBox}\begin{todoT}
		#1
	\end{todoT}\end{todoBox}
}

%% hyperref sørger for at vi kan lave links og referencer i vores dokument.
%% Pga. et problem med timing mellem opsætning af hyperref i
%% ``settings-book-style'' og den generelle inklusion af biblioteket her,
%% bliver vi nødt til at indsætte den efter.
\usepackage{hyperref}

%% Indsæt et xspace efter \LaTeX, dette gør at der kommer et mellemrum efter
%% \LaTeX, når det giver mening.
\let\oldlatex\LaTeX
\renewcommand{\LaTeX}{\oldlatex\xspace}

%% AMS-Math: Har makroer til forskellige matematiske notationer, for
%% eksempel parvise funktioner, hvilket bruges til at diskutere
%% rekursion.
\usepackage{amsmath}

%% PDF Pages: Denne tillader at vi kan inkludere sider fra andre PDF'er, her brugt 
%% til at indlæse forsiden og bagsiden der er lavet i InDesign.
\usepackage{pdfpages}

%% Denne pakke giver os mulighed for at lave FloatBarriers, så vi kan forhindre at 
%% figurer flyder for langt væk fra deres sektion.
\usepackage{placeins}

%% Hjælpe kommando, der kan sørge for at lave tomme sider indtil vi er på en side 
%% til venstre.
\newcommand*\cleartoleftpage{%
	\clearpage
	\ifodd\value{page}\hbox{}\newpage\fi
}
