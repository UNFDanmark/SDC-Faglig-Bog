%%
% CODE HIGHLIGHTING LaTeX file, DO NOT TOUCH
% Denne fil skal ikke ændres af andre end LaTeX guruerne. Du skal holde 
% nallerne væk, og ændre i den fil der ligger i den undermappe af ``sections'' 
% som du skal bruge.

%% Primitiv syntax rendering
\usepackage{fancyvrb}

%% Mere fancy syntax rendering
\usepackage{listings}
%% lstautogobble er en custom package, der fjerner leading spaces i lstlisting 
%% miljøer.
\usepackage{lstautogobble}

\usepackage{xcolor}

\newif\ifsyntaxcolor
\syntaxcolortrue
\ifsyntaxcolor
\definecolor{code-back}{RGB}{249,245,249}
\definecolor{code-comment}{RGB}{57,93,161}
\definecolor{code-keyword}{RGB}{127,0,85}
\definecolor{code-number}{RGB}{51,51,51}
\definecolor{code-string}{RGB}{51,151,51}
\else
\definecolor{code-back}{rgb}{1,1,1}
\definecolor{code-comment}{rgb}{0.5,0.5,0.5}
\definecolor{code-keyword}{rgb}{0.5,0.5,0.5}
\definecolor{code-number}{rgb}{0.5,0.5,0.5}
\definecolor{code-string}{rgb}{0.5,0.5,0.5}
\fi

\lstdefinestyle{basic}{
	%% Syntax color
	backgroundcolor=\color{code-back},
	commentstyle=\color{code-comment},
	keywordstyle=\color{code-keyword},
	numberstyle=\color{code-number},
	stringstyle=\color{code-string},
	basicstyle=\footnotesize,
	%% Line numbers
	numbers=left,
	numbersep=5pt,
	frame=leftline,
	%% Place caption
	captionpos=b,
	%% Hide spaces in strings
	showstringspaces=false,
	%% Tabsize
	tabsize=4,
	autogobble
}

\lstnewenvironment{JavaCode}[2]
	{\lstset{language=Java, style=basic, caption=#1, label=#2}}
	{}

%\newenvironment{JavaCode}[2]
%	{\begin{XJavaCode}{#1}{#2}}
%	{\end{XJavaCode}}
	
\lstnewenvironment{XmlCode}[2]
	{\lstset{language=XML, style=basic, caption=#1, label=#2}}
	{}
	
%\newenvironment{XmlCode}[2]
%	{\begin{XXmlCode}{#1}{#2}}
%	{\end{XXmlCode}}
	
\lstnewenvironment{LaTeXCode}[2]
	{\lstset{language=[LaTeX]TeX, style=basic, caption=#1, label=#2}}
	{}
	
%\newenvironment{LaTeXCode}[2]
%	{\begin{XJavaCode}{#1}{#2}}
%	{\end{XJavaCode}}