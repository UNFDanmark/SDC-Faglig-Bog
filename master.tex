% !TeX spellcheck = da_DK

%% MAIN LaTeX file, DO NOT TOUCH
%% Denne fil skal ikke ændres af andre end LaTeX guruerne. Du skal holde
%% nallerne væk, og ændre i den fil der ligger i den undermappe af ``sections''
%% som du skal bruge.

%% Designet og strukturen i dette LaTeX dokument er primært udarbejdet af Lukas 
%% Jørgensen (LUP). Jeg er lidt stolt af den, og har arbejdet meget på den så 
%% behandl den pænt!
%% Den er baseret på ``The Legrand Orange Book'', men er modificeret kraftigt.
%% Er der nogen spørgsmål, eller tvivl, til bogen og hvordan LaTeX koden fungerer,
%% så kan jeg formentlig findes med en Google søgning efter ``LukasPJ'' eller på
%% lup@unf.dk. I værste fald, så prøv lpjoergensen@gmail.com.

%% Sæt dokumentet til draft-mode. Dette gør to-do's synlige, viser alle elementer i 
%% glossary og tilføjer ``example'' kapitlet.
\newif\ifdraftmode
\draftmodefalse

%% Dette indlæser vores ``preamble'' som er den fil der sørger for opsætningen
%% af vores dokument.
%% PREAMBLE LaTeX file, DO NOT TOUCH
%% Denne fil skal ikke ændres af andre end LaTeX guruerne. Du skal holde
%% nallerne væk, og ændre i den fil der ligger i den undermappe af ``sections''
%% som du skal bruge.

%% Sæt dokumentet op til at være skrift-størrelse 12 og af typen 'book'.
\documentclass[12pt]{book}

%% Sæt dokumentet til dansk
\usepackage[danish]{babel}
%% Brug UTF8 encoding.
\usepackage[utf8]{inputenc}
%% Brug 8-bit encoding der har 256 glyphs, something something accenter.
\usepackage[T1]{fontenc}

%% blindtext lader os generere place-holder ``lorem-ipsum'' tekst.
\usepackage{blindtext}

%% xcolor lader os navngive vores farver
\usepackage{xcolor}

%% tikz bliver brugt til at tegne forskellige figurer, som kan bruges til
%% f.eks. forsiden
\usepackage{tikz}

%% imakeidx lader os lave et overblik over referencer til vigtige emner i vores
%% bog. Se det som en slags indholdsfortegnelse, af den slags der er
%% bagerst i fagbøger.
\usepackage{imakeidx}
\makeindex % Fortæller LaTeX at den skal generere filerne der bruges til index.

%% Dette er den gennemgående farve i dokumentet som vores ``settings'' filer
%% benytter. Den er pt. sat til ``UNF-blå'' fra logoet.
\definecolor{ocre}{RGB}{19,67,149}

\addto\captionsdanish{%
  %\renewcommand{\figurename}{\textit{Bild}}%
  %\renewcommand{\tablename}{\textit{Tabelle}}%
}
\addto\extrasdanish{%
  \def\figureautorefname{Figur}%
  \def\tableautorefname{Tabel}%
  \def\subsectionautorefname{Sektion}%
  \def\sectionautorefname{Sektion}%
  \def\subsubsectionautorefname{Sektion}%
  \def\equationautorefname{Ligning}%
}

%% Dette indeholder al vores opsætning af code-highlighting.
%%
% CODE HIGHLIGHTING LaTeX file, DO NOT TOUCH
% Denne fil skal ikke ændres af andre end LaTeX guruerne. Du skal holde 
% nallerne væk, og ændre i den fil der ligger i den undermappe af ``sections'' 
% som du skal bruge.

%% Primitiv syntax rendering
\usepackage{fancyvrb}

%% Mere fancy syntax rendering
\usepackage{listings}
%% lstautogobble er en custom package, der fjerner leading spaces i lstlisting 
%% miljøer.
\usepackage{lstautogobble}

\usepackage{xcolor}

\newif\ifsyntaxcolor
\syntaxcolortrue
\ifsyntaxcolor
\definecolor{code-back}{RGB}{249,245,249}
\definecolor{code-comment}{RGB}{57,93,161}
\definecolor{code-keyword}{RGB}{127,0,85}
\definecolor{code-number}{RGB}{51,51,51}
\definecolor{code-string}{RGB}{51,151,51}
\else
\definecolor{code-back}{rgb}{1,1,1}
\definecolor{code-comment}{rgb}{0.5,0.5,0.5}
\definecolor{code-keyword}{rgb}{0.5,0.5,0.5}
\definecolor{code-number}{rgb}{0.5,0.5,0.5}
\definecolor{code-string}{rgb}{0.5,0.5,0.5}
\fi

% macro to select a scaled-down version of Bera Mono (for instance)
\makeatletter
\newcommand\BeraMonottfamily{%
	\def\fvm@Scale{0.75}% scales the font down
	\fontfamily{fvm}\selectfont% selects the Bera Mono font
}
\makeatother

\lstdefinestyle{basic}{
	%% Syntax color
	backgroundcolor=\color{code-back},
	commentstyle=\color{code-comment},
	keywordstyle=\color{code-keyword},
	numberstyle=\color{code-number},
	stringstyle=\color{code-string},
	basicstyle=\BeraMonottfamily,
	%% Line numbers
	numbers=left,
	numbersep=5pt,
	frame=leftline,
	%% Place caption
	captionpos=b,
	%% Hide spaces in strings
	showstringspaces=false,
	%% Tabsize
	tabsize=4,
	autogobble
}

\lstnewenvironment{JavaCode}[2]
	{\lstset{language=Java, style=basic, caption={#1}, label=#2}\noindent}
	{}
	
\newcommand{\JavaInline}{\lstset{language=Java, style=basic}\lstinline}
	
\lstnewenvironment{XmlCode}[2]
	{\lstset{language=XML, style=basic, caption={#1}, label=#2}\noindent}
	{}
	
\newcommand{\XmlInline}{\lstset{language=Xml, style=basic}\lstinline}
	
\lstnewenvironment{LaTeXCode}[2]
	{\lstset{language=[LaTeX]TeX, style=basic, caption={#1}, label=#2}\noindent}
	{}

\newcommand{\LaTeXInline}{\lstset{language=[LaTeX]TeX, style=basic}\lstinline}

	

%% Dette indeholder opsætning og makroer af noter i margin.
%%
% MARGINNOTE LaTeX file, DO NOT TOUCH
% Denne fil skal ikke ændres af andre end LaTeX guruerne. Du skal holde 
% nallerne væk, og ændre i den fil der ligger i den undermappe af ``sections'' 
% som du skal bruge.

\usepackage{marginnote}

\usepackage{graphicx}
\usepackage{caption}
\usepackage{float}

\newcommand{\marginfigure}[2]{
	\marginnote{
		\begin{center}
			\includegraphics[width=\marginparwidth]{#1}
			\captionof{figure}{#2}
		\end{center}
	}[0cm]
}

\newcommand{\bottomfigure}[2]{
	\begin{figure*}[b]
		\centering
		\includegraphics[width=\textwidth,height=2cm,keepaspectratio]{#1}
		\caption{#2}
	\end{figure*}
}

%% Denne fil, indeholder opsætning af ``boxes'' såsom ``exercise'' eller
%% ``example''
%%
% Denne fil er baseret på den book-template der er beskrevet i licensen 
% herunder. Derfor er den ikke dokumenteret på dansk, og lever ikke op til mine 
% (Lukas Jørgensen's) kode-standarder. Den er ærlig talt grim og kompliceret. 
% Så pil ikke ved noget, medmindre du ved hvad du laver!
%


%%%%%%%%%%%%%%%%%%%%%%%%%%%%%%%%%%%%%%%%%
% The Legrand Orange Book
% Structural Definitions File
% Version 2.0 (9/2/15)
%
% Original author:
% Mathias Legrand (legrand.mathias@gmail.com) with modifications by:
% Vel (vel@latextemplates.com)
% 
% This file has been downloaded from:
% http://www.LaTeXTemplates.com
%
% License:
% CC BY-NC-SA 3.0 (http://creativecommons.org/licenses/by-nc-sa/3.0/)
%
%%%%%%%%%%%%%%%%%%%%%%%%%%%%%%%%%%%%%%%%%

%----------------------------------------------------------------------------------------
%	THEOREM STYLES
%----------------------------------------------------------------------------------------

\usepackage{amsmath,amsfonts,amssymb,amsthm} % For math equations, 
%theorems, 
%symbols, etc

\newcommand{\intoo}[2]{\mathopen{]}#1\,;#2\mathclose{[}}
\newcommand{\ud}{\mathop{\mathrm{{}d}}\mathopen{}}
\newcommand{\intff}[2]{\mathopen{[}#1\,;#2\mathclose{]}}
\newtheorem{notation}{Notation}[chapter]

%TODO: Jeg har udkommenteret \@ifnotempty fordi at den opførte sig dumt.

% Boxed/framed environments
\newtheoremstyle{ocrenumbox}% % Theorem style name
{0pt}% Space above
{0pt}% Space below
{\normalfont}% % Body font
{}% Indent amount
{\small\bf\sffamily\color{ocre}}% % Theorem head font
{\;}% Punctuation after theorem head
{0.25em}% Space after theorem head
{\small\sffamily\color{ocre}\thmname{#1}\nobreakspace\thmnumber{#2}%\@ifnotempty{#1}{}\@upn{#2}}%
	% Theorem text (e.g. Theorem 2.1)
	\thmnote{\nobreakspace\the\thm@notefont\sffamily\bfseries\color{black}---\nobreakspace#3.}}
% Optional theorem note
\renewcommand{\qedsymbol}{$\blacksquare$}% Optional qed square

\newtheoremstyle{blacknumex}% Theorem style name
{5pt}% Space above
{5pt}% Space below
{\normalfont}% Body font
{} % Indent amount
{\small\bf\sffamily}% Theorem head font
{\;}% Punctuation after theorem head
{0.25em}% Space after theorem head
{\small\sffamily{\tiny\ensuremath{\blacksquare}}\nobreakspace\thmname{#1}\nobreakspace\thmnumber{#2}%\@ifnotempty{#1}{}\@upn{#2}}%
	% Theorem text (e.g. Theorem 2.1)
	\thmnote{\nobreakspace\the\thm@notefont\sffamily\bfseries---\nobreakspace#3.}}%
% 
%Optional theorem note

% Defines the theorem text style for each type of theorem to one of the three 
%styles above
\newcounter{dummy} 
\numberwithin{dummy}{section}
\theoremstyle{ocrenumbox}
\newtheorem{exerciseT}{Øvelse}[chapter]
\theoremstyle{blacknumex}
\newtheorem{exampleT}{Eksempel}[chapter]

%----------------------------------------------------------------------------------------
%	DEFINITION OF COLORED BOXES
%----------------------------------------------------------------------------------------

\RequirePackage[framemethod=default]{mdframed} % Required for creating the 
%theorem, definition, exercise and corollary boxes

% Exercise box	  
\newmdenv[skipabove=7pt,
skipbelow=7pt,
rightline=false,
leftline=true,
topline=false,
bottomline=false,
backgroundcolor=ocre!10,
linecolor=ocre,
innerleftmargin=5pt,
innerrightmargin=5pt,
innertopmargin=5pt,
innerbottommargin=5pt,
leftmargin=0cm,
rightmargin=0cm,
linewidth=4pt]{eBox}	

% Creates an environment for each type of theorem and assigns it a theorem text 
%style from the "Theorem Styles" section above and a colored box from above
\newenvironment{exercise}{\begin{eBox}\begin{exerciseT}}{\hfill{\color{ocre}\tiny\ensuremath{\blacksquare}}\end{exerciseT}\end{eBox}}


\newenvironment{example}{\begin{exampleT}}{\hfill{\tiny\ensuremath{\blacksquare}}\end{exampleT}}

%----------------------------------------------------------------------------------------
%	REMARK ENVIRONMENT
%----------------------------------------------------------------------------------------

\newenvironment{remark}{\par\vspace{10pt}\small % Vertical white space above 
	%the remark and smaller font size
	\begin{list}{}{
			\leftmargin=35pt % Indentation on the left
			\rightmargin=25pt}\item\ignorespaces % Indentation on the right
		\makebox[-2.5pt]{\begin{tikzpicture}[overlay]
			\node[draw=ocre!60,line 
			width=1pt,circle,fill=ocre!25,font=\sffamily\bfseries,inner 
			sep=2pt,outer 
			sep=0pt] at (-15pt,0pt){\textcolor{ocre}{B}};\end{tikzpicture}} % 
			%Orange R in a 
		%circle
		\advance\baselineskip -1pt}{\end{list}\vskip5pt} % Tighter line spacing 
		%and 
%white space after remark

%% Denne fil, indeholder opsætning af figurer og tabeller
\input{settings/figures-and-tables}

%% Dette er en grim fil, der definerer udseendet af dokumentet. Hvis man
%% sletter denne fil, får dokumentet et standard ``LaTeX look''.
\input{settings/book-style}

% En lille fil som giver flotte \todo makroer.

%%%%%%%%%%%%%%%%%%%%%%%%%%%%%%%%%%%%%%%%%%%%%%%%%%%%%%%%%%%%%%%%%%%%%%%%%%%%%%%%
%
% Hov! Hej. Undskyld for rodet, det er bare sådan at jeg kan have pæne TODO
% kasser, og som er nemme at se blandt al den skide tekst.

\newtheoremstyle{redbox}% % Theorem style name
{0pt}% Space above
{0pt}% Space below
{\normalfont}% % Body font
{}% Indent amount
{\small\bf\sffamily\color{red}}% % Theorem head font
{\;}% Punctuation after theorem head
{0.25em}% Space after theorem head
{\small\sffamily\color{red}\thmname{#1}%\@ifnotempty{#1}{}\@upn{#2}}%
	% Theorem text (e.g. Theorem 2.1)
	\thmnote{\nobreakspace\the\thm@notefont\sffamily\bfseries\color{black}---\nobreakspace#3.}}

\theoremstyle{redbox}
\newtheorem{todoT}{TODO}

% Exercise box
\newmdenv[skipabove=7pt,
skipbelow=7pt,
rightline=false,
leftline=true,
topline=false,
bottomline=false,
backgroundcolor=red!10,
linecolor=red,
innerleftmargin=5pt,
innerrightmargin=5pt,
innertopmargin=5pt,
innerbottommargin=5pt,
leftmargin=0cm,
rightmargin=0cm,
linewidth=4pt]{todoBox}

\newcommand{\todo}[1]{
	\begin{todoBox}\begin{todoT}
		#1
	\end{todoT}\end{todoBox}
}

%% hyperref sørger for at vi kan lave links og referencer i vores dokument.
%% Pga. et problem med timing mellem opsætning af hyperref i
%% ``settings-book-style'' og den generelle inklusion af biblioteket her,
%% bliver vi nødt til at indsætte den efter.
\usepackage{hyperref}

%% Indsæt et xspace efter \LaTeX, dette gør at der kommer et mellemrum efter
%% \LaTeX, når det giver mening.
\let\oldlatex\LaTeX
\renewcommand{\LaTeX}{\oldlatex\xspace}

%% AMS-Math: Har makroer til forskellige matematiske notationer, for
%% eksempel parvise funktioner, hvilket bruges til at diskutere
%% rekursion.
\usepackage{amsmath}

%% PDF Pages: Denne tillader at vi kan inkludere sider fra andre PDF'er, her brugt 
%% til at indlæse forsiden og bagsiden der er lavet i InDesign.
\usepackage{pdfpages}

%% Denne pakke giver os mulighed for at lave FloatBarriers, så vi kan forhindre at 
%% figurer flyder for langt væk fra deres sektion.
\usepackage{placeins}

%% Hjælpe kommando, der kan sørge for at lave tomme sider indtil vi er på en side 
%% til venstre.
\newcommand*\cleartoleftpage{%
	\clearpage
	\ifodd\value{page}\hbox{}\newpage\fi
}


\usepackage[toc]{glossaries}

\makeglossaries

\newglossaryentry{returnvalue}
{
	name={retur-værdi},
	description={(\textit{return value, output}): Den værdi som en funktion 
				 producerer.},
	plural={retur-værdier}
}

\newglossaryentry{returntype}
{
	name={retur-type},
	description={(\textit{return type}): Typen af en funktions 
				 \gls{returnvalue}.},
	plural={retur-typer}
}

\newglossaryentry{argument}
{
	name={argument},
	description={(\textit{argument, input}): Hvad der sættes mellem 
				 parenteserne.},
	plural={argumenter}
}

\newglossaryentry{string}
{
	name={streng},
	description={(\textit{string}): Typen af tekst værdier.},
	plural={strenge}
}

\newglossaryentry{interface}
{
	name={grænseflade},
	description={(\textit{interface, UI, GUI}): Det man ser på skærmen af
		 		 mobilen.},
	plural={grænseflader}
}
\newglossaryentry{frame}
{
	name={stilbillede},
	description={Et enkelt billede i en række billeder som danner en video},
	plural={stilbilleder}
}


%% Her er vores ``hoved-dokument''. Den sørger for den over-ordnede struktur af
%% bogen.
\begin{document}
	
	\graphicspath{{figures/}}
	%% Indsæt forsiden.
	\includepdf[page=2]{figures/forsidebagside.pdf}
	
	%% Indsæt forfatterlisten
	\cleardoublepage
\begin{titlepage}
	\centering
	\vspace{3cm}
	{\Huge Forfattere \par} \vspace{0.5cm}
	{\small\itshape For at denne bog blev skrevet, har det faglige hold brugt en masse af deres fritid på at skrive, diskutere, udtænke og designe bogen og dens indhold. Tak for jeres indsats!\par} \vspace{0.75cm}
	{\itshape Andreas Jensen (AMJ) \, Alexandra Hou (AOL) \\ 
		Beatrice Nyhus (BEN) \, Emil Aagreen (EFLA) \, Jon Aanes (JMA) \\ 
		Jonas 'Bamse' Andersen (JOBA) \, Lukas Jørgensen (LUP)\par}
	\vfill
	{\large Få bogen signeret\par} \vspace{0.5cm}
	{\fontsize{50}{80} $\rightarrow$\par} \vspace{2cm}
	{Kompileret \today\par}
\end{titlepage}
	
	%% Indsæt forord
	\cleardoublepage
\begin{titlepage}
	\centering
	\vspace{3cm}
	{\Huge Forfattere \par} \vspace{0.5cm}
	{\includegraphics[width=6cm]{digital_resource_lifespan}\\
		\tiny \url{https://xkcd.com/1909/} \par} \vspace{0.5cm}	
\end{titlepage}


	%% Print indholdsfortegnelsen
	\tableofcontents

	%% Herfra begynder det egentlige indhold. Vi bruger ``graphicspath'' til at
	%% isolere forskellige emners figurer fra hinanden, så der ikker bliver
	%% pillet i andres figurer, eller slettet figurer man ikke vidste andre
	%% brugte.
	%
	%% Filstrukturen er designet til at være:
	%% sections/<part>/<chapter>/chapter.tex
	%% sections/<part>/<chapter>/figures/

	\part{Java udvikling}

	% Hvis vi arbejder i draft mode, så inkluderer vi en eksempel side der 
	% viser nogle af de elementer der er i denne bog.
	\ifdraftmode
		\graphicspath{{sections/manual/example/figures/}}
		\chapter{Introduktion til Java}
Java er det programmeringssprog som I vil lære at bruge til at programmere apps med. Det er vidt anvendeligt. Det har i mange år været, og er stadig, det mest anvendte programmeringssprog på verdensplan.

\section{Hvad er et program, egentlig?}
Et program er en serie af instruktioner som skal udføres i rækkefølge. Typisk skrives instruktionerne som helt almindelig tekst, hvor hver instruktion står på sin egen linje.

\section{Sådan skriver du programmer}
Når du skal skrive programmer kan du i princippet gøre det i næsten hvilket som helst skriveprogram som ikke er Word, eksempelvis Notesblok på Windows. 

Det kan dog være lidt besværligt, og derfor har nogle valgt at udvikle programmer specielt til at skrive kode i, som kan hjælpe med alt muligt smart. 

Et program I kommer til at bruge til at skrive Java-kode i, er IntelliJ. Det kan blandt mange andre ting, hjælpe med at man får skrevet gyldige Java-programmer og køre dit program ved et enkelt klik på en knap.

\begin{remark}
	Hvis du har lyst til en lille udfordring, så prøv at skriv et program i notesblok eller tilsvarende, og kør programmet ved at bruge en terminal (kaldet kommandoprompt på Windows). 
	
	Dette gør du ved først at navigere til den mappe, hvor du har gemt dit program ved hjælp af commandoen \texttt{cd} (change directory), skriv \texttt{cd test} hvis du vil ind i mappen der hedder test. Hvis du gerne vil se hvilke mapper du har i den mappe du er i, så brug kommandoen \texttt{dir} på Windows, eller \texttt{ls} på de fleste andre systemer. 
	
	Når du er i den rigtige mappe skal du kompilere dit program ved at skrive \texttt{javac Test.java} (c'et er for compile), hvis dit program hedder "Test" (det er vigtigt at Java-programmer slutter på ".java"), dette laver en fil ved navn \texttt{Test.class}, som er dit færdige program. 
	
	Til sidst kan du køre programmet ved at skrive \texttt{java Test}.
\end{remark}

\section{Hello World!}
Et "Hello World!" program er ofte det første man prøver i et nyt programmeringssprog, for at få en lille smule føling med, hvordan sproget skrives.

\begin{JavaCode}{A Hello World program in Java}{lst:helloworld}
public class Hello {
	public static void main(String[] args) {
		System.out.println("Hello World!");
	}
}
\end{JavaCode}

Til at starte med ser dette måske lidt skræmmende ud, men bare rolig, vi guider dig igennem det hele, skridt for skridt. Hvis du har programmet i Listing \ref{lst:helloworld} til at fungere, vil det sige \texttt{Hello World!} som output, men hvis du synes det er lidt kedeligt skal du bare ændre teksten imellem anførselstegnene på linje 3, til noget du synes er sjovere.

\begin{remark}
	Husk at programmet skal gemmes som "Test.java" med stort forbogstav, og skal hedde det samme som det der står på linjen, hvor der står \texttt{public class Test \{}
\end{remark}

Det eneste der sker i dit program er præcis de ting der står mellem \{ \} efter linjen \texttt{public static void main(String[] args) \{}, så husk at hvis ikke det står der (eller bliver refereret derfra) så bliver det ikke udført.

\begin{remark}
	Undervejs kan det være du har lyst til at skrive noter/kommentarer til dine kodelinjer. Dette er heldigvis nemt at gøre, og bruges rigtig meget i virkeligheden. I Java er der to slags kommentarer. En en-linjes kommentar starter med \texttt{//} og gør resten af linjen til en kommentar, det vil altså sige at det ikke bliver "set" af computeren som en del af programmet. En fler-linjes kommentar starter med \texttt{/*} og slutter med \texttt{*/}. Et par eksempler på kommentarer kan ses i Listing \ref{lst:comments}.
\end{remark}

\begin{JavaCode}{Eksempler på kommentarer}{lst:comments}
public class Comments {
	public static void main(String[] args) {
		/*
		  This 
		  is
		  a 
		  multiline 
		  comment
		*/
		System.out.println("Hello World!");	// This is a single line comment after some code

		// This line contains no code, only this comment
		
		/* A multiline comment doesn't need to span multiple lines */
	}
}
\end{JavaCode}

\section{Variabler}
Tit vil man gerne referere til en bestemt værdi flere gange, og måske vil man gerne ændre værdien undervejs i sit program, til det bruger man variabler. 

Det kan være nyttigt at se på en variabel som en slags papkasse, hvor man skriver udenpå, hvad for noget der er indeni, eksempelvis "penge". Ved at referere til "penge" kan man finde ud af, hvor mange penge man har i kassen. Man kan også lægge penge til dem man har i kassen, eller trække fra. Man skal dog passe på man ikke kommer til at overskrive sine penge, for kassen kan kun huske det sidste man har lagt i den.

\begin{JavaCode}{Penge eksempel kode}{lst:money}
public class Money {
	public static void main(String[] args) {
		int penge = 50;
		
		System.out.println("Jeg starter med kr " + penge);
		
		penge = penge + 20;
		
		System.out.println("Nu har jeg kr " + penge);
		
		penge = 30;

		System.out.println("Til sidst har jeg kr " + penge);
	}
}
\end{JavaCode}
\todo{Linjerne er for lange, skal de wrappes eller hvad gør vi?}

I Listing \ref{lst:money} kan man se et lille program der viser brugen af en variabel. I linje 3 opretter vi variablen, og med det samme lægger vi tallet 50 i, som repræsentation af 50 kr. Det vil også fremgå af outputtet fra programmet, fra linje 5. Java sørger for at hvis vi har noget tekst og skriver + bagved så bliver det næste også tolket som tekst og derfor sat bagved. Det der står bagved er penge, og fordi det står uden anførselstegn er det en reference til variablen (eller papkassen) med navnet penge, og når det bliver refereret bruger man så det der er gemt i den.

På linje 7 opdaterer vi, hvad der er i variablen penge. Vi refererer variablen penge og bruger = som tegn på at penge skal opdateres til hvad end der kommer bagefter. Det er altså væsentligt forskelligt fra det = vi kender fra matematik. Vi siger altså at penge skal opdateres til at være den værdi, der var i penge i forvejen og så lægge 20 til. Det vil altså sige at der gerne skulle være 70 kr i den nu, hvilket også fremgår af outputtet efterfølgende.

Til sidst demonstreres vigtigheden af at "papkassen" kun kan "huske", hvad der sidst blev lagt i den. Hvis vi lader som om vi lige har "tjent" 30 kr som vi gerne vil gemme i "papkassen", så skal vi lægge det til det der var i i forvejen. Hvis ikke vi gør det, så bliver det gamle "glemt" fordi det bliver overskrevet. Det kan ses på linje 11, hvor penge opdateres til 30. Det vil fremgå af output at vi til sidst kun har 30 kr, men måske var meningen i virkeligheden at de skulle være lagt til, så vi havde 100 kr.

\section{Typer}
Computeren skal vide, hvordan den skal forstå visse ting, f.eks. er der forskel på tekst og tal. Man siger at tekst og tal er forskellige typer. Der er nogle grundlæggende typer som man bliver nødt til at lære sig, men det er ikke så slemt, når man har brugt dem lidt.

\begin{itemize}
	\item \texttt{int} er standard typen for heltal, altså 1, 2, 3, osv. Det er forkortet af det engelske ord "integer" som betyder netop heltal.
	\item \texttt{double} er standard typen for kommatal/decimaltal, altså 0.1, 1.5, 3.14, osv. Normaltvis hed kommatal "float", men da man ønskede at kunne repræsentere flere decimaler for at øge præcisionen, lavede man en ny type med dobbelt så mange bits, dermed navnet "double".
	\item \texttt{String} er standard typen for tekst, som vi nogen gange kalder tekst-strenge. Det skyldes at tekst egentlig bare er en sekvens (streng) af enkelte tegn.
	\item \texttt{boolean} er typen for sandhedsværdier også kaldet boolske værdier, dvs. \texttt{true} eller \texttt{false}. Navnet kommer fra manden George Boole, som var den første til at formalisere denne form for logik.
\end{itemize}

Der findes flere men dette er de mest anvendte.

Java kræver at man fortæller hvilken type en variabel har, første gang man refererer den, man kan sige at det er idet man "opretter" den. Det er derfor der står \texttt{int} foran \texttt{penge} i linje 3 i Listing \ref{lst:money}. Det kan måske virke lidt besværligt i starten at man skal huske at gøre det første gang, men ikke må gøre det andre gange, men i det lange løb betyder det faktisk at Java kan hjælpe én rigtig meget, når man laver fejl. I Listing \ref{lst:types} kan du se nogle flere eksempler på oprettelse af variabler med de forskellige typer.

\begin{remark}
	Bemærk, at \texttt{String} modsat de andre typer står med stort forbogstav. Dette skyldes (lidt teknisk) at strenge er objekter og ikke primitive typer, som de andre kaldes. Som sagt er de opbygget af tegn/karakterer, disse tegn er af den primitive type kaldet \texttt{char}. Når vi senere hen skaber vores egne "typer" vil de også være skrevet med stort forbogstav.
\end{remark}

\begin{JavaCode}{Eksempler på oprettelse og anvendelse af variabler med forskellige typer}{lst:types}
public class Types {
	public static void main(String[] args) {
		int answer = 42;
		double price = 4.95;
		String name = "Bill Gates";
		boolean running = true;
		
		System.out.println("The ultimate answer to life, universe and everything: " + answer);
		System.out.println("An apple in my shop kosts: " + price);
		System.out.println("Founder of Microsoft: " + name);
		System.out.println("Is the program running? " + running);
	}
}
\end{JavaCode}

\subsection{Aritmetik/regneregler}
Her er nogle eksempler på forskellige operationer og brug af operatorer på de typer vi har set. Mange af dem virker nok ret indlysende.

\begin{itemize}
	\item Plus, minus, gange og division med heltal: \\
	\texttt{int a = 19+23;} \\
	\texttt{int b = 4-1;} \\
	\texttt{int c = 3*4;} \\
	\texttt{int d = 23/5;} Bemærk at der her bruges såkaldt heltalsdivision, hvilket vil sige at den sidste rest som 5 ikke kan dele bliver ignoreret, dermed bliver \texttt{d} i dette eksempel 4, da 4*5 giver 20 og 5 ikke kan dele de sidste 3 i hele dele.
	
	\item Plus, minus, gange og division med kommatal:\\
	\texttt{double e = 1.23+3.45;}\\
	\texttt{double f = 4.0-0.86;}\\
	\texttt{double g = 5.0*0.5;}\\
	\texttt{double h = 6.0/2.0;} Bemærk brugen af "6.0" og "2.0" for at sikre kommatals division.
	
	\item Plus mellem strenge sætter dem efter hinanden, også kaldet konkatenering.\\
	\texttt{String i = "hello " + "there";} Husk at inkludere mellemrum i en af strengene, ellers bliver de sat helt op ad hinanden.
	
	\item AND og OR mellem boolske udtryk:\\
	\texttt{boolean j = true \&\& true;} giver \texttt{true}.\\
	\texttt{boolean k = false || false;} giver \texttt{false}.
	
	\item Sammenlignings operatorer mellem tal, resulterer i en \texttt{boolean}:\\
	\texttt{boolean l = 1 < 2;}\\
	\texttt{boolean m = 3.5 >= 4.2;} giver \texttt{false}.\\
	\texttt{boolean n = 5 == 5;} dette er en præcis sammenlignings operator, altså er udtrykket kun \texttt{true} når de to ting er præcis lig med hinanden. Bruges ikke til sammenligninger mellem strenge, der bruges istedet \texttt{.equals()}. Det vender vi tilbage til senere.\\
	\texttt{boolean o = 1.5 != 2.3;} betyder det modsatte af \texttt{==} altså "ikke lig med" eller "forskellig fra".
\end{itemize}

\todo{Ovenstående skal måske anvende noget JavaCode i stedet...}

\section{Logik}
Nu hvor vi har lært om typen \texttt{boolean}, så kan vi lære om forskellige logiske udtryk. Man kan f.eks. få en boolsk værdi ved at "spørge", om 1 er større end 2, hvilket vi ved er falsk, så den boolske værdi er \texttt{false}. Det smarte er, at når vi sammensætter forskellige logiske udtryk, så får vi et nyt logisk udtryk som til sidst giver en boolsk værdi. Vi sammensætter som regel enten med AND eller OR, i Java repræsenteret med hhv. \texttt{\&\&} og \texttt{||}. Husk at man også her kan bruge variabler, f.eks. kan det være man gerne vil vide om man skal købe et hus, så man spørger om man har penge nok og om man i forvejen har nok huse. 

\begin{JavaCode}{Logik}{lst:logik}
	boolean buyHouse = money > 1000000 && houses < 1;
\end{JavaCode}

Listing \ref{lst:merelogik} kunne være et eksempel på brugen af udtrykket i Listing \ref{lst:logik}.

\begin{JavaCode}{Anvendelse af logik fra Listing \ref{lst:logik}}{lst:merelogik}
public class Logic {
	public static void main(String[] args) {
		int money = 2000000;
		int houses = 0;
		boolean buyHouse = money > 1000000 && houses < 1;
		System.out.println("Should I buy a house? " + buyHouse);
	}
}
\end{JavaCode}

\begin{remark}
	For at et udtryk sammensat med AND kan være sandt skal begge sider være sande, dette betyder faktisk at hvis den venstre side (som bliver evalueret først i programmet) er falsk, bliver den højre side ikke evalueret.
	
	Et udtryk sammensat med OR er sandt, hvis bare én af siderne er sande, derfor gælder det tilsvarende, at hvis venstre side er sand, bliver den højre side ikke evalueret.
\end{remark}

\section{if-else sætninger}
Når man har styr på logiske udtryk kan man bruge dem til at vælge, hvilken vej i programmet computeren skal gå. Man siger simpelthen at \textbf{hvis} et eller andet er sandt så vil man gøre noget \textbf{ellers} vil man gøre noget andet. F.eks. \textbf{hvis} man har penge nok så vil man gerne købe en bil \textbf{ellers} vil man tjene flere penge. Et eksempel på hvordan det kunne skrives er vist i Listing \ref{lst:if-intro}.

\begin{JavaCode}{If-else statement}{lst:if-intro}
	if (money > 100000) {
		System.out.println("Buy the car!");
	} else {
		System.out.println("You need to earn more money.");
	}
\end{JavaCode}

Jeg håber du kan se at denne mulighed skaber mange flere muligheder for at skrive interessante programmer end tidligere, da vi pludselig kan tage beslutninger baseret på "ukendt" input. If-else sætninger kan både gøres kortere og længere, hvis man kun vil gøre en ekstra ting, i et specialtilfælde, eller hvis man har brug for at vælge mellem flere ting, begge dele kan ses i Listing \ref{lst:elseif}.

\begin{JavaCode}{Eksempler på if-sætninger og else-if-sætninger}{lst:elseif}
if (money > 100000) {
	System.out.println("Buy the car!");
}

if (age >= 70) {
	System.out.println("You are old.");
} else if ( age >= 30) {
	System.out.println("You are an adult.");
} else {
	System.out.println("You are young.");
}
\end{JavaCode}

\section{Metoder}
Nogen gange har man noget kode som man gerne vil genbruge flere steder i sin kode, så kan det være en god idé at lave det til en såkaldt funktion. I Java er alle funktioner også kaldet metoder, så det er ikke så vigtigt, hvad man siger. Heldigvis er der nogen der har lavet nogle brugbare for os, så vi kan lære, hvordan man bruger dem, før vi selv skal lære at lave dem. 

Kendetegnende for metoder er, at der er parenteser efter navnet. Det man skriver inden i parentesen kaldes for argumenter (det kan også ske nogle kalder dem for parametre), og er input som metoden skal bruge til arbejde med. En metode som vi har set nogle gange er print-metoden. Det, den tager som input-argument, er den tekst-streng vi gerne vil have skrevet ud på skærmen.

\begin{JavaCode}{Print-metoden, tager en tekst-streng som argument/input.}{lst:helloagain}
System.out.println("Hello World!");
\end{JavaCode}

Tidligere nævnte vi også at at for at sammenligne strenge skulle man helst skrive \texttt{.equals()}. Dette er også et eksempel på en metode, igen skal man give en tekst-streng som argument. I Listing \ref{lst:equals} er et eksempel på brugen af \texttt{.equals()}-metoden.

\begin{JavaCode}{Eksempel på brug af \texttt{.equals()}}{lst:equals}
String x = "Horse";
if (x.equals("Horse")) {
	System.out.println("Yeehaaw!");
}
\end{JavaCode}

\section{Fejl og exceptions}
Det er menneskeligt at fejle, og programmering er et sted, hvor det er umuligt at undgå. Selv verdens bedste programmør vil lave fejl en gang imellem. Overordnet findes der tre typer fejl.

\begin{enumerate}
	\item Syntaks fejl
	\item Run-time fejl
	\item Semantiske fejl
\end{enumerate}

\subsection{Syntaks fejl}
Syntaks handler om den meget bestemte måde Java kræver man skriver programmer. F.eks. skal man huske semikolon efter en endt instruktion og en if-sætning skal have et boolsk udtryk imellem parenteserne. Syntaks i et Java program bliver tjekket når programmet kompileres (og i visse programmer undervejs mens man skriver), hvilket betyder at man slet ikke kan køre et Java program, der er syntaktisk ukorrekt. En syntaks fejl I næsten garanteret kommer til at lave på et tidspunkt er at skrive noget i retningen af \texttt{if (x = y) \{}, hvor I skulle have brugt \texttt{==} til at sammenligne. Den fejl kan også siges at være en semantisk fejl.

\subsection{Run-time fejl}
Run-time fejl er de fejl, der sker mens programmet kører. Et godt eksempel er hvis man f.eks. har skrevet noget lignende \texttt{int x = 100 / y;}, men inden denne linje udføres er \texttt{y} blevet lig med 0 og division er derfor udefineret, det skal selvfølgelig resultere i en fejl. Alle run-time fejl kaldes exceptions og Java tilbyder faktisk mulighed for at håndtere dem, hvis man ønsker det. Dette gøres med en try-catch konstruktion, se et eksempel i Listing \ref{lst:trycatch}

\begin{JavaCode}{Eksempel på try-catch}{lst:trycatch}
public class TryCatch {
	public static void main(String[] args) {
		int y = 0;
		try {
			int x = 100 / y;
			System.out.println("Result: " + x); 
		} catch (Exception e) {
			System.out.println("An error occured");
			System.out.println(e);
		}
	}
}

\end{JavaCode}

\begin{remark}
	Som du kan se printer vi \texttt{e} på linje 9, det er selve fejlen, som er blevet fanget, defineret i linje 7. Det er muligt at definere sine egne fejl-typer (exceptions), hvis man laver mere komplicerede programmer.
	
	Det er også muligt at forlænge sin try-catch konstruktion med en finally blok. Det er blot kode man vil køre uanset om der blev fanget en exception eller ej.
\end{remark}

\subsection{Semantiske fejl}
Semantiske fejl er den slags fejl som ikke kan fanges af computeren, ligesom de to andre slags fejl kan. Det handler om at man ikke har skrevet præcist det man mente. Som tidligere nævnt kan et eksempel være at man har lavet en "sammenligning" med \texttt{=} istedet for \texttt{==} og derfor laver man en tildeling af værdier istedet for sammenligning. Et andet eksempel som vi tidligere har stødt på, er hvor man opdaterer en variabel med en fast værdi, i stedet for at lægge til. 

Det er i virkeligheden en slags semantisk fejl der ligger bag joken om konen der beder manden gå ned og købe to mælk, og hvis der er æg, så skal han købe ti. Da han kommer tilbage med ti mælk, spørger konen forvirret, hvorfor han har ti mælk, hvortil han svarer "der var æg".

%\blindtext
%\marginfigure{sample}{Some margin note here.}
%
%\bottomfigure{sample}{Some bottom note here.}
%
%
%\begin{JavaCode}{The Fish class}{lst:fish-class}
%	public class Fish {
%		public static void main(String[] args) {
%			doAThing();
%			int a = 32;
%			String fish = tish;
%			Abe b = new Abe(fish, a, "AbeMand");
%			return;
%		}
%	}
%\end{JavaCode}
%\marginnote{Sej note her}
%
%\blindtext
%
%\section{Boxes}
%
%\subsection{Example}
%\begin{example}
%	An example here
%\end{example}
%
%\subsection{Exercise}
%\begin{exercise}
%	An exercise here
%\end{exercise}
%
%\subsection{Remark}
%\begin{remark}
%	Some remark here
%\end{remark}
	\fi
	
	\graphicspath{{sections/java/1/figures/}}
	\chapter{Introduktion til Java}
Java er det programmeringssprog som I vil lære at bruge til at programmere apps med. Det er vidt anvendeligt. Det har i mange år været, og er stadig, det mest anvendte programmeringssprog på verdensplan.

Det kan være lidt besværligt at komme igang med Java, men vi vil gøre vores bedste for at lære dig alle de ting du har brug for, for at komme grundigt igang.

\todo{Addresserer vi "du/dig" eller "I/jer?"}

\subsubsection{Hvad er et program, egentlig?}
Et program er en serie af instruktioner, som skal udføres i rækkefølge. Typisk skrives instruktionerne som helt almindelig tekst, hvor hver instruktion står på sin egen linje.

\section{Sådan skriver du programmer}
Når du skal skrive programmer, kan du i princippet gøre det i næsten hvilket som helst skriveprogram, som ikke er Word, eksempelvis Notesblok på Windows. 

Det kan dog være lidt besværligt, og derfor har nogle valgt at udvikle programmer specielt til at skrive kode i, som kan hjælpe med alt muligt smart. 

Et program I kommer til at bruge til at skrive Java-kode i, er IntelliJ. Det kan blandt mange andre ting hjælpe med, at man får skrevet gyldige Java-programmer, og køre dit program ved et enkelt klik på en knap.

\begin{remark}
	Hvis du har lyst til en lille udfordring, så prøv at skriv et program i notesblok eller tilsvarende, og kør programmet ved at bruge en terminal (kaldet kommandoprompt på Windows). 
	
	Dette gør du ved først at navigere til den mappe, hvor du har gemt dit program ved hjælp af kommandoen \texttt{cd} (change directory), skriv \texttt{cd test}, hvis du vil ind i mappen, der hedder test, \texttt{cd ..} går et mappeniveau op. Hvis du gerne vil se hvilke mapper, du har i den mappe du er i, så brug kommandoen \texttt{dir} på Windows, eller \texttt{ls} på de fleste andre systemer. 
	
	Når du er i den rigtige mappe, skal du kompilere dit program ved at skrive \texttt{javac Test.java} (c'et er for compile), hvis dit program hedder "Test" (det er vigtigt at Java-programmer slutter på ".java"), dette laver en fil ved navn \texttt{Test.class}, som er dit færdige program. 
	
	Til sidst kan du køre programmet ved at skrive \texttt{java Test}.
\end{remark}

\todo{Anførselstegn fjerner mellemrum efterfølgende?}

\subsection{Hello World!}
Et "Hello World!" program er ofte det første, man prøver i et nyt programmeringssprog, for at få en lille smule føling med, hvordan sproget skrives.

\begin{JavaCode}{Et Hello World program i Java}{lst:helloworld}
	public class Hello {
		public static void main(String[] args) {
			System.out.println("Hello World!");
		}
	}
\end{JavaCode}

Til at begynde med ser dette måske lidt skræmmende ud, men bare rolig, vi guider dig igennem det hele, skridt for skridt. Hvis du har fået programmet i \autoref{lst:helloworld} til at fungere, vil det sige \texttt{Hello World!} som output, men hvis du synes, det er lidt kedeligt, skal du bare ændre teksten imellem anførselstegnene på linje 3, til noget du synes er sjovere.

\begin{remark}
	Husk at programmet skal gemmes som "Test.java" med stort forbogstav, og skal hedde det samme som det, der står på linjen, hvor der står \JavaInline|public class Test {|
\end{remark}

Det eneste, der sker i dit program, er præcis de ting, der står mellem \{ \} efter linjen \JavaInline|public static void main(String[] args) {|, så husk, at hvis ikke det står der (eller bliver refereret derfra), så bliver det ikke udført.

\begin{remark}
	Undervejs kan det være, du har lyst til at skrive noter/kommentarer til dine kodelinjer. Dette er heldigvis nemt at gøre, og bruges rigtig meget i virkeligheden. I Java er der to slags kommentarer. En en-linjes kommentar starter med \JavaInline|//| og gør resten af linjen til en kommentar, det vil altså sige at det ikke bliver "set" af computeren som en del af programmet. En fler-linjes kommentar starter med \JavaInline|/*| og slutter med \JavaInline|*/|. Et par eksempler på kommentarer kan ses i \autoref{lst:comments}.
\end{remark}

\begin{JavaCode}{Eksempler på kommentarer}{lst:comments}
	public class Comments {
		public static void main(String[] args) {
			/*
			This 
			is
			a 
			multiline 
			comment
			*/
			
			// This line contains no code, only this comment
			
			System.out.println("Hello World!");	// Comment
			
			/* A multiline comment can be on a single line */
		}
	}
\end{JavaCode}

\begin{exercise}
	Lav et program der printer jeres navne på skærmen. 
	Tip: Man kan lave linjeskift med \textbackslash n
\end{exercise}

\section{Variabler}
Tit vil man gerne referere til en bestemt værdi flere gange, og måske vil man gerne ændre værdien undervejs i sit program. Til det bruger man variabler. 

Det kan være nyttigt, at se på en variabel som en slags papkasse, hvor man skriver udenpå, hvad for noget, der er indeni, eksempelvis "penge". Ved at referere til "penge", kan man finde ud af, hvor mange penge, man har i kassen. Man kan også lægge penge til dem man har i kassen, eller trække fra. Man skal dog passe på, man ikke kommer til at overskrive sine penge, for kassen kan kun huske det sidste man har lagt i den.

\begin{JavaCode}{Penge eksempel kode}{lst:money}
public class Money {
	public static void main(String[] args) {
		int penge = 50;
		
		System.out.println("Jeg starter med kr " + penge);
		
		penge = penge + 20;
		
		System.out.println("Nu har jeg kr " + penge);
		
		penge = 30;
		
		System.out.println("Til sidst har jeg kr " + penge);
	}
}
\end{JavaCode}
\todo{Linjerne er for lange, skal de wrappes eller hvad gør vi?}

I \autoref{lst:money} kan man se et lille program, der viser brugen af en variabel. I linje 3 opretter vi variablen, og med det samme lægger vi tallet 50 i, som repræsentation af 50 kr. Det vil også fremgå af outputtet fra programmet, fra linje 5. Java sørger for, at hvis vi har noget tekst og skriver + bagved, så bliver det næste også tolket som tekst og derfor sat bagved. Det, der står bagved er penge, og fordi det står uden anførselstegn, er det en reference til variablen (eller papkassen) med navnet penge, og når det bliver refereret, bruger man så det, der er gemt i den.

På linje 7 opdaterer vi, hvad der er i variablen penge. Vi refererer variablen penge og bruger = som tegn på, at penge skal opdateres til, hvad end der kommer bagefter. Det er altså væsentligt forskelligt fra det =, vi kender fra matematik. Vi siger altså, at penge skal opdateres til, at være den værdi, der var i penge i forvejen og så lægge 20 til. Det vil altså sige, at der gerne skulle være 70 kr i den nu, hvilket også fremgår af outputtet efterfølgende.

Til sidst demonstreres vigtigheden af, at "papkassen" kun kan "huske", hvad der sidst blev lagt i den. Hvis vi lader som om vi lige har "tjent" 30 kr, som vi gerne vil gemme i "papkassen", så skal vi lægge det til det, der var i i forvejen. Hvis ikke vi gør det, så bliver det gamle "glemt", fordi det bliver overskrevet. Det kan ses på linje 11, hvor penge opdateres til 30. Det vil fremgå af output, at vi til sidst kun har 30 kr, men måske var meningen i virkeligheden, at de skulle være lagt til, så vi havde 100 kr.

\subsection{Typer}
Computeren skal vide, hvordan den skal forstå visse ting, f.eks. er der forskel på tekst og tal. Man siger, at tekst og tal er forskellige typer. Der er nogle grundlæggende typer, som man bliver nødt til at lære sig, men det er ikke så slemt, når man har brugt dem lidt.

\begin{itemize}
	\item \JavaInline|int| er standard typen for heltal, altså 1, 2, 3, osv. Det er en forkortelse af det engelske ord "integer" som netop betyder heltal.
	\item \JavaInline|double| er standard typen for kommatal/decimaltal, altså 0.1, 1.5, 3.14 osv. Normalt hed kommatal "float", men da man ønskede at kunne repræsentere flere decimaler for at øge præcisionen, lavede man en ny type med dobbelt så mange bits, dermed navnet "double".
	\item \JavaInline|String| er standard typen for tekst, som vi nogen gange kalder tekst-strenge. Det skyldes at tekst egentlig bare er en sekvens (streng) af enkelte tegn.
	\item \JavaInline|boolean| er typen for sandhedsværdier også kaldet boolske værdier, dvs. \JavaInline|true| eller \JavaInline|false|. Navnet kommer fra manden George Boole, som var den første til at formalisere denne form for logik.
\end{itemize}

Der findes flere, men dette er de mest anvendte.

Java kræver, at man fortæller, hvilken type en variabel har første gang, man refererer til den, man kan sige at det er idet, man "opretter" den. Det er derfor, der står \JavaInline|int| foran \JavaInline|penge| i linje 3 i \autoref{lst:money}. Det kan måske virke lidt besværligt i starten, at man skal huske at gøre det første gang, men ikke må gøre det andre gange, men i det lange løb betyder det faktisk, at Java kan hjælpe én rigtig meget, når man laver fejl. I \autoref{lst:types} kan du se nogle flere eksempler på oprettelse af variabler med de forskellige typer.

\begin{remark}
	Bemærk, at \JavaInline|String| modsat de andre typer står med stort forbogstav. Dette skyldes (lidt teknisk) at strenge er objekter og ikke primitive typer, som de andre kaldes. Som sagt er de opbygget af tegn/karakterer, disse tegn er af den primitive type kaldet \JavaInline|char|. Når vi senere hen skaber vores egne "typer", vil de også være skrevet med stort forbogstav.
\end{remark}

\begin{JavaCode}{Eksempler på oprettelse og anvendelse af variabler med forskellige typer.}{lst:types}
	public class Types {
		public static void main(String[] args) {
			int answer = 42;
			double price = 4.95;
			String name = "Bill Gates";
			boolean running = true;
			
			System.out.println("An apple in my shop kosts: " 
						+ price);
			System.out.println("Founder of Microsoft: " + name);
			System.out.println("Is the program running? " 
						+ running);
			
			System.out.println("The ultimate answer to life, "
						+ "universe and everything: " 
						+ answer);
		}
	}
\end{JavaCode}

\begin{exercise}
	Opret variable for værdierne 10, 10.6, 'a', true og "UNF". Kør jeres kode for at tjekke det virker.
	Note: brug ikke nogen access modifiers, såsom private eller public
\end{exercise}

\subsection{Aritmetik/regneregler}
Her er nogle eksempler på forskellige operationer og brug af operatorer på de typer, vi har set. Mange af dem virker nok ret indlysende.

\begin{itemize}
	\item Plus, minus, gange og division med heltal: \\
	\JavaInline|int a = 19+23;| \\
	\JavaInline|int b = 4-1;| \\
	\JavaInline|int c = 3*4;| \\
	\JavaInline|int d = 23/5;| Bemærk at der her bruges såkaldt heltalsdivision, hvilket vil sige at den sidste rest, som \JavaInline|5| ikke kan dele, bliver ignoreret, dermed bliver \JavaInline|d| i dette eksempel \JavaInline|4|, da \JavaInline|4*5| giver \JavaInline|20| og \JavaInline|5| ikke kan dele de sidste 3 i hele dele.
	
	\item Plus, minus, gange og division med kommatal:\\
	\JavaInline|double e = 1.23+3.45;|\\
	\JavaInline|double f = 4.0-0.86;|\\
	\JavaInline|double g = 5.0*0.5;|\\
	\JavaInline|double h = 6.0/2.0;| Bemærk brugen af "6.0" og "2.0" for at sikre kommatals division.
	
	\item Plus mellem strenge sætter dem efter hinanden, også kaldet konkatenering.\\
	\JavaInline|String i = "hello " + "there";| Husk at inkludere mellemrum i en af strengene, ellers bliver de sat helt op ad hinanden.
	
	\item AND og OR mellem boolske udtryk:\\
	\JavaInline|boolean j = true && true;| giver \JavaInline|true|.\\
	\JavaInline?boolean k = false || false;? giver \JavaInline|false|.
	
	\item Sammenlignings operatorer mellem tal, resulterer i en \JavaInline|boolean|:\\
	\JavaInline|boolean l = 1 < 2;|\\
	\JavaInline|boolean m = 3.5 >= 4.2;| giver \JavaInline|false|.\\
	\JavaInline|boolean n = 5 == 5;| dette er en præcis sammenlignings operator, altså er udtrykket kun \JavaInline|true|, når de to ting er præcis lig med hinanden. Bruges ikke til sammenligninger mellem strenge, der bruges istedet \JavaInline|.equals()|. Det vender vi tilbage til senere.\\
	\JavaInline|boolean o = 1.5 != 2.3;| betyder det modsatte af \JavaInline|==| altså "ikke lig med" eller "forskellig fra".
\end{itemize}

\todo{Ovenstående skal måske anvende noget JavaCode i stedet...}

\begin{exercise}
	Opret 3 variable a, b og c som heltal. Lav nu en ny variabel d som har værdien af resultatet hvis man ganger a med b, lægger c til og trækker a fra. 
	Print d så i kan se om den gør som i forventer.
\end{exercise}

\begin{exercise}
	Lav to variable hight og weight som reelle tal. Beregn BMI og print dette.
	Hint: BMI regnes ved at tage ens vægt i kg og dividere med højde i meter to gange.
\end{exercise}

\subsection{Logik}
Nu hvor vi har lært om typen \JavaInline|boolean|, så kan vi lære om forskellige logiske udtryk. Man kan f.eks. få en boolsk værdi ved at "spørge", om 1 er større end 2, hvilket vi ved er falsk, så den boolske værdi er \JavaInline|false|. Det smarte er, at når vi sammensætter forskellige logiske udtryk, så får vi et nyt logisk udtryk, som til sidst giver en boolsk værdi. Vi sammensætter som regel enten med AND eller OR, i Java repræsenteret med hhv. \JavaInline|&&| og \JavaInline?||?. Husk at man også her, kan bruge variabler, f.eks. kan det være, man gerne vil vide, om man skal købe et hus, så man spørger, om man har penge nok og om man i forvejen har nok huse. 

\begin{JavaCode}{Logik}{lst:logik}
	boolean buyHouse = money > 1000000 && houses < 1;
\end{JavaCode}

\autoref{lst:merelogik} kunne være et eksempel på brugen af udtrykket i \autoref{lst:logik}.

\begin{JavaCode}{Anvendelse af logik fra \autoref{lst:logik}}{lst:merelogik}
	public class Logic {
		public static void main(String[] args) {
			int money = 2000000;
			int houses = 0;
			boolean buyHouse = money > 1000000 && houses < 1;
			System.out.println("Should I buy a house? " 
						+ buyHouse);
		}
	}
\end{JavaCode}

\begin{remark}
	For at et udtryk sammensat med AND kan være sandt, skal begge sider være sande, dette betyder faktisk, at hvis den venstre side (som bliver evalueret først i programmet) er falsk, bliver den højre side ikke evalueret.
	
	Et udtryk sammensat med OR er sandt, hvis bare én af siderne er sande, derfor gælder det tilsvarende, at hvis venstre side er sand, bliver den højre side ikke evalueret.
\end{remark}

\section{if-else sætninger}
Når man har styr på logiske udtryk, kan man bruge dem til at vælge, hvilken vej i programmet computeren skal gå. Man siger simpelthen, at \textbf{hvis} et eller andet er sandt, så vil man gøre noget, \textbf{ellers} vil man gøre noget andet. F.eks. \textbf{hvis} man har penge nok, så vil man gerne købe en bil, \textbf{ellers} vil man tjene flere penge. Et eksempel på, hvordan det kunne skrives, er vist i \autoref{lst:if-intro}.

\begin{JavaCode}{If-else statement}{lst:if-intro}
	if (money > 100000) {
		System.out.println("Buy the car!");
	} else {
		System.out.println("You need to earn more money.");
	}
\end{JavaCode}

Jeg håber, du kan se at denne mulighed skaber mange flere muligheder for, at skrive interessante programmer end tidligere, da vi pludselig kan tage beslutninger baseret på "ukendt" input. If-else sætninger kan både gøres kortere og længere, hvis man kun vil gøre en ekstra ting, i et specialtilfælde, eller hvis man har brug for at vælge mellem flere ting, begge dele kan ses i \autoref{lst:elseif}.

\begin{JavaCode}{Eksempler på if-sætninger og else-if-sætninger.}{lst:elseif}
	if (money > 100000) {
		System.out.println("Buy the car!");
	}
	
	if (age >= 70) {
		System.out.println("You are old.");
	} else if ( age >= 30) {
		System.out.println("You are an adult.");
	} else {
		System.out.println("You are young.");
	}
\end{JavaCode}

\begin{exercise}
	Gammel nok til at komme ind på diskotekerne i Gaden? lav en variabel: alder, og brug en if-else sætning til at printe ja hvis alderen er høj nok, ellers nej.
\end{exercise}

\begin{exercise}
	Hvad hvis man er i USA? brug variablen alder og en if-else sætning til at printe ja hvis alderen er høj nok, ellers nej.
\end{exercise}

\begin{exercise}
	Hvad hvis man er tre som går i gaden sammen og "danser"?
	Kan alle så komme ind? lav 3 variable til at holde alder som heltal og print ja hvis alle må komme ind, ellers nej.
\end{exercise}

\section{Metoder}
Nogen gange har man noget kode, som man gerne vil genbruge flere steder i sin kode, så kan det være en god idé, at lave det til en såkaldt funktion. I Java er alle funktioner også kaldet metoder, så det er ikke så vigtigt, hvad man siger. Heldigvis er der nogen, der har lavet nogle brugbare metoder for os, så vi kan lære, hvordan man bruger dem, før vi selv skal lære at lave dem. 

Kendetegnene for metoder er, at der er parenteser efter navnet. Det man skriver inden i parentesen kaldes for argumenter (det kan også ske, nogle kalder dem for parametre), og de er input, som metoden skal bruge til at arbejde med. En metode, som vi har set nogle gange, er print-metoden. Det, den tager som input-argument, er den tekst-streng, vi gerne vil have skrevet ud på skærmen.

\begin{JavaCode}{Print-metoden tager en tekst-streng som argument/input.}{lst:helloagain}
	System.out.println("Hello World!");
\end{JavaCode}

Tidligere nævnte vi også at for at sammenligne strenge skulle man helst skrive \JavaInline|.equals()|. Dette er også et eksempel på en metode, hvor man igen skal give en tekst-streng som argument. I \autoref{lst:equals} er et eksempel på brugen af \JavaInline|.equals()|-metoden.

\begin{JavaCode}{Eksempel på brug af \texttt{.equals()}}{lst:equals}
	String x = "Horse";
	if (x.equals("Horse")) {
		System.out.println("Yeehaaw!");
	}
\end{JavaCode}

\pagebreak

\begin{exercise}
	I en underlig fantasiverden kan man komme ind på diskotek uanset alder så længe man er flot.
	Variable: En meget fiktiv persons alder og om vedkommende er flot
	Print hvorvidt vedkommende kan komme ind på diskotek i vores fantasiverden.
\end{exercise}

\begin{exercise}
	I en anden underlig fantasiverden kan man kun komme ind på diskotek hvis man er flot og endnu ikke er fyldt 40.
	Variable: En meget fiktiv persons alder og om vedkommende er flot
	Print hvorvidt vedkommende kan komme ind på diskotek i vores anden fantasiverden.
\end{exercise}


\begin{exercise}
	I en irriterende verden som kun kan eksistere i hovedet på en gal faglig hjælper kan man kun komme ind på diskotek hvis man:
	
	- endnu ikke er fyldt 50
	
	- ikke er præcis 25 år
	
	- er fyldt 18
	eller
	
	er fyldt 16 hvis bare man er flot
	
	\noindent Variable: En meget fiktiv persons alder og om vedkommende er flot
	Print om vedkommende kan komme ind på diskotek i den irriterende verden
\end{exercise}

\section{Fejl og exceptions}
Det er menneskeligt at fejle, og programmering er et sted, hvor det er umuligt at undgå. Selv verdens bedste programmør vil lave fejl en gang imellem. Overordnet findes der tre typer fejl.

\begin{enumerate}
	\item Syntaks fejl
	\item Run-time fejl
	\item Semantiske fejl
\end{enumerate}

\subsection{Syntaks fejl}
Syntaks handler om den meget bestemte måde, Java kræver, man skriver programmer på. F.eks. skal man huske semikolon efter en endt instruktion, og en if-sætning skal have et boolsk udtryk imellem parenteserne. Syntaks i et Java program bliver tjekket når programmet kompileres (og i visse programmer undervejs mens man skriver), hvilket betyder, at man slet ikke kan køre et Java program, der er syntaktisk ukorrekt. En syntaks fejl I næsten garanteret kommer til at lave på et tidspunkt, er at skrive noget i retning af \JavaInline|if (x = y) {|, hvor I skulle have brugt \JavaInline|==| til at sammenligne. Den fejl kan også siges at være en semantisk fejl.

\subsection{Run-time fejl}
Run-time fejl er de fejl, der sker mens programmet kører. Et godt eksempel, er hvis man f.eks. har skrevet noget lignende \JavaInline|int x = 100 / y;|, men inden denne linje udføres, er \JavaInline|y| blevet lig med 0 og division er derfor udefineret, det skal selvfølgelig resultere i en fejl. Alle run-time fejl kaldes exceptions, og Java tilbyder faktisk mulighed for at håndtere dem, hvis man ønsker det. Dette gøres med en try-catch konstruktion, se et eksempel i \autoref{lst:trycatch}

\begin{JavaCode}{Eksempel på try-catch}{lst:trycatch}
	public class TryCatch {
		public static void main(String[] args) {
			int y = 0;
			try {
				int x = 100 / y;
				System.out.println("Result: " + x); 
			} catch (Exception e) {
				System.out.println("An error occured");
				System.out.println(e);
			}
		}
	}
	
\end{JavaCode}

\begin{remark}
	Som du kan se, printer vi \JavaInline|e| på linje 9, det er selve fejlen, som er blevet fanget, defineret i linje 7. Det er muligt at definere sine egne fejl-typer (exceptions), hvis man laver mere komplicerede programmer.
	
	Det er også muligt at forlænge sin try-catch konstruktion med en finally blok. Det er blot kode, man vil køre uanset om der blev fanget en exception eller ej.
\end{remark}

\subsection{Semantiske fejl}
Semantiske fejl er den slags fejl, som ikke kan fanges af computeren, ligesom de to andre slags fejl kan. Det handler om, at man ikke har skrevet præcist det, man mente. Som tidligere nævnt kan et eksempel være at man har lavet en "sammenligning" med \JavaInline|=| istedet for \JavaInline|==|, og derfor laver man en tildeling af værdier istedet for en sammenligning. Et andet eksempel, som vi tidligere har stødt på, er hvor man opdaterer en variabel med en fast værdi i stedet for at lægge den til. 

Det er i virkeligheden en slags semantisk fejl, der ligger bag joken om konen, der beder manden gå ned og købe to mælk, og hvis der er æg, så skal han købe ti. Da han kommer tilbage med ti mælk, spørger konen forvirret, hvorfor han har ti mælk, hvortil han svarer "der var æg".

\begin{exercise}
	Valutaomregner: 100 DKK er 15.3626 USD, 100 DKK er 13.4476541 EUR og 100 USD er 87.535014 EUR.
	Variable er et beløb, den valuta beløbet er i og den valuta som beløbet skal omregnes til.
	Print resultatet.
\end{exercise}
	
	\graphicspath{{sections/java/2/figures/}}
	
%%%%%%%%%%%%%%%%%%%%%%%%%%%%%%%%%%%%%%%%%%%%%%%%%%%%%%%%%%%%%%%%%%%%%%%%%%%%%%%%

\chapter{Kontrol og Metoder}

	I sidste kapitel blev variabler, aritmetiske udtryk og \emph{if-statements}
	introduceret. Med disse dele kan man lave det der heder et
	\emph{straight-line program}. Men vi er ofte stillede over for problemer som
	ikke kan løses ekslusivt med sådanne programmer. Vi behøver mere
	udtrykskraft, end hvad aritmetik og basal beslutningsevner kan give os.
	For eksempel vil vi gerne kunne sige ting som ``gør \emph{det her} indtil
	\emph{dette} sker'' eller ``gør \emph{dette}, først for \emph{1}, og så for
	\emph{2}, osv.''

	Af denne grund indeholder mange programmerings-sprog, inklusiv Java,
	mekanismer for at kunne udtrykke sådanne problemer, og det er disse
	mekanismer vi vil dække i dette kapitel.

	Emner i dette kapitel:

	\begin{itemize} % Denne udgiver har også følgende selvhjælps-bøger på markedet:
		\item Løkker: Sådan får du mere kontrol
		\item Funktioner: Forbedre dine metoder
		\item Algoritmer: Find din indre effektivitet
	\end{itemize}

\section{Løkker}

	\subsection{While-løkker}

		\JavaInline{while} er den simpleste af løkkerne. Dette er ligesom at
		sige: ``Mens at A gælder, sørg for at gøre B.'' For eksempel:
		``Mens at gulvet er beskidt, bør du feje gulvet.''

		Dette er i modsætning til \JavaInline{if}-udstryk, hvor der udtrykkes
		``Hvis A gælder, gør B èn enkelt gang.''

		\todo{ Skal fyldes mere på. Sektion: \JavaInline{while} }

		\todo{ En illustration som denne kunne være brugbar:
		\url{https://upload.wikimedia.org/wikipedia/commons/4/43/While-loop-diagram.svg} }

		Anatomien af en \JavaInline{while}-udstryk er meget lig det af et
		\JavaInline{if}-udstryk, du skifter bare \JavaInline{if} ud med
		\JavaInline{while}, som kan ses i eksempel~\ref{lst:while-example-1}.

		\begin{JavaCode}{``Fej Gulvet'' pseudo-Java.}{lst:while-example-1}
			while (gulvet er beskidt) {
				fej gulvet;
			}
		\end{JavaCode}

		Man kalder koden mellem paranteserne for while-udtrykkets ``betingelse'',
		koden mellem tuborg-klammerne for ``kroppen'' af
		\JavaInline{while}-udtrykket, og man kalder hver gennemgang af kroppen
		for en ``iteration''.

		Nu kan man spørge: ``Hvis jeg kan lave et program som aldrig stopper, en
		såkaldt uendelig løkke?'' Ja, det er faktisk ret nemt. Så nemt at mange
		kommer til at lave en ueønsket uendelig løkke, nu og da. Se
		eksempel~\ref{lst:while-hello-world}
		for et program som gentagende skriver ``hej''. Det er selvfølgelig ikke
		så brugbart et program, men det viser en problematik: Hvad gør vi hvis
		vores program aldrig stopper? \todo{Svar på ``Hvad gør vi hvis
		vores program aldrig stopper?''} \footnote{Dette stiller selvfølgelig spørgsmålet ``kan man lave et program som kan sige om et program stopper eller ej?''. Det er muligt at lave et program, som svarer for \emph{nogle} programmer, men umuligt at lave en som kan svare for \emph{alle} programmer. Denne problematik kaldes for \emph{The Halting Problem}, og blev bevist af \emph{Allan Turing}.}

		\begin{JavaCode}{Et program der aldrig ender}{lst:while-hello-world}
			System.out.println("Hello World!");
			while (true) {
				System.out.println("Hello again!");
			}
		\end{JavaCode}

		 Oftests vil man bruge
		en betingelse som kan variere mellem iterationerne. I
		eksempel~\ref{lst:while-example-3} kan ses hvordan man kan tælle op fra
		1 til 10.

		\begin{JavaCode}{Et ``brugbart'' while program}{lst:while-example-3}
			int n = 1;
			while (n <= 10) {
				System.out.println(n + " linjer af ligegyldig data.");
				n++;
			}
		\end{JavaCode}



	\subsection{For-løkker}

		I sidste sektion så vi kan tælle op med et \JavaInline{while}-udtryk.
		Dette er da ikke en idéel måde at tælle på, da \JavaInline{n}
		defineres, sammenlignes og opdateres på helt forskellige steder i koden.
		Det kan blive svært at holde styr på i store programmer, hvilket er
		hvorfor Java har \JavaInline{for}-løkken, som samler den information i
		èt sted: \JavaInline{for (before_loop; condition_of_loop; after_every_iteration)}

		\todo{ Uddyb og udfyl. Sektion: \JavaInline{for} }


		For strukturen af for-løkker er ret involveret, men det hjælper at
		sammenligne med \JavaInline{while}-udstryk, og se hvordan at forskellige
		dele af koden som tidligere var sprædt, nu er samlet.

		\begin{JavaCode}{Et ``brugbart'' for program. Sammenlign med eksempel~\ref{while-example-3}.}{lst:for-example-1}
			for (int n = 1; n <= 10; n++) {
				System.out.println(n + " linjer af ligegyldig data.");
			}
		\end{JavaCode}

		\begin{exercise}
			Det er ofte fortalt hvordan at Gauss i sine unge dage, forpurrerede sin
			lærers plan om at få ham til at tie stille. Hun bedte ham lægge tallene
			fra \(1\) til \(100\) sammen, i håbet at dette ville optage det unge
			geni i lidt tid. Gauss realiserede dog at der måtte være en bedre måde
			at løse problemet på, og udledte formularen \(\frac{n\cdot(n+1)}{2}\).
			Han svarede hurtigt \(5050\), og så måtte læreren finde på en anden
			kedelig opgave.

			Brug et \emph{for-loop} til at beregne \(1^2+2^2+\dots+100^2\).
		\end{exercise}

		\begin{exercise}
			Der er selvfølgelig ikke noget som forhindrer at du bruger et
			\emph{for-loop} ind i et andet \emph{for-loop}, og dette kan nogle gang
			være vældig praktisk.

			TODO: Find på opgave!
		\end{exercise}

	\subsection{Break-udtryk}

		\todo{ Sektion: \JavaInline{break} }

		\todo{
			Som bonus vil jeg gerne have noter om \JavaInline{do ... while} og
			\JavaInline{continue}, da disse er niché, men er samtidige nemme at
			forstå.

			Jeg vil ikke have noter om mere niché ting, som f.eks
			\emph{named-breaks}, da disse kræver introduktion til nye koncepter.
		}

		\todo{
			Som perspektiverende bonus kunne jeg tænke mig at fortælle lidt af
			historien om \emph{structured programming}, og hvorfor \JavaInline{goto}
			ikke er en ting i Java. \emph{Ikke pensum, bare for sjov.}
		}

\blindtext

\section{Funktioner}

	\todo{
		Denne sektion vil dække \emph{metode-definitioner}, \emph{retur-typer},
		\emph{argument-lister} og \JavaInline{return}.
		Jeg vil ikke dække statiske metoder, overloading, eller andre advanceret
		emner, da disse igen er ret niché, og kræver introduktion til nye
		koncepter.
	}

	\todo{
		\emph{pass-by-value} vs. \emph{pass-by-reference} bør nok dækkes, da det
		garenteret vil dukke op. Det er bare ikke specielt relevant før de kender
		til objekter. Antager ikke de vil rende ind i problemet før under projektet.
		Jeg er ikke vild med at introducere dem til sådanne niché fagtermer, men
		det vil være nemmere for en hjælper at kunne sige ``læs dette stykke om
		hvorfor dit program ikke fungere korrekt.''
	}

\blindtext

\section{Algoritmer eller Lister}

	\todo{
		Jeg kan ikke helt beslutte mig, hvad denne sidste sektion skal være om; enten
		kort om algoritmer; eller røre lidt ved lister.
		Understående giver pitches for begge.
	}

	\todo{
		Denne forlæsning og sektion dækker kort hvad Fanden en algoritme er, og
		giver simple eksempler, som viser hvordan man kan anvende dette kapitels
		tidligere emner.
		Et klart eksempel er fakultet-funktionen (eller muligvis fibonnaci,)
		da denne både kan udtrykkes med en for-lykke, rekursivt, og endda
		tail-rekursivt med en akkumulator! (Denne sidste er klart for
		advanceret, men min indre Olivier kan ikke lade vær.)
	}

	\todo{
		Lister er en fundamentel datastruktur, sandeligt er datamaten selv en
		kæmpe liste af data.
		Denne sektion vil dække \JavaInline{ArrayList} eller måske bare
		\JavaInline{int[]}? Jeg ved det ikke. Den første kræver introduktion til
		objekter, og den anden kræver introduktion til arrays og subscripts.
		Ingen af dem er elegante eller simple.
	}

	\todo{Note til Lukas: Kan vi bruge en mono-spaced font til kode, både
	inline og til blokke? Den nuværende er hæslig til begge dele. Er meget tilfreds med lstautogobble!}

	\begin{JavaCode}{Iterativ Fakultet}{lst:factorial-iterative}
		public int factorial (int n) {
			int a = 1;
			for (int i = 1; i <= n; i++)
				a = a * i;
			return a;
		}
	\end{JavaCode}

	\begin{JavaCode}{Rekursiv Fakultet}{lst:factorial-recursive}
		public int factorial (int n) {
			if (n <= 1)  return 1;
			return n * factorial(n-1);
		}
	\end{JavaCode}

	\begin{JavaCode}{Akkumulator Fakultet}{lst:factorial-accumulator}
		public int factorial (int n, int a) {
			if (n <= 1)  return a;
			return factorial(n-1, n * a);
		}
	\end{JavaCode}

	\begin{JavaCode}{Fakultet funktionen hvis Java var godt}{lst:factorial-good}
		fun factorial 0  =  1
		  | factorial n  =  n * factorial (n - 1)
	\end{JavaCode}

\blindtext

	
	\graphicspath{{sections/java/3/figures/}}
	\chapter{Objekter og klasser}

I kernen af objekt-orienteret programmering er de to nøglebegreber; objekter og klasser. Når man skriver i et objektorienteret sprog vil de objekter, man opretter, vises i problemdomenet og udgør de dele af modellen, som dit computerprogram forsøger at emulere. Objekter kategoriseres af klasser. En klasse beskriver alle objekter af en bestemt type. Dette kan virke lidt abstrakt, så lad os forsøge at forstå dette ved hjælp af et eksempel på at simulere en boghandel.

\begin{example}
	Hvis vi ønsker at skrive et computerprogram, der er en simpel simulation af en boghandel, skal vi arbejde med et par forskellige enheder, hvoraf en kunne være bøger. Er en bog en klasse eller et objekt? Hvad er genren? Hvem er forfatteren? Hvornår blev det skrevet? Hvad er titlen?
	
	Alle ovenstående spørgsmål giver kun mening i forbindelse med en enkelt bog. Dette skyldes, at "bog" i denne sammenhæng refererer til klassen Bog, mens en enkelt bog er et objekt af denne klasse. En enkelt genstand henvises til som et eksempel. Du kan oprette flere forekomster af en klasse, som i ovenstående eksempel vil give os mulighed for at lave mange forskellige bøger i mange forskellige genrer med forskellige titler, forfattere og publikationsdatoer.
\end{example}

\section{Klasser}
Når vi skal lave en ny klasse i java, gør vi det typisk i en ny fil. Vi skal kalde filen det vi gerne vil kalde klassen, så hvis vi gerne vil oprette en klasse \JavaInline|Book| så starter vi med at lave en ny fil Book.java

Et eksempel på deklaration af en klasse ses i \autoref{lst:book-class-header}

\begin{JavaCode}{Book klasse deklaration}{lst:book-class-header}
	public class Book {
		// This is the class body
	}
\end{JavaCode}

Klasse deklarationen består essentielt af 3 ting, først en access modifier (\JavaInline|public|) der definerer hvorfra man kan få adgang til klassen, derefter ordet \JavaInline|class| der definerer at det er en klasse vi laver, sidst har vi navnet på vores klasse (\JavaInline|Book|). Bemærk at klassenavnet er stavet med stort forbogstav. Mellem de to tuborgklammer i klassens krop, det er der vi definerer hvad klassens rent faktisk kan bidrage med.

\subsection{Access modifiers}

I har set flere eksempler på brug af access modifiers allerede, både når vi definerer nye klasse, men også når vi definerer metoder. Access modifiers definerer hvorfra i vores kode vi vil kunne få fat i vores klasser og metoder. Der er 4 forskellige acces modifiers:

\subsubsection{Private}

i klasse

\subsubsection{public} 

over alt

\subsubsection{protected} 

i pakke, gennem nedarvning udefor pakke, kun på metoder, constructor og variable

\subsubsection{default} 

i pakke

\subsection{Fields}
En klasse kan have en række variable der kan holde information om hvilket state klassen er i lige nu, men også til at holde al den information der er fælles for alle instanser af den givne klasse men hvis værdi vil variere fra instans til instans disse betegnes fields. Når vi kigger på en bog, så har den en titel, en forfatter, en genre, måske et årstal, alle bøger har dette, men det vil ikke være det samme for hver bog, vi kan derfor have det som fields i vores bog klasse, se \autoref{lst:book-class-variables}

\begin{JavaCode}{Book klasse fields}{lst:book-class-variables}
	public class Book {
		//fields
		private String title;
		private String author;
		private String genre;
		private int publishYear;
		
		//rest of class body
	}
\end{JavaCode}

Læg mærke til at variablerne er private, dette er af hensyn til at de så ikke kan ændres udefra, men kun er tilgængelige for de metoder der ligger inde i klassen.

\subsection{Constructor}

En constructor er en særlig metode der bliver kaldt når man opretter et objekt af den givne klasse. Det er her vi gerne instantierer alle variable vi skal bruge i den givne klasse, og generelt sætter klassen op så den er klar til brug. Constructoren varierer fra andre metoder i at metodenavnet skal være det samme som klassenavnet, ellers fungerer det ligesom andre metoder, der kan have nogle parametre, en constructor kan dog aldrig returnere noget. Et eksempel på en constructor kan ses i \autoref{lst:book-class-constructor}

\begin{JavaCode}{Book klasse constructor}{lst:book-class-variables}
	public class Book {
		//fields
		private String title;
		private int publishYear;
		
		//constructor
		public book(String title, int publishedIn) {
			this.title = title;
			publishYear = publishedIn;
		}
		
		//rest of class body
	}
\end{JavaCode}

Constructoren tager imod nogle parametre vi gemmer i vores fields, på den måde er vores bog initialiseret med en titel og udgivelsesår, hvilket vi kan benytte når vi senere hen skal til at lave andre metoder i klassen. Bemærk at der står \JavaInline|this.title| for at referere til title field, dette skyldes at vi fortæller constructoren at vi vil assigne værdien af parametret title til field'et title, hvis vi ikke skrev \JavaInline|this.| så ville den assigne værdien til sig selv, og vores field ville stadig være tomt. Det er derfor ikke nødvendigt at skrive \JavaInline|this.| foran \JavaInline|publishYear| da der ikke er noget sammenfald i navgivningen og der ikke kan være tvivl om hvad der menes, det er dog helt i orden at gøre alligevel.

\subsection{Metoder}

Metoder virker på samme måde som vi allerede har set, der er dog to standard metoder man ofte ser i klasser, nemlig \JavaInline|get| og \JavaInline|set| metoder, disse gør det fx muligt at få adgang til at se eller ændre fields. 

Hvis vi gerne vil kunne se hvad der ligger i de forskellige fields uden for vores klasse skal vi tilføje metoder der er public som kan hive det ud, disse kaldes \JavaInline|get| metoder, da det henter information. I \autoref{lst:book-class-get}

\begin{JavaCode}{Book klasse get metode}{lst:book-class-get}
	public String getTitle() {
		return title;
	}
\end{JavaCode}

\JavaInline|get| metoder returnerer altid noget, og i sin rene form navngives den \JavaInline|get<<fieldname>>| og returner \JavaInline|<<field>>|.

Hvis det skal være muligt at ændre værdien i sine fields kan dette gøres med \JavaInline|Set| metoder, et eksempel på at sætte prisen på en bog kan ses i \autoref{lst:book-class-set}

\begin{JavaCode}{Book klasse set metode}{lst:book-class-set}
	public void setPrice(double price) {
		this.price = price;
	}
\end{JavaCode}

\JavaInline|set| metoder vil i sin rene form aldrig returnerer noget, men have et parameter der erstatter værdien af \JavaInline|<<field>>| og hedde \JavaInline|set<<field>>|

Vi ender dermed med den fulde klassedefinition der kan ses i \autoref{lst:book-class}

\begin{JavaCode}{Book klasse}{lst:book-class}
	public class Book {
		//fields
		private String title;
		private int publishYear;
		private double price;
		
		//constructor
		public book(String title, int publishedIn) {
			this.title = title;
			publishYear = publishedIn;
			price = 0.0;
		}
	
		//get methods
		public String getTitle() {
			return title;
		}
		public String getPublishYear() {
			return publishYear;
		}
		public String getPrice() {
			return price;
		}
		
		//set methods
		public void setPrice(double price) {
			this.price = price;
		}
	}
\end{JavaCode}

I dette eksempel er det muligt at ændre prisen, og der er ikke nødvendigt at kende prisen for at lave en bog, vi initialiserer dog prisen til 0 når vi laver bogen, så er vi sikker på at vi har initialiseret alle vores fields. Der er get metoder til alle fields så det er muligt at spørge klassen om hvad titlen, udgivelsesåret og prisen er på bogen. Der er kun opretter set metode til prisen, da det ikke giver mening at kunne ændre titel og udgivelsesår efter en bog er udgivet, hvis disse skulle ændres ser vi det som en ny bog.

\subsection{Nedarvning}

Alle klasser i Java nedarver fra objekt klassen

\section{Objekter}

hvad har vi allerede set af eksempler på

\subsection{Oprettelse af objekter}

Eksempel: lav bøger

\subsection{Kalde metoder på objekter}

Eksempel: get og set stuff på bog

\subsection{Casting}

\section{Klasser i Java der er gode at kende}

\subsection{Lists}

Arraylists

\subsection{Hashmaps}
Et HashMap er næsten ligesom en Arrayliste. Det er en datastruktur. Altså en måde at organizere en samling variabler på. Forskellen mellem en HashMap og en ArrayListe er hvordan man tilgår de gemte variabler. En ArrayListe giver simpelt hver variabel et tal, og så kan man bruge minArrayListe.get(tal); Ved HashMaps bestemmer I hvad hver variabel har af "nøgle" til at blive tilgået. i \autoref{hashmap1} er et eksempel med en simpel telefonliste.
\begin{JavaCode}{telefonListe}{hashmap1}
	HashMap<String, Integer> telefonliste = new HashMap<>();
	
	public void addContact(String name, int number){
		telefonliste.put(name, number);
	}
	
	public int getContact(String name){
		telefonliste.get(name);	
	}
\end{JavaCode}
Lad os gå igennem eksemplet. HashMaps skal have to typer hvor ArrayLister kun skal have en. \marginnote{Ved en faktisk implementation ville man nok bruge en anden type til et telefonnummer for at sikre at der er 8 cifre}Dette er fordi at Hashmaps både skal kende typen af nøglen, og typen af den variabel der gemmes. Her er nøglen sat til en string, da vi forbinder de gemte telefonnumre til et navn. Og den gemte variabel er en integer for der er et nummer.  For at tilføje elementer til listen bruger vi .put(nøgle, variabel); og for at få en variabel bruger vi .get(nøgle); Nøglen kan være af alle de normale typer, så man kan faktisk lave en ArrayListe med et HashMap som bruger integers som nøgler. Yderligere funktionalitet for HashMaps kan ses i \autoref{hashmap2}
\begin{JavaCode}{Metoder for Hashmap}{hashmap2}
	ArrayList<Integer> alleNumre = telefonliste.values();
	//Faa alle variabler i hashmappet som en arrayliste
	
	telefonliste.remove("Klods Hans");
	//Fjern en variabel fra hashmappet
	
	boolean harLeasy = telefonliste.containsValue(88888888);
	//True hvis variablen ligger i hashmappet
	
	boolean harSnehvide = telefonliste.containsKey("Snehvide");
	//True hvis noeglen ligger i hashmappet
	
	telefonliste.size();
	//Antal variabler i listen
\end{JavaCode}
En sidste bemærkning er at der kun kan være en variabel for hver nøgle. Hvis du putter en varibel med en nøgle som allerede er i mappet fjernes den gamle variabel som havde den nøgle. Altså den overskrives.

\subsection{Enumerables}
Enumerable er en speciel type klasse hvor objekter kan være fra et specifikt defineret sæt af muligheder. Lad os demonstere med \autoref{enums1}.
%TODO fix javacode her. Ikke alt tekst kommer med
\begin{JavaCode}{Ugedage}{enums1}
	public enum Weekday{
		MONDAY, TUESDAY, WEDNESDAY, THURSDAY, FRIDAY, SATURDAY, SUNDAY
	}
	
	if(today == Weekday.FRIDAY){
		party();
	}
\end{JavaCode}
Altså er det når man har brug for en variabeltype som repræsenterer en mulighed ud af en gruppe af elementer. Man kan oprette en enum klasse og så oprette variabler af den type og sætte dem til den type og sammenligne om de er af den type. Husk at bruge EnumNavn.elementNavn istedet for bare elementNavn når du opretter variablens værdi.


	
	\part{Android apps}
	
	\graphicspath{{sections/android/layouts/figures/}}
	% !TeX spellcheck = da_DK
\chapter{Android udvikling}

Når vi snakker om Android, så snakker vi om et "styresystem". Det kan man 
anskue som et kæmpe stort program, der administrerer alle de apps der kører på 
mobil-telefonen. I denne bog, bruger vi begreberne "styresystem" og "platform" 
i flæng.

De eneste relevante styresystemer til mobiler er, pt., de følgende to:

\begin{itemize}
	\item IOS (IPhone, IPads etc.)
	\item Android (Samsung, HTC, Sony etc.)
\end{itemize}

På SDC, kigger vi på den sidste af de to, nemlig Android platformen. Selvom 
app-udvikling til IOS og Android minder meget om hinanden, så er der en række 
væsentlige forskelle i de værktøjer man bruger. Derfor kigger vi kun på den ene 
af de to styresystemer.

\section{Hvad er Android apps?}

\marginfigure{xkcd_app.png}{Billede lånt af: \url{https://xkcd.com}}

Når man skal beskrive hvad en ``Android app'' er, så kan det gøres lidt 
simplere ved at forstå hvad en ``app'' er. App er en forkortelse for 
applikation, hvilket i denne kontekst er synonymt med ``et program''. Når vi 
snakker om en app, snakker vi derfor om et computer program, som er kodet med 
en række instruktioner. Derfor er de programmer du har på din mobil, som 
WordFeud, SnapChat og Facebook, alle eksempler på apps. Computerspil, som 
Farming Simulator 2017 og Battlefield 4 er faktisk også eksempler på apps, så 
det er faktisk et meget vidt begreb. 

I Android verdenen, er stort set alt apps. Lige fra låseskærmen og 
``indstillinger''-menuen til Super Mario Go, og Farmville. Derfor er der meget 
få grænser for hvad man kan opnå ved app-udvikling til Android.

\subsection{Anatomien af en Android app}
Android apps består af mange forskellige dele. De primære dele kan dog deles op i følgende tre kategorier:

\subsubsection{Ressourcer}
Ressourcer er indstillinger, billeder, layouts (se \autoref{sec:android:layouts}) og alt mulig andet data der bruges i en app. De findes i mappen \texttt{res/} og har forskellige undermapper, afhængig af hvilken type ressource det er.

\subsubsection{Activities}
Activities er aktiviteter man kan foretage sig i app'en. Der kan f.eks. være en "tag billede" activity, en "gem billedet" activity og en "se dit galleri af billeder" activity. Disse activities vil blive udforsket yderligere i \autoref{cha:activities-intents}.

\subsubsection{Manifest}
Et app-manifest er en beskrivelse af app'en som Android styresystemet gør brug af når den kører app'en på mobilen. I manifestet beskriver man hvilke rettigheder app'en har brug for, samt hvilke teknologier den gør brug af.

Derudover kan manifestet indeholde information om hvordan andre apps kan kommunikere med denne app, og andre indstillinger der binder app'en sammen med resten af platformen.

\section{Layouts}
\label{sec:android:layouts}

Layouts er en ressource som app'en gør brug af når den skal vise den 
\gls{interface}\footnote{En \gls{interface} er, i denne kontekst, det grafiske 
design som brugeren interagerer med. Det kan f.eks. være en menu med knapper 
eller et billede.} som brugeren skal se på inde i app'en. Det er en beskrivelse 
af strukturen i app'ens \gls{interface}.

For bedre at forstå layouts, skal vi først forstå den måde vi skriver et layout på. Ligesom med programmeringssprog, hvor vi har forskellige sprog til at beskrive kode på (som f.eks. Java), findes der forskellige måder at beskrive strukturer og layouts. Vi kalder disse sprog for \textit{strukturelle sprog}, en af de mere udbredte af sådanne sprog er det man kalder for \textit{XML}\footnote{e\textbf{X}tensible \textbf{M}arkup \textbf{L}anguage}.

\subsection{XML}
I sin kerne er XML et meget simpelt sprog. Det er bygget op af ``tags'', hvoraf 
der findes to slags tags: et ``opening-tag'' og et ``closing-tag''. 
Opening-tags er skrevet ved at have et ``mindre-end'' tegn ($<$), efterfulgt af 
noget tekst og så afsluttet med et ``større-end'' tegn ($>$). For eksempel, er 
følgende et opening-tag: \XmlInline|<Button>|.

Closing-tags er meget lig opening-tags, der er blot en skråstreg efter 
``mindre-end'' tegnet, for at indikere at dette tag lukker et opening-tag der 
er skrevet tidligere i dokumentet. Et eksempel på et closing-tag, der lukker 
det opening-tag der var i det tidligere eksempel, kan ses her: 
\XmlInline|</Button>|.

Det er vigtigt at man altid lukker sine tags. Man må aldrig have et opening-tag uden at det bliver lukket, og man må aldrig have et closing-tag der ikke har noget opening-tag at lukke.

Man kan tilknytte ``attributter'' til sine opening-tags. Det gør man ved at 
skrive navnet på attributten, efterfulgt af et lig-med tegn, efterfulgt af 
attributtens værdi. Denne værdi er typisk skrevet inden i to gåse-øjne ("). Et 
eksempel på et opening-tag med en attribut er følgende: 
\XmlInline|<Button Text="Hej med dig!">|.

Et eksempel på et XML dokument ses herunder. Det beskriver et vindue, med en 
``label'' dvs. et kort stykke tekst, med to knapper. Vi har skrevet de tags, 
der hører til vores label og knapperne, imellem vinduets opening- og 
closing-tags. Dette betyder at de kommer til at høre til vinduet, altså at 
vinduet har en label og to knapper. Denne relation bliver ofte beskrevet som at 
vinduet er forældreren til de tre børn (label og de to knapper).

\begin{example}
	\begin{XmlCode}{Et lille XML dokument}{first-valid-xml-document}
		<Window>
			<Label Text="Vil du lukke vinduet?"></Label>
			<Button Text="Ja"></Button>
			<Button Text="Nej"></Button>
		</Window>
	\end{XmlCode}
\end{example}


For at simplificere XML dokumentet, kan vi udnytte os af et så-kaldt 
self-closing tag. Altså et opening-tag der lukkes med det samme, uden man 
behøver at bruge et closing-tag. Dette skrives ved at putte en skrå-streg ind 
før ``større-end'' tegnet:

\begin{example}
	\begin{XmlCode}{Et lille XML dokument med self-closing tags}{self-closing-tags-dokument}
		<Window>
			<Label Text="Vil du lukke vinduet?"/>
			<Button Text="Ja"/>
			<Button Text="Nej"/>
		</Window>
	\end{XmlCode}
\end{example}

\begin{remark}
	Bemærk at vi ikke kunne bruge et self-closing tag til vores \XmlInline|<Window>| tag, fordi at det ikke skulle lukkes med det samme, da det skulle indeholde dets tre børn.
\end{remark}

Vi har nu taget en kort intro til XML, det er faktisk et meget mere komplekst sprog end det vi har gennemgået her, men det burde ikke blive nødvendigt at forstå det hele for det vi skal arbejde med her.

Hvis man gerne vil lære mere om XML, kan man besøge W3Schools: \url{https://www.w3schools.com/xml/}.


\subsection{Hvad er layouts?}
Layouts er, som nævnt, en måde at beskrive app'ens udseende på igennem det strukturelle sprog XML.

Vi lægger layouts i ressource mappen \texttt{res/layout/}, og et layout er 
faktisk bare et XML-tag der kan indeholde andre tags. De beskriver hvordan 
flere visuelle elementer skal pladseres i forhold til hinanden.

I Android Studio er der mulighed for at arbejde med disse layouts igennem en 
visuel editor, så man slipper for at rode med XML-kode. Det er dog stadig en 
god idé at forstå hvordan de fungerer i koden, både for forståelsens skyld og 
for at kunne rette fejl hvis det visuelle redigeringsværktøj driller.

\subsubsection{Frame layouts}

Det første eksempel på et layout er det såkaldte \textit{frame layout}, det er et meget simeplt og effektivt layout der er designet til at pladsere et enkelt element et specifikt sted på skærmen.

Det er svært at håndtere dette layout hvis det har mere end et barn. Derfor bør 
det typisk kun indeholde et enkelt barn. Deraf kommer navnet ``frame'', det 
bruges som en ``ramme'' til et andet element.

I \autoref{fig:android:layouts:frame-layout} er der et eksempel på hvordan man 
kan have to ``børn'' indeni et frame layout, for at placere det ene oven på det 
andet.

\begin{figure}[h]
	\begin{center}
		\includegraphics[width=5cm]{frame_layout.png}
		\caption{Et frame layout i en app}
		\label{fig:android:layouts:frame-layout}
	\end{center}
\end{figure}

\clearpage

\begin{XmlCode}{Det layout der giver udseendet i \autoref{fig:android:layouts:frame-layout}}{frame-layout}
	<?xml version="1.0" encoding="utf-8"?>
	<FrameLayout 
		xmlns:android=
			"http://schemas.android.com/apk/res/android"
		xmlns:tools="http://schemas.android.com/tools"
		android:layout_width="match_parent"
		android:layout_height="match_parent"
		android:paddingBottom=
			"@dimen/activity_vertical_margin"
		android:paddingLeft=
			"@dimen/activity_horizontal_margin"
		android:paddingRight=
			"@dimen/activity_horizontal_margin"
		android:paddingTop=
			"@dimen/activity_vertical_margin"
		tools:context=
			"com.example.lukas.myapplication.MainActivity">
	
		<TextView
			android:layout_width="wrap_content"
			android:layout_height="wrap_content"
			android:text="Hello World!"
			android:textSize="100sp"/>
		<TextView
			android:layout_width="fill_parent"
			android:layout_height="wrap_content"
			android:layout_gravity="center_vertical"
			android:layout_marginBottom="50dp"
			android:background="#ddaa55"
			android:text="Block out the sun!"
			android:textSize="50sp"/>
	</FrameLayout>
\end{XmlCode}

\subsubsection{Linear layouts}
Lineære layouts er det simpleste layout man kan bruge. Det tager alle sine 
børn, og arrangere dem således at de ligger enten under hinanden (vertikalt) 
eller ved siden af hinanden (horisontalt).

Der er ikke meget der kan ændres på hvordan dette layout virker, men det er 
rigtig simpelt til at vise f.eks. lister eller at pladsere elementer ved siden 
af hinanden.

I \autoref{fig:android:layouts:linear-layout} er der et eksempel på et lineært 
layout med tre børn, det første er bare sat ind og ligger øverst. Det næste er 
sat til at være lige så bredt som skærmen, og bliver automatisk sat nedenunder 
det første. Til sidst fylder det nederste og sidste element resten af skærmen.

\begin{figure}[h]
	\begin{center}
		\includegraphics[width=5cm]{linear_layout.png}
		\caption{Et lineært layout i en app}
		\label{fig:android:layouts:linear-layout}
	\end{center}
\end{figure}

\clearpage

\begin{XmlCode}{Det layout der giver udseendet i %
\autoref{fig:android:layouts:linear-layout}}{linear-layout}
	<?xml version="1.0" encoding="utf-8"?>
	<LinearLayout 
		xmlns:android=
			"http://schemas.android.com/apk/res/android"
		xmlns:tools="http://schemas.android.com/tools"
		android:layout_width="match_parent"
		android:layout_height="match_parent"
		android:orientation="vertical"
		android:paddingBottom=
			"@dimen/activity_vertical_margin"
		android:paddingLeft=
			"@dimen/activity_horizontal_margin"
		android:paddingRight=
			"@dimen/activity_horizontal_margin"
		android:paddingTop=
			"@dimen/activity_vertical_margin"
		tools:context=
			"com.example.lukas.myapplication.MainActivity">
		
		<TextView
			android:layout_width="wrap_content"
			android:layout_height="wrap_content"
			android:text="Hello World!"
			android:background="#FF0000"
			android:textSize="50sp"/>
		<TextView
			android:layout_width="fill_parent"
			android:layout_height="wrap_content"
			android:background="#ddaa55"
			android:text="Fill the parent to the sides"
			android:textSize="20sp"/>
		<TextView
			android:layout_width="fill_parent"
			android:layout_height="fill_parent"
			android:layout_gravity="center_vertical"
			android:background="#44aaaa"
			android:text="Fill the parent to the bottom!"
			android:textSize="20sp"/>
	</LinearLayout>
\end{XmlCode}

\clearpage

\subsubsection{Table layouts}
Table layouts arrangerer sine børn i rækker og kolonner. De er simple og 
bruge, og er rigtig gode hvis man skal have noget der ligner en tabel men de er 
meget rigide. Alt man laver med tabular-layout vil se ud som en tabel.
 
 
I \autoref{fig:android:layouts:table-layout} er der et eksempel på et table 
layout. Der er 3 rækker og 3 kolonner i layoutet. Man laver nye rækker ved at 
putte det der skal være i rækken inden i et ``TableRow'' tag.
 
\begin{figure}[h]
	\begin{center}
		\includegraphics[width=5cm]{table_layout.png}
		\caption{Et table layout i en app}
		\label{fig:android:layouts:table-layout}
	\end{center}
\end{figure}
 
\clearpage
 
\begin{XmlCode}{Det layout der giver udseendet i %
 		\autoref{fig:android:layouts:table-layout}}{linear-layout}
	<?xml version="1.0" encoding="utf-8"?>
	<TableLayout 
		xmlns:android=
			"http://schemas.android.com/apk/res/android"
		xmlns:tools=
			"http://schemas.android.com/tools"
		android:layout_width="match_parent"
		android:layout_height="match_parent"
		android:paddingBottom=
			"@dimen/activity_vertical_margin"
		android:paddingLeft=
			"@dimen/activity_horizontal_margin"
		android:paddingRight=
			"@dimen/activity_horizontal_margin"
		android:paddingTop=
			"@dimen/activity_vertical_margin"
		tools:context=
			"com.example.lukas.myapplication.MainActivity">
		
		<TableRow>
			<TextView
			android:text="Row 1 column 1"
			android:background="#22ddee"
			android:padding="10dp"
			android:layout_margin="5dp"/>
		</TableRow>
		
		<TableRow>
			<TextView
			android:text="Row2 column 1-3"
			android:background="#00AAFF"
			android:padding="10dp"
			android:layout_span="3"
			android:layout_margin="5dp"/>
		</TableRow>
		
		<TableRow>
			<TextView
			android:text="Row3 column 1"
			android:background="#FF0000"
			android:padding="10dp"
			android:layout_margin="5dp"/>
			<TextView
			android:text="Row3 column 2"
			android:background="#00FF00"
			android:padding="10dp"
			android:layout_margin="5dp"/>
			<TextView
			android:text="Row3 column 3"
			android:background="#FFFF00"
			android:padding="10dp"
			android:layout_margin="5dp"/>
		</TableRow>
	</TableLayout>
\end{XmlCode}

\begin{exercise}
	Lav et tabular-lignende layout ved hjælp af linear layouts.
\end{exercise}

\begin{exercise}
	Lav et linear-lignende layout ved hjælp af tabular layouts (både horisontalt og vertikalt).
\end{exercise}

\clearpage

\subsubsection{Relative layouts}
Relative layouts er et meget fleksibelt og stærkt værktøj. De kan arrangere 
deres børn relativt til hinanden. F.eks. får man mulighed for at sige ``den 
røde kasse skal være til højre for den blå kasse''. Der er næsten ikke det man 
ikke kan med relative layouts.

\begin{figure}[h]
	\begin{center}
		\includegraphics[width=5cm]{relative_layout.png}
		\caption{Et relative layout i en app}
		\label{fig:android:layouts:relative-layout}
	\end{center}
\end{figure}

\clearpage

\begin{XmlCode}{Det layout der giver udseendet i %
		\autoref{fig:android:layouts:relative-layout}}{relative-layout}
	<?xml version="1.0" encoding="utf-8"?>
	<RelativeLayout 
		xmlns:android=
			"http://schemas.android.com/apk/res/android"
		xmlns:tools=
			"http://schemas.android.com/tools"
		android:layout_width="match_parent"
		android:layout_height="match_parent"
		android:paddingBottom=
			"@dimen/activity_vertical_margin"
		android:paddingLeft=
			"@dimen/activity_horizontal_margin"
		android:paddingRight=
			"@dimen/activity_horizontal_margin"
		android:paddingTop=
			"@dimen/activity_vertical_margin"
		tools:context=
			"com.example.lukas.myapplication.MainActivity">
		
		<TextView
			android:id="@+id/text1"
			android:layout_width="wrap_content"
			android:layout_height="wrap_content"
			android:text="Hello World!"
			android:textSize="20dp"
			android:padding="10dp"
			android:background="#FF0000"
			android:layout_toRightOf="@id/text1"/>
		<TextView
			android:layout_width="wrap_content"
			android:layout_height="wrap_content"
			android:text="Centered in parent, and below red"
			android:textSize="20dp"
			android:padding="10dp"
			android:background="#00FF00"
			android:layout_below="@id/text1"
			android:layout_centerInParent="true"/>
		<TextView
			android:layout_width="wrap_content"
			android:layout_height="wrap_content"
			android:text="Centered in parent!"
			android:textSize="20dp"
			android:padding="10dp"
			android:background="#FF00FF"
			android:layout_centerInParent="true"/>
		<TextView
			android:layout_width="wrap_content"
			android:layout_height="wrap_content"
			android:text="To right of red"
			android:textSize="20dp"
			android:padding="10dp"
			android:background="#00FFFF"
			android:layout_toRightOf="@id/text1"/>
	</RelativeLayout>
\end{XmlCode}

\cleardoublepage

\section{Listeners}
I Android app-udvikling finder der noget der hedder en ``listener'', som navnet 
antyder så er der tale om noget der lytter. I app'ens \gls{interface}, er der 
tale om elementer der lytter efter brugerens input. F.eks. hvis man gerne vil 
skrive noget kode der reagerer på at brugeren trykker på en knap, er der tale 
om en ``On-Click-Listener''.

Disse listeners er altså måden hvorpå vi kan knytte vores \gls{interface} 
sammen med den kode vi gerne vil køre. Indtil videre har vi kun snakket om 
layouts, hvordan de virker og ser ud. I praksis er et layout næsten altid 
knyttet sammen med en Activity, som vi snakker mere om i 
\autoref{cha:activities-intents}. Lad os indtil videre antage at der kun er et 
layout (activity\_main.xml) og en activity (MainActivity.java). Hvis vi ikke 
har ændret i vores Activity, men blot bruger den Activity som Android Studio 
har lavet til os, vil koden se således ud:

\begin{JavaCode}{En tom MainActivity}{main-activity}
	public class MainActivity extends AppCompatActivity {
		
		@Override
		protected void onCreate(Bundle savedInstanceState) {
			super.onCreate(savedInstanceState);
			setContentView(R.layout.activity_main);
		}
	
	}
\end{JavaCode}

Læg mærke til dette stykke kode: 
\JavaInline|setContentView(R.layout.activity_main)|. I den linje knytter vi 
vores layout (activity\_main.xml) sammen med vores MainActivity, ved at sætte 
MainActivity's indhold til at være activity\_main layoutet.

Vi kan nu ændre vores activity\_main layout, ved at tilføje en knap i midten af 
skærmen, som vist i \autoref{fig:android:layouts:button-layout}.

\begin{figure}[h]
	\begin{center}
		\includegraphics[width=5cm]{button_layout.png}
		\caption{Et relative layout med en knap i midten}
		\label{fig:android:layouts:button-layout}
	\end{center}
\end{figure}

Hvis vi nu gerne vil have at app'en reagerer på at vi trykker på denne knap, så 
kan vi tilføje ``android:onClick'' attributten til knappen og give den attribut 
en værdi der svarer til den funktion vi gerne vil have til at køre når vi 
trykker på knappen. Et eksempel på dette kan ses i \autoref{lst:button-layout}.
Herefter kan vi blot tilføje den funktion vi tilføjede i ``onClick'' 
attributten til vores activity, som vist i \autoref{lst:button-activity}.

Vi kan se at ``buttonClicked'' funktionen får et ``View'' med som argument. 
Dette View er det element i app'ens \gls{interface} der har udløst funktionen. 
I dette tilfælde er det knappen man trykkede på. Hvis vi lægger $10$ til dens 
$x$ værdi, vil den flytte sig $10$ ``pixels'' til højre.

\begin{exercise}
	Implementer layoutet beskrevet i \autoref{lst:button-layout} og aktiviteten 
	i \autoref{lst:button-activity}.
\end{exercise}

\begin{exercise}
	Hvorfor skal vi lægge noget til $x$ for at flytte knappen mod højre? Hvad 
	betyder $x$?
\end{exercise}

\begin{exercise}
	Hvordan flytter man i stedet knappen op, ned eller til venstre?
\end{exercise}

\begin{exercise}
	Lav et layout med to knapper, den ene skal flytte sig op når man trykker, 
	den anden skal flytte sig mod højre.
\end{exercise}

\clearpage


\begin{XmlCode}{Et layout med en knap der kalder ``buttonClicked'' funktionen% 
når den bliver trykket på.}{lst:button-layout}
	<?xml version="1.0" encoding="utf-8"?>
	<RelativeLayout 
		xmlns:android=
			"http://schemas.android.com/apk/res/android"
		xmlns:tools="http://schemas.android.com/tools"
		android:layout_width="match_parent"
		android:layout_height="match_parent"
		tools:context=
			"com.example.lukas.myapplication.MainActivity">
	
	<Button
		android:id="@+id/catchMeButton"
		android:layout_width="wrap_content"
		android:layout_height="wrap_content"
		android:layout_centerInParent="true"
		android:onClick="buttonClicked"
		android:text="Can't catch me!" />
		
	</RelativeLayout>
\end{XmlCode}

\clearpage

\begin{JavaCode}{En activity der flytter en knap nedad når den bliver trykket %
på.}{lst:button-activity}
	public class MainActivity extends AppCompatActivity {
		
		@Override
		protected void onCreate(Bundle savedInstanceState) {
			super.onCreate(savedInstanceState);
			setContentView(R.layout.activity_main);
		}
		
		
		public void buttonClicked(View view) {
			view.setX(view.getX() + 10);
		}
	}
\end{JavaCode}



	
	\graphicspath{{sections/android/activities/figures/}}
	% !TeX spellcheck = da_DK
\chapter{Activities og Intents}

Indtil videre har vi kigget kort på vores MainActivity som er entry point for en app. Hvis vi gerne vil have en ny activity, så skal vi starte denne med en intent. Dette kapitel vil kigge nærmere på en activity og dens livscyclus, forklare oprettelsen af nye activities, samt hvordan man kan starte disse activities med intents.

\section{Activities}

Dybere forklaring af MainActivity.

\subsection{Activity life cycle}

\begin{figure}[H]
	\begin{center}
		\includegraphics[width=7cm]{developerandroidcom_activitylifecycle.png}
		\caption{Activity lifecycle lånt af developer.android.com}
		\label{fig:android:activities:activitylifecycle}
	\end{center}
\end{figure}

Ovenstående figur viser de forskellige stadier en activity er i, og kan hjælpe til at give et overblik over hvornår de forskellige metoder bliver kaldt, og dermed også hvad man gerne vil have skal ske ide forskellige steps undervejs.

\subsection{Oprettelse af nye activities}

For at oprette en ny activity starter vi med at lave en ny java fil ved siden af MainActivity.java (i ??? folderen) En ny activity skal altid nedarve fra Activity, ligesom MainActivity. 

\begin{example}\noindent
	\begin{JavaCode}{Eksempel på en activity}{pop-up-activity}
		package com.example.housa.myapplication
		
		import android.app.Activity;
		import android.os.bundle;
		
		public class PopUpActivity extends Activity {
		  
		  @Override
		  protected void onCreate(Bundle savedInstanceState) {
		    super.onCreate(savedInstanceState);
		    setContentView(R.layout.activity_popup);
		  }
		}
	\end{JavaCode}
\end{example}

Ud over a skrive selve Java koden der definerer opførslen for en activity skal den til føjes til manifestet. Dette gøres ved at tilføje nedenstående kode til manifestet inde i application tag'et.

\begin{example}\noindent
	\begin{XmlCode}{XML der tilføjer PopUpActivity til manifestet}{add-activity-to-manifest}
		<activity android:name=".PopUpActivity" />
	\end{XmlCode}
\end{example}

For at kunne starte ovenstående activity skal vi benytte os af Intents som vil blive forklaret i næste sektion.


\section{Intents}

En intent er en besked man kan sende for at bede en anden komponent om at gøre noget. Intents kan bruges på flere forskellige måder, men de tre primære brugsscenarier er:

\subsubsection{Starte en anden activity}

Man kan starte en ny instans af en activity ved at sende en intent med som parameter til startActivity(). 

\subsubsection{Starte en service}

En service er et komponent der kører i baggrunden og ikke har noget interface. Man starter en service på lidt samme måde som en activity, nemlig ved at sende en intent med som parameter til metoden startService().

\subsubsection{Sende en broadcast}

En broadcast er en besked som enhver app kan modtage. Der er forskellelige broadcasts for system events som fx foretag opkald eller opladning igang. Man kan sende en broadcast ved at sende en intent med som parameter til sendBroadcast() eller sendOrderesBroadcast().
En standard broadcast kan ikke blive stoppet, og kan ikke videregive resultater. En orderedBroadcast vil sende broadcasten til alle relevante BroadCastRecievers en af gangen og tillade resultatet at propagere.

\subsection{Explicit vs Implicit}

Der er to typer intents explicitte og implicitte. I explicitte intents specificerer man præcist hvilket komponent man vil have til at reagere. Da man kender navngivningen i ens egen app, vil dette være den typeske måde at håndtere kommunikationen indenfor ens egen app, som at starte nye activities inde i appen.

De implicitte intents fortæller ikke præcist hvilket komponent man vil have til at reagere, men i stedet hvilken handling man gerne ville have udført, så kan alle komponenter der har den givne funktionalitet tilbyde at udføre handlingen. Dette benyttes fx hvis man gerne vil have ens app til at sende en mail, men gerne bare vil bruge hvad end mailklient brugeren allerede har på telefonen.

\subsection{Start en activity med en Intent}

hvordan det virker + hvor man gør det henne

eksempel kode

sende beskeder med intents

\subsection{Start en activity fra en anden app med en Intent}

implicit intents

eksempel kode på send mail der benytter indbygget mailklient

\subsection{Start en activity med en Intent fra en anden app}

modtag implicitte intents fra andre apps

broadcastrecievers


\begin{exercise}
	Lav en app der har en main activity, og en anden activity der viser et billede. Tilføj en knap til din main activity, der åbner din anden activity.
\end{exercise}

\begin{exercise}
	Lav en app der har en main activity, og en anden activity med et EditText felt. Tilføj en knap og et EditText felt til din main activity, der åbner din anden activity og viser teksten fra EditText i din main activity.
\end{exercise}

\begin{exercise}
	Lav en Activity der har en ‘send Emil’ knap, der åbner en Email dialog
\end{exercise}

\begin{exercise}
	Lav en activity der agerer som et alternativ til email klienten når man trykker på ‘send mail’ knappen fra opgave 3
\end{exercise}

\begin{exercise}
	Lav en BroadCastReciever, der blokerer alle udgående opkald.
\end{exercise}

\begin{exercise}
	Lav en activity der sender en Broadcast, og en BroadCastReciever der modtager den givne BroadCast.
\end{exercise}


	
	\graphicspath{{sections/android/animations/figures/}}
	% !TeX spellcheck = da_DK

\chapter{Animation}
Animation handler om at få ting til at bevæge sig og ændre udseende. Mere specifikt lærer I at få Views til at flytte sig på skærmen, rotere, ændre størrelse og på andre måder ændre udseende.
\section{Grundlæggende teori om animation}
Før vi går igang med at lave animationer i Android skal vi lige have på plads hvordan animationer virker.
Animationer og video generelt er en række af \gls{frame}s som skifter så hurtigt at det ligner en flydende bevægelse. \\
%TODO Illustration af still frames i række
Når man laver en animation foregår det altså ved at man laver små ændringer i layoutet så hurtigt efter hinanden at det bliver "flydende". En anden måde at sige det på er at billedet ændrer sig i små hak, og man gør de hak så små at det ikke kan ses at der arbejdes i hak. Det vi skal lave er altså noget der virker ligsom stopmotion. \\
Grunden til at man arbejder i hak er at det tager en hel del arbejde for en computer \marginnote{En mobil er også en computer} at udregne hvordan den skal vise et frame, så det gælder om at finde en balance mellem udseende og hvor meget af computerens regnekraft animationen tager. Det animation vi arbejder med har nu et fastsat tempo for hvor hurtigt billedet ændrer sig, så I skal ikke overveje denne balance.
\subsection{Animationens rammer}
En animation starter et sted og slutter på et andet sted. Det kan både være at ens view starter i bunden af skærmen og flytter sig til toppen, at det starter med at være lille og slutter med at være stor eller at det starter på hovedet og roterer til den står lige.
Animationen tager også et stykke tid. Det kunne være et halvt sekund eller ti sekunder det tager for den at nå fra start til slut. Her gælder det om at lave en balance mellem at animationen tager tid nok så brugeren kan nå at se hvad der sker, og at brugeren ikke sidder og venter på at animationen bliver færdig. Hvis der ikke er stor forskel på start og slut i din animation kan det være at det slet ikke er nødvendigt at sætte en animation op, men bare at sætte dit View til slutpunktet med det samme. \marginnote{Hvis du bare vil have en oversigt over mulighederne for at ændre ens view så se sektionen "Hvad kan vi ændre"} \\
Ud fra disse rammer kan vi så regne hvor langt vi skal være nået på et givet tidspunkt i animationen. 
\begin{example}
	Vi skal rykke et View fra 30 til 50 pixels horisontalt på 2 sekunder. Vi skal vide hvilken pixel vi er kommet til efter 0,5 sekunder, 1 sekund, 1,23 sekunder etc. Det vi kan gøre er at plotte en funktion, hvilket er meget simpelt hvis den er lineær:
	\begin{equation}
	pixel=10\cdot tid+30
	\end{equation}
	Hvor 10 er antallet af pixels Viewet bevæger sig i sekundet og 30 er startværdien
\end{example}
Vi skal nu ikke til at udregene en hel masse funktioner for animationen. Det gør Android for os.
\section{En helt simpel animation i Android}
For at lave en animation bruger vi den klasse der hedder ValueAnimator. Du kan tænke på et objekt af ValueAnimator som en animation. Du opretter en ValueAnimator som vist i \autoref{animation1}
\begin{JavaCode}{Oprettelse af ValueAnimator}{animation1}
	ValueAnimator minAnimation = ValueAnimator.ofInt(startInt, slutInt);
\end{JavaCode}
\marginnote{Noter at du ikke skal skrive new. Den tekniske grund hertil er at ofInt() er en metode som returnerer en ny ValueAnimator.}
ofInt() tager to integers som er din startværdi og din slutværdi. Det kunne f.eks. være 0 og 360 hvis man ville lave en animere et View til at rotere 360 grader. \\
Det næste din animation har brug for er hvor lang tid den skal køre. Den sættes som vist i \autoref{animation2}
\begin{JavaCode}{Sæt duration}{animation2}
	minAnimation.setDuration(1000);	
\end{JavaCode}
Tiden er i millisekunder. Altså betyder 1000 et sekund. Fem sekunder ville være 5000 og et kvart sekund ville være 250. Værdien skal være et heltal.
\subsection{At lave en frame}
Nu har vi fortalt Android hvor den skal starte og slutte, og hvor lang tid animationen skal tage. Nu skal vi så fortælle Android om den skal flytte noget, rotere noget eller gøre noget helt tredje. Dette gør vi ved at skrive en funktion som bliver kaldt hver gang billedet skal animeres endnu et hak. Altså skal vi programmere hvad der sker hver gang det er tid til en ny frame. Koden ser ud som i \autoref{animation3}
\begin{JavaCode}{Opsæt metode til at lave et frame}{animation3}
	minAnimation.addUpdateListener(new ValueAnimator.AnimatorUpdateListener() {
		@Override
		public void onAnimationUpdate(ValueAnimator animation) {
			int vaerdiTilFrame = (int) animation.getAnimatedValue();
			
			mitView.setRotation(vaerdiTilFrame);
		}
	});
\end{JavaCode}
Her bruges noget advanceret kode som I ikke behøver at forstå. Den korte version er at vi giver Android funktionen onAnimationUpdate() og fortæller ValueAnimator at den skal kalde funktionen når den vil have en ny frame.\\
\marginnote{Hvis I vil have den længere udgave så spørg os eller internettet om anonyme klasser}
Den anden ting der sker er at vi får en integer fra animationen, som repræsenterer hvor langt vi er i animationen. Det er den variabel der hedder "vaerdiTilFrame" som man får fra getAnimatedValue(). Vi udfører så ændringen på vores view med denne værdi, i eksemplet sætter vi rotationen med setRotation, men I kan bruge denne værdi til lige hvad I vil. De mulige måder at ændre views på kommer om lidt.\\
Til sidst skal du kalde .start() på din ValueAnimator og så kører animationen. Et fuldt eksempel på en animation kan ses i \autoref{animation4}
\begin{JavaCode}{420 Rotate it}{animation4}
	ValueAnimator minAnimation = ValueAnimator.ofInt(0, 420);
	minAnimation.setDuration(1337);
	minAnimation.addUpdateListener(new ValueAnimator.AnimatorUpdateListener() {
		@Override
		public void onAnimationUpdate(ValueAnimator animation) {
			int vaerdiTilFrame = (int) animation.getAnimatedValue();
			mitView.setRotation(vaerdiTilFrame);
		}
	});
	minAnimation.start();
\end{JavaCode}
I kan nu lave animationer i Android. Huzzah!

\section{Flere grundteknikker}
Nu hvor det helt basale er på plads kan vi udvide med nogle flere teknikker
\subsection{Brug af decimaltal}
Man kan også oprette en ValueAnimator som bruger decimaltal i form af typen float istedet for at bruge int. Koden til dette ses i \autoref{animationfloat}
\begin{JavaCode}{decimaltal}{animationfloat}
	ValueAnimator minAnimation = ValueAnimator.ofFloat(0.0, 1.0);
\end{JavaCode}
På denne måde kan man få noget mere præcist animation istedet for at alting afrundes til heltal.

\subsection{Hvad kan vi ændre}
Her er der så en liste over hvad man kan lave med Views:\\
\begin{itemize}
	\item View rotation
	I kan sætte et views rotation som i \autoref{animationroter}
	\begin{JavaCode}{Roter et view}{animationroter}
		mitView.setRotation(jeresRotation);
	\end{JavaCode}
	Rotationen er i grader. 0 er normal rotation og 180 grader er på hovedet. Rotationen er en float. 
	\\
	\item View position
	Her sættes vertikal og horizontal position seperat. Koden ses i \autoref{animationflyt}
	\begin{JavaCode}{Flyt et view}{animationflyt}
		mitView.setX(jeresPositionX);
		mitView.setY(jeresPositionY);
	\end{JavaCode}
	Her skal det noteres at det er \textbf{øverste} venstre hjørne der er position (0,0). Ligeledes er det viewets øverste venstre hjørne der placeres. Hvis I vil placere et view op af skærmens højre side skal i altså sætte X til skærmbredden minus viewets bredde. I kan få bredden og højden af jeres view som i \autoref{animationbredde}
	\begin{JavaCode}{Få bredden af jeres view}{animationbredde}
		mitView.getWidth();
		mitView.getHeight();
	\end{JavaCode}
	Man kan også placere et views position ud fra at (0,0) er viewets startposition med funktionerne getTranslationX() og getTranslationY().
	\item View størrelse
	I kan sætte størrelsen som i \autoref{animationscale} 
	\begin{JavaCode}{Sæt størrelsen af jeres view}{animationscale}
		mitView.setScaleX(jeresStoerlseX);
		mitView.setScaleY(jeresStoerlseY);
	\end{JavaCode}
	Her er 1 den størrelse som viewet starter med. Skal størrelsen fordobles skal I give slutværdien 2. Skal den halveres skal I give slutværdi 0.5 etc. Hvis i vil sætte størrelsen til et præcist antal pixels skal i lave en udregning ud fra viewets nuværende størrelse. \\
	\item View gennemsigtighed
	Et view kan gøres gennemsigtigt (Man kan stadig interagere med det) med denne metoden set i \autoref{animationgennemsigtig}
	\begin{JavaCode}{Sæt et views gennemsigtighed}{animationgennemsigtig}
		mitView.setAlpha(jeresGennemsigtighed);
	\end{JavaCode}
	Her er 0 usynlig og 1 er fuldt synlig. Værdien som tages er selvfølgelig en float.
\end{itemize}
\subsection{Størrelsen af skærmen}
Der findes mange forskellige Android mobiler, og de har mange forskellige skærmstørrelser, især hvis man også arbejder med tablets. 
\begin{example}
	Den mobil I arbejder med har en skærmbredde på 480 pixels. I har et view som skal krydse skærmen og sætter det til at starte i 0 og bevæge sig 480 pixels. Og det virker lige som det skal på jeres test mobil. I prøver så at køre appen på en tablet med en skærmbredde på 1080 pixels. Jeres view når ikke engang halvvejs over skærmen. 
\end{example}
Det gælder derfor om altid at arbejde ud fra skærmens størrelse. Jeres view i eksemplet skal bevæge sig en hel skærmbredde. En figur som står i baggrunden og hopper skal hoppe 1/3 af skærmenhøjden. Etc.\\

Og siden skærmstørrelsen er så kritisk skulle man tro at den var lettere at få fat på. Men nej, koden for at få skærmstørrelsen ser ud som i \autoref{animation5}
\begin{JavaCode}{Kode til at få skærmstørrelse}{animation5}
	//Inde i onCreate()
	final View layout = findViewById(R.id.layout);
	ViewTreeObserver observer = layout.getViewTreeObserver();
	observer.addOnGlobalLayoutListener(new ViewTreeObserver.OnGlobalLayoutListener() {
		@Override
		public void onGlobalLayout() {
			screenHeight = layout.getHeight();
			screenWidth = layout.getWidth();
			//kode der bruger height og width her
			layout.getViewTreeObserver().removeOnGlobalLayoutListener(this);
		}
	});
\end{JavaCode}

I skal også give jeres yderste layout et ID og hente det med findViewByID(). Og så skal i have screenHeight og screenWidth som feltvariabler. Og hvis I skal bruge størrelsen af skærmen i onCreate() skal den kode også stå inde i onGlobalLayout().
Grunden hertil er at Android ikke har udregnet hvordan grafikken placeres når OnCreate bliver kaldt. Vi får den så til at kalde onGlobalLayout når den har fundet størrelsen af layoutet.
\subsection{Interpolatorer}
Det vi har defineret for Android er egentlig kun startpunkt, slutpunkt og tid imellem dem. Hvordan funktionen mellem de to punkter ser ud kan også defineres. Det kan være at den bare er lineær, men den kunne også være en exponentionel funktion hvor hastigheden starter med at være langsom og så derefter bliver hurtigere. Man styrer dette ved at sætte en interpolator. 
Hvis du ikke sætter en interpolator så vil Android bruge AccelerateDecelerateInterpolator. Altså vil hastigheden for animationen være langsommere i starten og slutningen og så være hurtig i midten. Du sætter en interpolator som i \autoref{animation6}
\begin{JavaCode}{Kode til at sætte interpolatoren}{animation6}
	animator.setInterpolator(new LinearInterpolator());
\end{JavaCode}

Du kan sætte interpolatoren mellem at du opretter animationen og starter den.
De helt basale muligheder er LinearInterpolator, AccelerateInterpolator, DecelerateInterpolator og så selvfølgelig AccelerateDecelerateInterpolator. Der findes dog flere som man kan finde online.
%TODO Illustrationer til interpolators
\subsection{Kontrolmetoder}
Animator har yderligere en række metoder som går den lettere at arbejde med. Du kan sætte animationen til først at starte noget tid efter at du kalder .start() ved at kalde koden i \autoref{animation7} \\
\begin{JavaCode}{Kode til at sætte et startDelay}{animation7}
	minAnimation.setStartDelay(minForsinkelse);
\end{JavaCode}
Forsinkelsen er igen i millisekunder, præcist som setDuration. 
Yderligere er der metoderne .pause(), .resume(), .cancel(),  .end() og .reverse() som meget vel giver sig selv. .end() sætter animationen til slutværdien og .cancel() efterlader den hvor den er nået til.
\section{Mere advancerede teknikker}
Her kommer vi omkring nogle lidt mere advancerede teknikker
\subsection{Flere animationer samlet}
Vi kan samle flere animationer i en, så vi kan flytte vores view diagonalt, eller få det til at ændre farve og rotere samtidig. Til dette bruger vi AnimatorSet. Det virker som set i \autoref{animation8}:
\begin{JavaCode}{Kode til at køre flere animationer samtidigt}{animation8}
AnimatorSet mitAnimationSet = new AnimatorSet();
mitAnimatorSet.playTogether(minAnimation, minAndenAnimation);
mitAnimatorSet.start();
\end{JavaCode}
Vi laver altså et AnimatorSet og så kalder vi .playTogether() med vores animationer som argumenter. Derefter starter vi vores sæt af animationer. Her skal det noteres at AnimatorSet opfører sig meget ligesom ValueAnimator og man kan stadig pause den og give den et startDelay m.m. 
Desuden kan man lave et AnimatorSet af andre AnimatorSets. Altså give et AnimatorSet som parameter til .playTogether(). Man kan desuden give playTogether lige så mange animationer som man har lyst til.
%Muligvis kortere navne til argumenter så linjen kan være i bogen
\begin{JavaCode}{Køre mange animationer samlet}{animation9}
	mitAnimatorSet.playTogether(minAnimation, minAndenAnimation, rotation, hop);
\end{JavaCode}

\subsection{Animationer i rækkefølge}
Vi kan også spille animationer efter hinanden. Her bruger vi igen AnimatorSet som ses i \autoref{animation10}
\begin{JavaCode}{Kode til at køre flere animationer efter hinanden}{animation10}
	AnimatorSet mitAnimationSet = new AnimatorSet();
	mitAnimatorSet.playSequentially(minAnimation, minAndenAnimation);
	mitAnimatorSet.start();
\end{JavaCode}
Så kører minAnimation først og derefter kører minAndenAnimation. Igen kan man give lige så mange animationer som man vil. 
%igen, muligvis forkort argument navne for at passe koden til siden
\begin{JavaCode}{Kør mange animationer efter hinanden}{animationer11}
	mitAnimatorSet.playSequentially(minAnimation, minAndenAnimation, minTrejdeAnimation);
\end{JavaCode}
\subsection{Gentagelser}
Vi kan også sætte vores animationer til at gentage. Det er meget simpelt, se \autoref{animation12}
\begin{JavaCode}{Kode til at lave gentagelser}{animation12}
	minAnimation.setRepeatCount(antalGentagelser);
\end{JavaCode}
Her er det værd at notere at hvis man sætter antalGentagelser til 2 så afspiller animationen \textbf{3 gange} i alt. Den gentager 2 gange.
Man kan også sætte gentagelsen som i \autoref{animation13}
\begin{JavaCode}{Sæt animationen til at køre tilbage}{animation13}
	minAnimation.setRepeatMode(ValueAnimator.REVERSE);
\end{JavaCode}
Hvilket gør at den på gentagelsen ikke går tilbage og kører forfra, men går fra slutværdien til startværdien. Den vil så igen gå fra start til slut på anden gentagelse etc.

\section{Hurtig guide}
Her er en lille 4 skridts guide til at lave en animation.
\begin{enumerate}
	\item Opret en ValueAnimator med enten ofInt() eller ofFloat() \autoref{animationguide1}
	\begin{JavaCode}{Opret animator}{animationguide1}
	ValueAnimator minAnimation = ValueAnimator.ofFloat(startFloat, slutFloat);
	\end{JavaCode}
	\item Sæt en varighed i millisekunder \autoref{animationguide2}
	\begin{JavaCode}{Sæt varighed}{animationguide2}
	minAnimation.setDuration(varighed);
	\end{JavaCode}
	\item Lav onAnimationUpdate og få den til at lave din ændring \autoref{animationsguide3}
	\begin{JavaCode}{Lav et billede}{animationsguide3}
		minAnimation.addUpdateListener(new ValueAnimator.AnimatorUpdateListener() {
			@Override
			public void onAnimationUpdate(ValueAnimator animation) {
				int vaerdiTilBillede = (int) animation.getAnimatedValue();
				//Manipuler dit view her
				mitView.setScale(vaerdiTilBillede);
			}
		});
	\end{JavaCode}
	\item Start animationen \autoref{animationguide4}
	\begin{JavaCode}{Start animationen}{animationguide4}
		minAnimation.start();
	\end{JavaCode}
\end{enumerate}

\subsubsection{Animationsopgaver}
Der er ingen tid afsat til specifikt at lave opgaver i animation, men programmering er et praktisk fag så det gælder om at øve sig. Her er nogle forslag til ting at øve sig med, men du kan også selv finde på udfordringer at øve dig på, inklusiv bare at gå direkte i gang med projektet. Hvis det ikke går godt så husk at tage små skridt og lave en ting af gangen.
\begin{exercise}
	Start helt fra bunden. Få en knap til at rotere 360 grader når du trykker på den.
\end{exercise}
\begin{exercise}
	Giv rotationen et startdelay så der går et øjeblik før den begynder at rotere.
\end{exercise}
\begin{exercise}
	Sæt din knap til at starte i venstre side af skærmen og så bevæge sig til midt på skærmen når du trykker på den.
\end{exercise}
\begin{exercise}
	Sæt bevægelsens interpolator til at være lineær
\end{exercise}
\begin{exercise}
	Kombiner de to første animationer så knappen først bevæger sig sidelæns og så derefter roterer
\end{exercise}
\begin{exercise}
	Sæt nu knappen til at starte i øverste højre hjørne. Få den til at bevæge sig lodret og vandret samtidigt til midt på skærmen, og derefter rotere en omgang. Noter hvordan at den lodrette bevægelse starter og slutter langsomt fordi den bruger den normale interpolator mens den horizontale bevægelse er lineær
\end{exercise}
\begin{exercise}
	Få rotationen til at gentage sig selv 2 gange. Sæt derefter dens repeatMode til at være reverse.
\end{exercise}
\begin{exercise}
	Lav en knap som bevæger sig yderligere 10 pixels til højre hver gang man trykker på den. Hint: Her ændrer startInt sig for hver animation.
\end{exercise}
\begin{exercise}
	Få helt styr på at arbejde med skærmens størrelse. Placer en knap i øverste venstre hjørne, og få den til at tage en fuld runde langs telefonens kant når du trykker på den. 
\end{exercise}

%Ordliste
%Frame
	
	\part{Miljø og konventioner}
	
	\graphicspath{{sections/ekstra/studio/figures/}}
	\chapter{Objekter og klasser}

I kernen af objekt-orienteret programmering er de to nøglebegreber; objekter og klasser. Når man skriver i et objektorienteret sprog vil de objekter, man opretter, vises i problemdomenet og udgør de dele af modellen, som dit computerprogram forsøger at emulere. Objekter kategoriseres af klasser. En klasse beskriver alle objekter af en bestemt type. Dette kan virke lidt abstrakt, så lad os forsøge at forstå dette ved hjælp af et eksempel på at simulere en boghandel.

\begin{example}
	Hvis vi ønsker at skrive et computerprogram, der er en simpel simulation af en boghandel, skal vi arbejde med et par forskellige enheder, hvoraf en kunne være bøger. Er en bog en klasse eller et objekt? Hvad er genren? Hvem er forfatteren? Hvornår blev det skrevet? Hvad er titlen?
	
	Alle ovenstående spørgsmål giver kun mening i forbindelse med en enkelt bog. Dette skyldes, at "bog" i denne sammenhæng refererer til klassen Bog, mens en enkelt bog er et objekt af denne klasse. En enkelt genstand henvises til som et eksempel. Du kan oprette flere forekomster af en klasse, som i ovenstående eksempel vil give os mulighed for at lave mange forskellige bøger i mange forskellige genrer med forskellige titler, forfattere og publikationsdatoer.
\end{example}

\section{Klasser}
Når vi skal lave en ny klasse i java, gør vi det typisk i en ny fil. Vi skal kalde filen det vi gerne vil kalde klassen, så hvis vi gerne vil oprette en klasse \JavaInline|Book| så starter vi med at lave en ny fil Book.java

Et eksempel på deklaration af en klasse ses i \autoref{lst:book-class-header}

\begin{JavaCode}{Book klasse deklaration}{lst:book-class-header}
	public class Book {
		// This is the class body
	}
\end{JavaCode}

Klasse deklarationen består essentielt af 3 ting, først en access modifier (\JavaInline|public|) der definerer hvorfra man kan få adgang til klassen, derefter ordet \JavaInline|class| der definerer at det er en klasse vi laver, sidst har vi navnet på vores klasse (\JavaInline|Book|). Bemærk at klassenavnet er stavet med stort forbogstav. Mellem de to tuborgklammer i klassens krop, det er der vi definerer hvad klassens rent faktisk kan bidrage med.

\subsection{Access modifiers}

I har set flere eksempler på brug af access modifiers allerede, både når vi definerer nye klasse, men også når vi definerer metoder. Access modifiers definerer hvorfra i vores kode vi vil kunne få fat i vores klasser og metoder. Der er 4 forskellige acces modifiers:

\subsubsection{Private}

i klasse

\subsubsection{public} 

over alt

\subsubsection{protected} 

i pakke, gennem nedarvning udefor pakke, kun på metoder, constructor og variable

\subsubsection{default} 

i pakke

\subsection{Fields}
En klasse kan have en række variable der kan holde information om hvilket state klassen er i lige nu, men også til at holde al den information der er fælles for alle instanser af den givne klasse men hvis værdi vil variere fra instans til instans disse betegnes fields. Når vi kigger på en bog, så har den en titel, en forfatter, en genre, måske et årstal, alle bøger har dette, men det vil ikke være det samme for hver bog, vi kan derfor have det som fields i vores bog klasse, se \autoref{lst:book-class-variables}

\begin{JavaCode}{Book klasse fields}{lst:book-class-variables}
	public class Book {
		//fields
		private String title;
		private String author;
		private String genre;
		private int publishYear;
		
		//rest of class body
	}
\end{JavaCode}

Læg mærke til at variablerne er private, dette er af hensyn til at de så ikke kan ændres udefra, men kun er tilgængelige for de metoder der ligger inde i klassen.

\subsection{Constructor}

En constructor er en særlig metode der bliver kaldt når man opretter et objekt af den givne klasse. Det er her vi gerne instantierer alle variable vi skal bruge i den givne klasse, og generelt sætter klassen op så den er klar til brug. Constructoren varierer fra andre metoder i at metodenavnet skal være det samme som klassenavnet, ellers fungerer det ligesom andre metoder, der kan have nogle parametre, en constructor kan dog aldrig returnere noget. Et eksempel på en constructor kan ses i \autoref{lst:book-class-constructor}

\begin{JavaCode}{Book klasse constructor}{lst:book-class-variables}
	public class Book {
		//fields
		private String title;
		private int publishYear;
		
		//constructor
		public book(String title, int publishedIn) {
			this.title = title;
			publishYear = publishedIn;
		}
		
		//rest of class body
	}
\end{JavaCode}

Constructoren tager imod nogle parametre vi gemmer i vores fields, på den måde er vores bog initialiseret med en titel og udgivelsesår, hvilket vi kan benytte når vi senere hen skal til at lave andre metoder i klassen. Bemærk at der står \JavaInline|this.title| for at referere til title field, dette skyldes at vi fortæller constructoren at vi vil assigne værdien af parametret title til field'et title, hvis vi ikke skrev \JavaInline|this.| så ville den assigne værdien til sig selv, og vores field ville stadig være tomt. Det er derfor ikke nødvendigt at skrive \JavaInline|this.| foran \JavaInline|publishYear| da der ikke er noget sammenfald i navgivningen og der ikke kan være tvivl om hvad der menes, det er dog helt i orden at gøre alligevel.

\subsection{Metoder}

Metoder virker på samme måde som vi allerede har set, der er dog to standard metoder man ofte ser i klasser, nemlig \JavaInline|get| og \JavaInline|set| metoder, disse gør det fx muligt at få adgang til at se eller ændre fields. 

Hvis vi gerne vil kunne se hvad der ligger i de forskellige fields uden for vores klasse skal vi tilføje metoder der er public som kan hive det ud, disse kaldes \JavaInline|get| metoder, da det henter information. I \autoref{lst:book-class-get}

\begin{JavaCode}{Book klasse get metode}{lst:book-class-get}
	public String getTitle() {
		return title;
	}
\end{JavaCode}

\JavaInline|get| metoder returnerer altid noget, og i sin rene form navngives den \JavaInline|get<<fieldname>>| og returner \JavaInline|<<field>>|.

Hvis det skal være muligt at ændre værdien i sine fields kan dette gøres med \JavaInline|Set| metoder, et eksempel på at sætte prisen på en bog kan ses i \autoref{lst:book-class-set}

\begin{JavaCode}{Book klasse set metode}{lst:book-class-set}
	public void setPrice(double price) {
		this.price = price;
	}
\end{JavaCode}

\JavaInline|set| metoder vil i sin rene form aldrig returnerer noget, men have et parameter der erstatter værdien af \JavaInline|<<field>>| og hedde \JavaInline|set<<field>>|

Vi ender dermed med den fulde klassedefinition der kan ses i \autoref{lst:book-class}

\begin{JavaCode}{Book klasse}{lst:book-class}
	public class Book {
		//fields
		private String title;
		private int publishYear;
		private double price;
		
		//constructor
		public book(String title, int publishedIn) {
			this.title = title;
			publishYear = publishedIn;
			price = 0.0;
		}
	
		//get methods
		public String getTitle() {
			return title;
		}
		public String getPublishYear() {
			return publishYear;
		}
		public String getPrice() {
			return price;
		}
		
		//set methods
		public void setPrice(double price) {
			this.price = price;
		}
	}
\end{JavaCode}

I dette eksempel er det muligt at ændre prisen, og der er ikke nødvendigt at kende prisen for at lave en bog, vi initialiserer dog prisen til 0 når vi laver bogen, så er vi sikker på at vi har initialiseret alle vores fields. Der er get metoder til alle fields så det er muligt at spørge klassen om hvad titlen, udgivelsesåret og prisen er på bogen. Der er kun opretter set metode til prisen, da det ikke giver mening at kunne ændre titel og udgivelsesår efter en bog er udgivet, hvis disse skulle ændres ser vi det som en ny bog.

\subsection{Nedarvning}

Alle klasser i Java nedarver fra objekt klassen

\section{Objekter}

hvad har vi allerede set af eksempler på

\subsection{Oprettelse af objekter}

Eksempel: lav bøger

\subsection{Kalde metoder på objekter}

Eksempel: get og set stuff på bog

\subsection{Casting}

\section{Klasser i Java der er gode at kende}

\subsection{Lists}

Arraylists

\subsection{Hashmaps}
Et HashMap er næsten ligesom en Arrayliste. Det er en datastruktur. Altså en måde at organizere en samling variabler på. Forskellen mellem en HashMap og en ArrayListe er hvordan man tilgår de gemte variabler. En ArrayListe giver simpelt hver variabel et tal, og så kan man bruge minArrayListe.get(tal); Ved HashMaps bestemmer I hvad hver variabel har af "nøgle" til at blive tilgået. i \autoref{hashmap1} er et eksempel med en simpel telefonliste.
\begin{JavaCode}{telefonListe}{hashmap1}
	HashMap<String, Integer> telefonliste = new HashMap<>();
	
	public void addContact(String name, int number){
		telefonliste.put(name, number);
	}
	
	public int getContact(String name){
		telefonliste.get(name);	
	}
\end{JavaCode}
Lad os gå igennem eksemplet. HashMaps skal have to typer hvor ArrayLister kun skal have en. \marginnote{Ved en faktisk implementation ville man nok bruge en anden type til et telefonnummer for at sikre at der er 8 cifre}Dette er fordi at Hashmaps både skal kende typen af nøglen, og typen af den variabel der gemmes. Her er nøglen sat til en string, da vi forbinder de gemte telefonnumre til et navn. Og den gemte variabel er en integer for der er et nummer.  For at tilføje elementer til listen bruger vi .put(nøgle, variabel); og for at få en variabel bruger vi .get(nøgle); Nøglen kan være af alle de normale typer, så man kan faktisk lave en ArrayListe med et HashMap som bruger integers som nøgler. Yderligere funktionalitet for HashMaps kan ses i \autoref{hashmap2}
\begin{JavaCode}{Metoder for Hashmap}{hashmap2}
	ArrayList<Integer> alleNumre = telefonliste.values();
	//Faa alle variabler i hashmappet som en arrayliste
	
	telefonliste.remove("Klods Hans");
	//Fjern en variabel fra hashmappet
	
	boolean harLeasy = telefonliste.containsValue(88888888);
	//True hvis variablen ligger i hashmappet
	
	boolean harSnehvide = telefonliste.containsKey("Snehvide");
	//True hvis noeglen ligger i hashmappet
	
	telefonliste.size();
	//Antal variabler i listen
\end{JavaCode}
En sidste bemærkning er at der kun kan være en variabel for hver nøgle. Hvis du putter en varibel med en nøgle som allerede er i mappet fjernes den gamle variabel som havde den nøgle. Altså den overskrives.

\subsection{Enumerables}
Enumerable er en speciel type klasse hvor objekter kan være fra et specifikt defineret sæt af muligheder. Lad os demonstere med \autoref{enums1}.
%TODO fix javacode her. Ikke alt tekst kommer med
\begin{JavaCode}{Ugedage}{enums1}
	public enum Weekday{
		MONDAY, TUESDAY, WEDNESDAY, THURSDAY, FRIDAY, SATURDAY, SUNDAY
	}
	
	if(today == Weekday.FRIDAY){
		party();
	}
\end{JavaCode}
Altså er det når man har brug for en variabeltype som repræsenterer en mulighed ud af en gruppe af elementer. Man kan oprette en enum klasse og så oprette variabler af den type og sætte dem til den type og sammenligne om de er af den type. Husk at bruge EnumNavn.elementNavn istedet for bare elementNavn når du opretter variablens værdi.


	
	
	\graphicspath{{figures/}}
	
	%% Indsæt ``glossary'', altså ordlisten, over alle de ord der er indsat ved 
	%% hjælp af \gls macroerne.
	\ifdraftmode
		\glsaddall
	\fi
	\printglossary
	
	%% Indsæt ``indexet'' som er et overblik over emner bogen indeholder, der 
	%% skabes ved hjælp af \index makroen.
	\input{index}
	
	%% Indsæt bagsiden.
	\input{butt-page}

\end{document}
