\usepackage[toc]{glossaries}

\makeglossaries

\newglossaryentry{returnvalue}
{
	name={retur-værdi},
	description={(\textit{return value, output}): Den værdi som en funktion 
				 producerer.},
	plural={retur-værdier}
}

\newglossaryentry{returntype}
{
	name={retur-type},
	description={(\textit{return type}): Typen af en funktions 
				 \gls{returnvalue}.},
	plural={retur-typer}
}

\newglossaryentry{argument}
{
	name={argument},
	description={(\textit{argument, input}): Hvad der sættes mellem 
				 parenteserne.},
	plural={argumenter}
}

\newglossaryentry{string}
{
	name={streng},
	description={(\textit{string}): Typen af tekst værdier.},
	plural={strenge}
}

\newglossaryentry{interface}
{
	name={grænseflade},
	description={(\textit{interface, UI, GUI}): Det man ser på skærmen af
		 		 mobilen.},
	plural={grænseflader}
}

\newglossaryentry{frame}
{
	name={frame},
	description={Et enkelt billede i en række billeder som danner en video},
	plural={frames}
}

\newglossaryentry{factorial}
{
	name={fakultets-funktion},
	description={(\textit{factorial}): Funktion der beregner n!},
	plural={fakultets-funktionen}
}