\chapter{Introduktion til bogens stil}
Dette kapitel er skrevet i filen \texttt{sections/manual/1/chapter.tex}.

\vspace{5mm}

\section{Noter i margen}

Vi kan bruge \LaTeXInline|\marginfigure{billede}{Caption}| til at 
indsætte et billede i margen. Her er et eksempel hvor vi skriver 
\LaTeXInline|\marginfigure{sample}{Some margin note here.}|, hvor 
``sample'' er et billede i den ``figures'' mappe der passer til dette dokument. 
Altså \texttt{sections/example/1/figures/}. 
\marginfigure{sample}{Some margin note here.}
Ligeledes kan vi bruge \LaTeXInline|\bottomfigure{billede}{Caption}|
til at indsætte et billede i bunden af siden.
\bottomfigure{sample}{Some bottom note here.}

Grundet \LaTeX\xspace magi, kan man komme ud for at skulle compile dokumentet 
to gange for at noter i margen virker. Det skal du ikke tænke for meget over.

\vspace{1cm}

Vi kan også nøjes med at indsætte tekst i margen ved 
\LaTeXInline|\marginnote{Tekst}|.
\marginnote{Sej note her}

\clearpage

\section{Kode og syntax highlighting}

JavaCode miljøet, lader os skrive Java-kode ind i vores .TeX fil, og få koden 
syntax-highlighted. Således kommer følgende \LaTeX kode:
\begin{LaTeXCode}{caption}{label}
\begin{JavaCode}{The fish class}{snippet1}
	public class Fish {
		public static void main(String[] args) {
			doAThing();
			int a = 32;
			String fish = tish;
			Abe b = new Abe(fish, a, "AbeMand");
			return;
		}
	}
\end{JavaCode}
\end{LaTeXCode}
Til at se så flot ud:
\begin{JavaCode}{The fish class}{lst:snippet1}
	public class Fish {
		public static void main(String[] args) {
			doAThing();
			int a = 32;
			String fish = tish;
			Abe b = new Abe(fish, a, "AbeMand");
			return;
		}
	}
\end{JavaCode}

Der er et tilsvarende miljø til XML, ved navn XmlCode.
\begin{XmlCode}{A XML document}{lst:xmlsnippet1}
	<?xml version="1.0" encoding="utf-8"?>
	<LinearLayout 
		xmlns:android="http://schemas.android.com/apk/res/android"
		android:layout_width="match_parent"
		android:layout_height="match_parent"
		android:orientation="vertical" >
		<TextView android:id="@+id/text"
			android:layout_width="wrap_content"
			android:layout_height="wrap_content"
			android:text="Hello, I am a TextView" />
		<Button android:id="@+id/button"
			android:layout_width="wrap_content"
			android:layout_height="wrap_content"
			android:text="Hello, I am a Button" />
	</LinearLayout>
\end{XmlCode}

Læg mærke til hvordan vi gav den en caption ``The fish class'' og en label 
``lst:snippet1'' således at vi nu kan referere til \autoref{lst:snippet1}, ved 
at skrive \LaTeXInline|\autoref{lst:snippet1}|.

\vspace{5mm}

\section{Boxes}

Her vil vi vise nogle eksempler på nogle content-boxes der kan laves.
\begin{samepage}
	\subsection{Example}
	\begin{LaTeXCode}{caption}{label}
	\begin{example}
		An example here
	\end{example}
	\end{LaTeXCode}
	\begin{example}
		An example here
	\end{example}
\end{samepage}

\subsection{Exercise}
\begin{LaTeXCode}{caption}{label}
\begin{exercise}
	An exercise here7
\end{exercise}
\end{LaTeXCode}
\begin{exercise}
	An exercise here
\end{exercise}

\subsection{Remark}
\begin{LaTeXCode}{caption}{label}
\begin{remark}
	Some remark here
\end{remark}
\end{LaTeXCode}
\begin{remark}
	Some remark here
\end{remark}

\section{Index i slutningen af bogen}
I slutningen af bogen er der mulighed for at tilføje en ``index'' over emnerne 
i bogen.
Disse emner kan blive tilføjet ved at bruge makroen \LaTeXInline|\index|, 
du 
skriver blot \LaTeXInline|\index{emne}|. Har du brug for 
at have et emne med et underemne, kan du bruge et ! til at adskille dem. 
\LaTeXInline|\index{Fisk!Makrel}|.