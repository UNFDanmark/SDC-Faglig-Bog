\chapter{Introduktion til \LaTeX}
Dette kapitel er skrevet i filen 
\texttt{sections/manual/intro-to-latex/chapter.tex}.

\vspace{5mm}

\section{Billeder og figurer}

For at indsætte billeder i \LaTeX, bruger man 
\LaTeXInline|\includegraphics[options]{billedefil}|. Dette wrapper man ofte i 
et \texttt{figure} miljø, sådan at det placeres et sted der giver mening. 
\LaTeX har en holdning til hvor billeder skal være og oftest skal man bare lade 
\LaTeX styre det for at få de bedste resultater. 

Det er dog sådan, at \LaTeX prøver at designe en bog for dig, hvor det 
forventes at der er mere tekst end figurer. Derfor skal man nogle gange hjælpe 
den lidt på vej, hvis man gerne vil have det til at se godt ud med mange 
figurer.

\begin{figure}[H]
	\IncludeWithFrame{handout-layout-pl}{Some caption here}{5cm}
\end{figure}