\chapter{Generelle LaTeX tips}
Her gennemgår jeg nogle generelle LaTeX tips, så der er mindre ``Googling'' 
involveret.

Det burde ikke være nødvendigt for jer at ændre på strukturen af dokumentet. 
Alle de nødvendige features skulle gerne være inkluderet, hvis ikke skal vi 
sørge for at få dem tilføjet. Så kontakt venligst \LaTeX guruerne for at få 
tilføjet manglende features. Såsom highlighting af et programmeringssprog eller 
et tip til hvordan man løser et problem.

\section{Linjeskift}
I \LaTeX er det velanset at adskille sine paragraffer ved at bruge to 
linjeskift. F.eks.
\begin{verbatim}
Første paragraf

Anden paragraf
\end{verbatim}
Ved at bruge denne teknik, kan man indstille hvordan linjeskift fungerer 
igennem indstillinger fra preamblen. Hvis man bruger 
\textbackslash\textbackslash kan man tvinge \LaTeX til at lave et linjeskift 
uden at indentere den næste paragraf. Dette skal dog undgås, og generelt ville 
det være at foretrække at gøre det på følgende måde:
\begin{verbatim}
Første paragraf

\noindent Anden paragraf
\end{verbatim}

\noindent Du kan styre præcis hvor meget rum der er mellem dine paragraffer ved 
hjælp af 
\textbackslash vspace makroen.
\begin{verbatim}
Første paragraf
\vspace{5mm}
Anden paragraf
\end{verbatim}

\section{Billeder og figurer}
Du kan inkludere forskellige former for grafik i dine dokumenter ved hjælp af 
\textbackslash includegraphics makroen. Dette er standard \LaTeX, og en Google 
søgning vil tjene dig godt her.

\section{Tabeller}

\section{Matematik}

\section{Formatering af tekst}

\section{Referencer}