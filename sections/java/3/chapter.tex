\chapter{Objekter og klasser}

I kernen af objekt-orienteret programmering er de to nøglebegreber; objekter og klasser. Når man skriver i et objektorienteret sprog vil de objekter, man opretter, vises i problemdomenet og udgør de dele af modellen, som dit computerprogram forsøger at emulere. Objekter kategoriseres af klasser. En klasse beskriver alle objekter af en bestemt type. Dette kan virke lidt abstrakt, så lad os forsøge at forstå dette ved hjælp af et eksempel på at simulere en boghandel.

\subsection{Et simpelt eksempel}

Hvis vi ønsker at skrive et computerprogram, der er en simpel simulation af en boghandel, skal vi arbejde med et par forskellige enheder, hvoraf en kunne være bøger. Er en bog en klasse eller et objekt? Hvad er genren? Hvem er forfatteren? Hvornår blev det skrevet? Hvad er titlen?

Alle ovenstående spørgsmål giver kun mening i forbindelse med en enkelt bog. Dette skyldes, at "bog" i denne sammenhæng refererer til klassen Bog, mens en enkelt bog er et objekt af denne klasse. En enkelt genstand henvises til som et eksempel. Du kan oprette flere forekomster af en klasse, som i ovenstående eksempel vil give os mulighed for at lave mange forskellige bøger i mange forskellige genrer med forskellige titler, forfattere og publikationsdatoer.

\section{Klasser}

Hvordan man laver og skriver klasser

Hvad en klasse består af

eksempel: bog klasse

\subsection{Access modifiers}

private (i klasse)

public (over alt)

protected (i pakke, gennem nedarvning udefor pakke, kun på metoder, constructor og variable)

default (i pakke)

\subsection{Constructor}

hvordan fungerer en constructor

eksempel: bog constructor

\subsection{Metoder}

samme som funktioner fra kapitel 2, men husk at overvej hvem der skal have adgang til metoderne

Eksempel: get og set metoder for bog

\subsection{Nedarvning}

\section{Objekter}

Hvordan laver vi dem, hvordan bruger vi dem

Eksempel: lav bøger

\subsection{Casting}

\section{Klasser i Java der er gode at kende}

\subsection{Lists}

\subsection{Hashmaps}

\subsection{Enumerables}


