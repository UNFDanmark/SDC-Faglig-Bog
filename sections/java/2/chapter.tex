
%%%%%%%%%%%%%%%%%%%%%%%%%%%%%%%%%%%%%%%%%%%%%%%%%%%%%%%%%%%%%%%%%%%%%%%%%%%%%%%%
%
% Hov! Hej. Undskyld for rodet, det er bare sådan at jeg kan have pæne TODO
% kasser, og som er nemme at se blandt al den skide tekst.

\newtheoremstyle{redbox}% % Theorem style name
{0pt}% Space above
{0pt}% Space below
{\normalfont}% % Body font
{}% Indent amount
{\small\bf\sffamily\color{red}}% % Theorem head font
{\;}% Punctuation after theorem head
{0.25em}% Space after theorem head
{\small\sffamily\color{red}\thmname{#1}%\@ifnotempty{#1}{}\@upn{#2}}%
	% Theorem text (e.g. Theorem 2.1)
	\thmnote{\nobreakspace\the\thm@notefont\sffamily\bfseries\color{black}---\nobreakspace#3.}}

\theoremstyle{redbox}
\newtheorem{todoT}{TODO}

% Exercise box
\newmdenv[skipabove=7pt,
skipbelow=7pt,
rightline=false,
leftline=true,
topline=false,
bottomline=false,
backgroundcolor=red!10,
linecolor=red,
innerleftmargin=5pt,
innerrightmargin=5pt,
innertopmargin=5pt,
innerbottommargin=5pt,
leftmargin=0cm,
rightmargin=0cm,
linewidth=4pt]{todoBox}

\newcommand{\todo}[1]{
	\begin{todoBox}\begin{todoT}
		#1
	\end{todoT}\end{todoBox}
}

%%%%%%%%%%%%%%%%%%%%%%%%%%%%%%%%%%%%%%%%%%%%%%%%%%%%%%%%%%%%%%%%%%%%%%%%%%%%%%%%

\chapter{Kontrol og Metoder}

	I sidste kapitel blev variabler, aritmetiske udtryk og \emph{if-statements}
	introduceret. Med disse dele kan man lave det der heder et
	\emph{straight-line program}. Men vi er ofte stillede over for problemer som
	ikke kan løses ekslusivt med sådanne programmer. Vi behøver mere
	udtrykskraft, end hvad aritmetik og basal beslutningsevner kan give os.
	For eksempel vil vi gerne kunne sige ting som ``gør \emph{det her} indtil
	\emph{dette} sker'' eller ``gør \emph{dette}, først for \emph{1}, og så for
	\emph{2}, osv.''

	Af denne grund indeholder mange programmerings-sprog, inklusiv Java,
	mekanismer for at kunne udtrykke sådanne problemer, og det er disse
	mekanismer vi vil dække i dette kapitel.

	Emner i dette kapitel:

	\begin{itemize} % Denne udgiver har også følgende selvhjælps-bøger på markedet:
		\item Looping: Sådan får du mere kontrol
		\item Funktioner: Dig og dine metoder
		\item Algoritmer: Find din indre effektivitet
	\end{itemize}

\section{Looping}

	\todo{
		Denne sektion vil dække \JavaInline{for}, \JavaInline{while} og \JavaInline{break}.
	}

	\todo{
		Som bonus vil jeg gerne have noter om \JavaInline{do ... while} og
		\JavaInline{continue}, da disse er niché, men er samtidige nemme at
		forstå.

		Jeg vil ikke have noter om mere niché ting, som f.eks
		\emph{named-breaks}, da disse kræver introduktion til nye koncepter.
	}

	\todo{
		Som perspektiverende bonus kunne jeg tænke mig at fortælle lidt af
		historien om \emph{structured programming}, og hvorfor \JavaInline{goto}
		ikke er en ting i Java. \emph{Ikke pensum, bare for sjov.}
	}

	\begin{exercise}
		Det er ofte fortalt hvordan at Gauss i sine unge dage, forpurrerede sin
		lærers plan om at få ham til at tie stille. Hun bedte ham lægge tallene
		fra \(1\) til \(100\) sammen, i håbet at dette ville optage det unge
		geni i lidt tid. Gauss realiserede dog at der måtte være en bedre måde
		at løse problemet på, og udledte formularen \(\frac{n\cdot(n+1)}{2}\).
		Han svarede hurtigt \(5050\), og så måtte læreren finde på en anden
		kedelig opgave.

		Brug et \emph{for-loop} til at beregne \(1^2+2^2+\dots+100^2\).
	\end{exercise}

	\begin{exercise}
		Der er selvfølgelig ikke noget som forhindrer at du bruger et
		\emph{for-loop} ind i et andet \emph{for-loop}, og dette kan nogle gang
		være vældig praktisk.

		TODO: Find på opgave!
	\end{exercise}


\blindtext

\section{Funktioner}

	\todo{
		Denne sektion vil dække \emph{metode-definitioner}, \emph{retur-typer},
		\emph{argument-lister} og \JavaInline{return}.
		Jeg vil ikke dække statiske metoder, overloading, eller andre advanceret
		emner, da disse igen er ret niché, og kræver introduktion til nye
		koncepter.
	}

	\todo{
		\emph{pass-by-value} vs. \emph{pass-by-reference} bør nok dækkes, da det
		garenteret vil dukke op. Det er bare ikke specielt relevant før de kender
		til objekter. Antager ikke de vil rende ind i problemet før under projektet.
		Jeg er ikke vild med at introducere dem til sådanne niché fagtermer, men
		det vil være nemmere for en hjælper at kunne sige ``læs dette stykke om
		hvorfor dit program ikke fungere korrekt.''
	}

\blindtext

\section{Algoritmer eller Lister}

	\todo{
		Jeg kan ikke helt beslutte mig, hvad denne sidste sektion skal være om; enten
		kort om algoritmer; eller røre lidt ved lister.
		Understående giver pitches for begge.
	}

	\todo{
		Denne forlæsning og sektion dækker kort hvad Fanden en algoritme er, og
		giver simple eksempler, som viser hvordan man kan anvende dette kapitels
		tidligere emner.
		Et klart eksempel er fakultet-funktionen (eller muligvis fibonnaci,)
		da denne både kan udtrykkes med en for-lykke, rekursivt, og endda
		tail-rekursivt med en akkumulator! (Denne sidste er klart for
		advanceret, men min indre Olivier kan ikke lade vær.)
	}

	\todo{
		Lister er en fundamentel datastruktur, sandeligt er datamaten selv en
		kæmpe liste af data.
		Denne sektion vil dække \JavaInline{ArrayList} eller måske bare
		\JavaInline{int[]}? Jeg ved det ikke. Den første kræver introduktion til
		objekter, og den anden kræver introduktion til arrays og subscripts.
		Ingen af dem er elegante eller simple.
	}

	\todo{Note til Lukas: Kan vi bruge en mono-spaced font til kode, både
	inline og til blokke? Den nuværende er hæslig til begge dele. Er meget tilfreds med lstautogobble!}

	\begin{JavaCode}{Iterativ Fakultet}{lst:factorial-iterative}
		public int factorial (int n) {
			int a = 1;
			for (int i = 1; i <= n; i++)
				a = a * i;
			return a;
		}
	\end{JavaCode}

	\begin{JavaCode}{Rekursiv Fakultet}{lst:factorial-recursive}
		public int factorial (int n) {
			if (n <= 1)  return 1;
			return n * factorial(n-1);
		}
	\end{JavaCode}

	\begin{JavaCode}{Akkumulator Fakultet}{lst:factorial-accumulator}
		public int factorial (int n, int a) {
			if (n <= 1)  return a;
			return factorial(n-1, n * a);
		}
	\end{JavaCode}

	\begin{JavaCode}{Fakultet funktionen hvis Java var godt}{lst:factorial-good}
		fun factorial 0  =  1
		  | factorial n  =  n * factorial (n - 1)
	\end{JavaCode}

\blindtext
