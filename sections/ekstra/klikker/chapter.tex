\chapter{Din første app}

Nu har I lært en hel masse om at udvikle apps, og det er tid til at bruge den viden i et samlet projekt. I skal derfor lave jeres første lille app, som er et clicker spil. I kender muligvis allerede populære clicker spil såsom Cookie Clicker, Clicker Hero eller Adventure Capitalist men hvis I ikke har hørt om konceptet før så er det meget simpelt. Man klikker på en knap for at indsamle cookies, penge eller gode ideer til hvad man kunne skrive som et tredje eksempel. Og det gælder om at samle så mange som muligt og så købe opgraderinger så man kan indsamle point hurtigere for at købe ting så man kan indsamle point endnu hurtigere.

\begin{exercise}
	Lav et nyt projekt i Android studio. Company Domain skal være "dk.unf.software". Husk at sætte minimum SDK til API 15 og vælg en Emtpy Acitivty som jeres Main activity.
	Tjek at I kan køre jeres tomme og meningsløse app.
	%TODO Find det rigtitge minimum API nummer
\end{exercise}

Nu har I en tom og meningsløs app. Lad os få implementeret kernen af et clicker spil
\begin{exercise}
	Lav en knap at clicke på, og et tekstfelt som tæller antal gange der er trykket på knappen.
\end{exercise}
Hurra, I har nu lavet en simpel clicker app! Hvad med at gøre den \textit{lidt} mere interesant?
\begin{exercise}
	Lav en knap hvis funktion simpelt gør at antallet af point man får per click bliver større
\end{exercise}
Jeres nye knap fungerer næsten som en opgradering. Men det er lidt for let når det bare er en gratis opgradering. 
\begin{exercise}
	Sæt en pris på opgraderingen, og træk prisen fra det totale antal point når den købes
\end{exercise}
Tryk på opgraderingsknappen helt i starten af spillet. Så får man et negativt antal point. Det går ikke, for man må ikke låne i et clicker spil
\begin{exercise}
	Sørg for at man kun kan købe jeres opgradering hvis man har nok point
\end{exercise}
Så er der styr på det. Men det er ikke sjovt kun at kunne købe en opgradering en gang.
\begin{exercise}
	Gør sådan at man kan købe opgraderingen lige så mange gange man har lyst
\end{exercise}
En type opgradering er ikke nok. 
\begin{exercise}
	Tilføj to yderligere knapper som giver bedre, dyrere opgraderinger.
\end{exercise}
Selv med fancy opgraderinger er man selv nødt til at klikke på knappen. Det bliver bare kedeligt efter noget tid.
\begin{exercise}
	Lav en opgradering som gør at der automatisk bliver trykket på knappen i et fast tidsinterval.
	Her skal du oprette og bruge et objekt som er en CountDownIimer. Prøv selv at finde ud af hvordan du implementerer den ved at søge på nettet.
\end{exercise}
Internettet har rigtig mange tips og tricks til at programmere. Du kan ofte finde løsningen på et programmerings problem ved at søge efter det.\\
Autoclickeren er meget fed, men lad os gøre den endnu bedre
\begin{exercise}
	Gør sådan at autoclickeren stopper når man trykker på knappen til autoclickeren igen.
\end{exercise}
\begin{exercise}
	Giv autoclicker opgraderingen en pris. Gør sådan at man kun skal betale første gang man starter den. 
\end{exercise}
I har nu en ret sej clickerapp. Men I kan sikkert gøre den endnu federe. Herfra er der fri leg med hvad I vil tilføje. 
\begin{exercise}
	Arbejd videre med det I lige har lyst til! \\
	Måske lav nogle flere opdateringer? Eller tillade opgradering af autoclickeren? Lave en seperat Activity til alle jeres opgraderinger?
	Muligvis et fedt layout? Måske give appen et fedt eventyr tema? Der er mange muligheder, så bare design og kod løs!
\end{exercise}\section{Clicker app}
Nu har I lært en hel masse om at udvikle apps, og det er tid til at bruge den viden i et samlet projekt. I skal derfor lave jeres første lille app, som er et clicker spil. I kender muligvis allerede populære clicker spil såsom Cookie Clicker, Clicker Hero eller Adventure Capitalist men hvis I ikke har hørt om konceptet før så er det meget simpelt. Man klikker på en knap for at indsamle cookies, penge eller gode ideer til hvad man kunne skrive som et tredje eksempel. Og det gælder om at samle så mange som muligt og så købe opgraderinger så man kan indsamle point hurtigere for at købe ting så man kan indsamle point endnu hurtigere.

\begin{exercise}
	Lav et nyt projekt i Android studio. Company Domain skal være "dk.unf.software". Husk at sætte minimum SDK til API 15 og vælg en Emtpy Acitivty som jeres Main activity.
	Tjek at I kan køre jeres tomme og meningsløse app.
	%TODO Find det rigtitge minimum API nummer
\end{exercise}

Nu har I en tom og meningsløs app. Lad os få implementeret kernen af et clicker spil
\begin{exercise}
	Lav en knap at clicke på, og et tekstfelt som tæller antal gange der er trykket på knappen.
\end{exercise}
Hurra, I har nu lavet en simpel clicker app! Hvad med at gøre den \textit{lidt} mere interesant?
\begin{exercise}
	Lav en knap hvis funktion simpelt gør at antallet af point man får per click bliver større
\end{exercise}
Jeres nye knap fungerer næsten som en opgradering. Men det er lidt for let når det bare er en gratis opgradering. 
\begin{exercise}
	Sæt en pris på opgraderingen, og træk prisen fra det totale antal point når den købes
\end{exercise}
Tryk på opgraderingsknappen helt i starten af spillet. Så får man et negativt antal point. Det går ikke, for man må ikke låne i et clicker spil
\begin{exercise}
	Sørg for at man kun kan købe jeres opgradering hvis man har nok point
\end{exercise}
Så er der styr på det. Men det er ikke sjovt kun at kunne købe en opgradering en gang.
\begin{exercise}
	Gør sådan at man kan købe opgraderingen lige så mange gange man har lyst
\end{exercise}
En type opgradering er ikke nok. 
\begin{exercise}
	Tilføj to yderligere knapper som giver bedre, dyrere opgraderinger.
\end{exercise}
Selv med fancy opgraderinger er man selv nødt til at klikke på knappen. Det bliver bare kedeligt efter noget tid.
\begin{exercise}
	Lav en opgradering som gør at der automatisk bliver trykket på knappen i et fast tidsinterval.
	Her skal du oprette og bruge et objekt som er en CountDownIimer. Prøv selv at finde ud af hvordan du implementerer den ved at søge på nettet.
\end{exercise}
Internettet har rigtig mange tips og tricks til at programmere. Du kan ofte finde løsningen på et programmerings problem ved at søge efter det.\\
Autoclickeren er meget fed, men lad os gøre den endnu bedre
\begin{exercise}
	Gør sådan at autoclickeren stopper når man trykker på knappen til autoclickeren igen.
\end{exercise}
\begin{exercise}
	Giv autoclicker opgraderingen en pris. Gør sådan at man kun skal betale første gang man starter den. 
\end{exercise}
I har nu en ret sej clickerapp. Men I kan sikkert gøre den endnu federe. Herfra er der fri leg med hvad I vil tilføje. 
\begin{exercise}
	Arbejd videre med det I lige har lyst til! \\
	Måske lav nogle flere opdateringer? Eller tillade opgradering af autoclickeren? Lave en seperat Activity til alle jeres opgraderinger?
	Muligvis et fedt layout? Måske give appen et fedt eventyr tema? Der er mange muligheder, så bare design og kod løs!
\end{exercise}