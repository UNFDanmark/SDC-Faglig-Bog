\cleardoublepage
\chapter{Sådan læser du bogen}
Hele vejen igennem denne bog, bruger vi forskellige visuelle elementer for at gøre bogen mere læselig og lettere at overskue.

\section{Elementer}

\subsection{Bemærkning}
\begin{remark}
	Dette er en bemærkning, den hjælper dig med at få øje på detaljer der er ekstra vigtige.
\end{remark}

\subsection{Eksempel}
\begin{example}
	Dette er et eksempel på et eksempel. Det indeholder et eksempel der for eksempel kunne vise et eksempel på en ligning. 
	$$2+x=4$$
\end{example}

\subsection{Opgave}
\begin{exercise}
	Dette er en opgave, den må du hellere lave hvis du gerne vil lære noget!
\end{exercise}

\subsection{QR-Koder}
Dette er en QR-Kode, dem har vi indsat i nærheden af billeder. Hvis du scanner den med din mobil, kan den tage dig til en digital version af billedet, som du kan zoome ind på så du kan se de små detaljer.

\qrcode{https://raw.githubusercontent.com/UNFDanmark/SDC-Faglig-Bog/master/sections/introduction/figures/qr-code.png}

\subsection{Kode-stykke}

\begin{JavaCodeH}{Eksempel på et kode-stykke.}{lst:example}
	// Dette er et kode-stykke.
	// Det gør koden i bogen lettere at læse 
	// ved at give den flotte farver!
	int tal = 24 + 24;
\end{JavaCode}