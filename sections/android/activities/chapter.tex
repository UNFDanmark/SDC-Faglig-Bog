% !TeX spellcheck = da_DK
\chapter{Activities og Intents}
\label{cha:activities-intents}

Indtil videre har vi kigget kort på vores MainActivity som er entry point for en app. Hvis vi gerne vil have en ny activity, så skal vi starte denne med en intent. Dette kapitel vil kigge nærmere på en activity og dens livscyclus, forklare oprettelsen af nye activities, samt hvordan man kan starte disse activities med intents.

\section{Activities}

En activity er et komponent i en app, typisk er apps bygget op af flere aktiviteter hvor hver activity har sin egen brugergrænseflade og layout, dermed kan en activity lidt sammenlignes med et vindue man åbner på sin computer. Activities ligger som en stak i applikationen, hvilket betyder at når man åbner en activity svarer det til man åbner et nyt vindue der ligger over det gamle (og ofte skjuler det) og når den nye activity lukkes kommer man så tilbage til den activity man i sin tid åbnede den fra.

MainActivity er altid der vi starter. Når vi åbner en app, så det der egentlig sker er, at vi starter vores MainActivity.


\subsection{Activity life cycle}

\begin{figure}[h]
	\begin{center}
		\includegraphics[width=7cm]{developerandroidcom_activitylifecycle.png}
		\caption{Activity lifecycle lånt af developer.android.com}
		\label{fig:android:activities:activitylifecycle}
	\end{center}
\end{figure}

I \autoref{fig:android:activities:activitylifecycle} viser de forskellige stadier en activity er i, og kan hjælpe til at give et overblik over hvornår de forskellige metoder bliver kaldt, og dermed også hvad man gerne vil have skal ske i de forskellige steps undervejs.

En activity vil typisk befinde sig i et af 4 stadier. Når activity'en er i forgrunden og brugeren kan interagere med den siger vi at den kører (running). Hvis en anden activity startes i forgrunden men den gamle stadig er synlig siger vi at den er paused. Hvis en anden activity åbnes så den gamle activity ikke længere kan ses siger vi det er stopped. Paused og stopped activities kan lukkes af systemet for at frigøre hukommelse, i det tilfælde skal activity'en genstartes når den skal bruges igen.

\subsection{Oprettelse af nye activities}

For at oprette en ny activity starter vi med at lave en ny java fil ved siden af MainActivity.java (i ??? folderen) En ny activity skal altid nedarve fra Activity, ligesom MainActivity. 

\begin{JavaCode}{Eksempel på en activity}{pop-up-activity}
	package com.example.housa.myapplication
	
	import android.app.Activity;
	import android.os.bundle;
	
	public class PopUpActivity extends Activity {
 
 @Override
 protected void onCreate(Bundle savedInstanceState) {
  super.onCreate(savedInstanceState);
  setContentView(R.layout.activity_popup);
 }
	}
\end{JavaCode}

Ud over at skrive selve Java koden der definerer opførslen for en activity skal den til føjes til manifestet. Dette gøres ved at tilføje koden i \autoref{lst:add-activity-to-manifest} til manifestet inde i application tag'et.

\begin{XmlCode}{XML der tilføjer PopUpActivity til manifestet}{lst:add-activity-to-manifest}
	<activity android:name=".PopUpActivity" />
\end{XmlCode}

For at kunne starte ovenstående activity skal vi benytte os af Intents som vil blive forklaret i næste sektion.


\section{Intents}

En intent er en besked man kan sende for at bede en anden komponent om at gøre noget. Intents kan bruges på flere forskellige måder, men de tre primære brugsscenarier er:

\subsubsection{Starte en anden activity}

Man kan starte en ny instans af en activity ved at sende en intent med som parameter til \JavaInline|startActivity()|. 

\subsubsection{Starte en service}

En service er et komponent der kører i baggrunden og ikke har noget interface. Man starter en service på lidt samme måde som en activity, nemlig ved at sende en intent med som parameter til metoden \JavaInline|startService()|.

\subsubsection{Sende en broadcast}

En broadcast er en besked som enhver app kan modtage. Der er forskellelige broadcasts for system events som fx foretag opkald eller opladning igang. Man kan sende en broadcast ved at sende en intent med som parameter til \JavaInline|sendBroadcast()| eller \JavaInline|sendOrderedBroadcast()|.
En standard broadcast kan ikke blive stoppet, og kan ikke videregive resultater. En orderedBroadcast vil sende broadcasten til alle relevante BroadCastRecievers en af gangen og tillade resultatet at propagere.

\subsubsection{Toasts}

En toast er en lill besked som popper op på skærmen de sættes ind ved hjælp af de her funktioner  

\begin{JavaCode}{Her kan man f.eks. vælge mellem en lang og en kort toast}{lst:toast-example}
	Toast toast = Toast.makeText(getActivity(), "Det her er min toast besked!", Toast.LENGTH_LONG.show(););
	toast.show();
	// Eller Toast.LENGTH_SHORT.show();
\end{JavaCode}

Som man kan se i \autoref{lst:toast-example} har toasteren brug for en activity, en besked og, hvor lang tid den skal vises.

\subsection{Explicit vs Implicit}

Der er to typer intents explicitte og implicitte. I explicitte intents specificerer man præcist hvilket komponent man vil have til at reagere. Da man kender navngivningen i ens egen app, vil dette være den typeske måde at håndtere kommunikationen indenfor ens egen app, som at starte nye activities inde i appen.

De implicitte intents fortæller ikke præcist hvilket komponent man vil have til at reagere, men i stedet hvilken handling man gerne ville have udført, så kan alle komponenter der har den givne funktionalitet tilbyde at udføre handlingen. Dette benyttes fx hvis man gerne vil have ens app til at sende en mail, men gerne bare vil bruge hvad end mailklient brugeren allerede har på telefonen.

\subsection{Start en activity med en Intent}

Først laves det intent vi vil starte vores activity med, her benyttes et explicit intent da det skal benyttes til at starte en specifik activity i vores applikation. For at vores intent er explicit skal vi definere hvilken klasse den skal sendes til. For at starte den valgte activity kalder vi derefter \JavaInline|startActivity|.

\begin{JavaCode}{Eksempel på et intent}{explicit-intent}
	Intent intent = new Intent(this, PopUpActivity.class);
	startActivity(intent);
\end{JavaCode}

Vi kan sende en besked med vores itent ved at benytte \JavaInline|putExtra| metoden der tager en nøgle og en værdi som input.

\begin{JavaCode}{Sende en besket med et intent}{putExtra-intent}
	public static final String POPUP_MESSAGE = "com.example.lukas.myapplication.POPUP_MESSAGE";
	...
	Intent intent = new Intent(this, PopUpActivity.class);
	intent.putExtra(POPUP_MESSAGE, "Hello World!");
	startActivity(intent);
\end{JavaCode}

\subsubsection{Adgang til meddelelsen fra et intent}

For at få fat i den besked vi sender med vha \JavaInline|putExtra| kan vi benytte \JavaInline|getIntent| fra den activity der bliver startet med vores intent. Vi kan få fat i værdien i beskeden med \JavaInline|intent.get(Type)Extra(key)| \todo{intent.get<<Type>>Extra(key)?} hvor ``\JavaInline|(Type)|'' skal udskiftes med typen på den værdi man vil have fat i.

\begin{JavaCode}{Adgang til beskeden fra et intent}{getExtra-intent}
	public class PopUpActivity extends Activity {
		
		@Override
		protected void onCreate(Bundle savedInstanceState) {
			super.onCreate(savedInstanceState);
			setContentView(R.layout.activity_popup);
			
			Intent intent = getIntent();
			String message = intent.getStringExtra(MainActivity.POPUP_MESSAGE);
			TextView textView = (TextView)findViewById(R.id.messageView);
			textView.setText(message);
			textView.invalidate();
		}
	}
\end{JavaCode}

\subsection{Start en activity fra en anden app med en Intent}

Vi kan inkorporere aktiviteter fra andre apps, hvis de lytter efter intents. Et eksempel er en e-mail-app, vi kan bruge den indbyggede e-mail-klient (eller en anden e-mail-klient) til at sende en email fra vores app. Vi skal bruge implicitte intents til dette formål, da vi ikke nødvendigvis kender de klasse samt sti til e-mail-klientens aktivitet, og vi vil muligvis gerne have brugeren til at vælge den e-mail-app, som de bedst kan lide.

For at skabe sådan en implicit intent skaber vi simpelthen en ny intent, sætter den relevante handling og giver den de nødvendige metadata, så aktiviteten, der modtager intent'en, kan fuldføre opgaven.

Vi vil genbruge e-mail-eksemplet. Hvis vi ønsker at sende en email fra vores app, bruger vi handlingen \JavaInline|Intent.ACTION_SEND|. Handlingen \JavaInline|ACTION_SEND| betyder, at vi vil sende nogle data.

\begin{JavaCode}{Eksempel på et implicit intent}{lst:implicit-intent}
	Intent mailIntent = new Intent();
	mailIntent.setAction(Intent.ACTION_SEND);
\end{JavaCode}

Nu, at vi har en intent med den relevante handling, skal vi indstille metadataene for vores intent således at den modtagende aktivitet ved, hvad de skal gøre med intent'en. Hvis du sender en mail, skal du sætte typen af intent'en til ``text/html'' og e-mail-adresse, emne og indhold til e-mailen tilføjes som vist i \autoref{lst:mailsending-intent}:

\begin{JavaCode}{Eksempel på et mail sendende intent}{lst:mailsending-intent}
	public void sendMail(View view) {
		Intent mailIntent = new Intent();
		mailIntent.setAction(Intent.ACTION_SEND);
		mailIntent.setType("text/html");
		mailIntent.putExtra(Intent.EXTRA_EMAIL, "test@someone.com");
		mailIntent.putExtra(Intent.EXTRA_SUBJECT, "NEWS");
		mailIntent.putExtra(Intent.EXTRA_TEXT, "Some mail content");
		startActivity(Intent.createChooser(mailIntent, "Send Mail"));
	}
\end{JavaCode}

\subsection{Start en activity med en Intent fra en anden app}

For at modtage en implicit intent skal du oprette den activity, der skal modtage intent'en, for eksempel kan det være en activity, der fanger ``send'' intent'en, men i stedet for at sende den, viser den faktisk mailen.

Først skal du oprette activity'en. Bare lav en tom aktivitet som i \autoref{lst:empty-activity}.

\begin{JavaCode}{Tom activity}{lst:empty-activity}
	public class FakeMailActivity extends Activity {
		@Override
		protected void onCreate(Bundle savedInstanceState) {
			super.onCreate(savedInstanceState);
		}
	}
\end{JavaCode}

Tilføj nu den nye activity til manifestet, og giv den et filter, der fanger intents sendt for ``text/html'' mimeTypes.

\begin{XmlCode}{XML der tilføjer FakeMailActivity til manifestet}{lst:add-activity-with-filter-to-manifest}
	<activity android:name=".FakeMailActivity">
		<intent-filter>
			<action android:name="android.intent.action.SEND" />
			<category android:name="android.intent.category.DEFAULT"/>
			<data android:mimeType="text/html" />
		</intent-filter>
	</activity>
\end{XmlCode}

Nu, når du sender ``Send'' intent'en for ``text/html'', vil den spørge, om du vil bruge Mail-appen eller vores egen app.

\begin{figure}[h]
	\begin{center}
		\includegraphics[width=7cm]{email_chooser.png}
		\caption{Email vælger}
		\label{fig:android:activities:email_chooser}
	\end{center}
\end{figure}

\subsubsection{BroadcastReceivers}

Din app kan sende en intent som en broadcast, og Android-systemet sender hele tiden broadcasts. For at modtage en sådan broadcast har du brug for en \JavaInline|BroadcastReceiver|-klasse, som kan se ud som \autoref{lst:broadcast-receiver}.

\begin{JavaCode}{BroadcastReciever}{lst:broadcast-receiver}
	public class NoPhoneCallsReceiver extends BroadcastReceiver {
		@Override
		public void onReceive(Context context, Intent intent) {
		}
	}
\end{JavaCode}

Vi kunne få denne broadcastreceiver til, modtage broadcasten ``OutgoingCalls'', vi skal blot tilføje den til manifestet (på samme sted hvor du ville tilføje en ny aktivitet) og give det et intentfilter som i \autoref{lst:add-broadcastreceiver-to-manifest}.

\begin{XmlCode}{BroadcastReceiver til opkald}{lst:add-broadcastreceiver-to-manifest}
	<receiver android:name=".NoPhoneCallsReceiver">
		<intent-filter>
			<action android:name="android.intent.action.NEW_OUTGOING_CALL" />
		</intent-filter>
	</receiver>
\end{XmlCode}

Opsnapning af disse udgående telefonopkald broadcasts, kan have mange anvendelser, såsom at ringe dine venner via VoIP i stedet for via telefonnetværket, eller man kan sørge for, at den korrekte områdekode er vedhæftet, eller for at blokere alle opkald til numre, der ikke er på en forældrekontrol whitelist.

En sjov ting at gøre kunne også være at få denne broadcastreciever til at stoppe alle udgående telefonsamtaler, som i \autoref{lst:no-calls-receiver}.

\begin{JavaCode}{At stoppe udgående opkald}{lst:no-calls-receiver}
	public class NoPhoneCallsReceiver extends BroadcastReceiver {
		@Override
		public void onReceive(Context context, Intent intent) {
			setResultData(null);
		}
	}
\end{JavaCode}

\begin{exercise}
	Lav en app der har en main activity, og en anden activity der viser et billede. Tilføj en knap til din main activity, der åbner din anden activity.
\end{exercise}

\begin{exercise}
	Lav en app der har en main activity, og en anden activity med et EditText felt. Tilføj en knap og et \JavaInline|EditText|felt til din main activity, der åbner din anden activity og viser teksten fra \JavaInline|EditText| i din main activity.
\end{exercise}

\begin{exercise}
	Lav en Activity der har en ‘send Emil’ knap, der åbner en Email dialog
\end{exercise}

\begin{exercise}
	Lav en activity der agerer som et alternativ til email klienten når man trykker på ‘send mail’ knappen fra opgave 3
\end{exercise}

\begin{exercise}
	Lav en \JavaInline|BroadCastReciever|, der blokerer alle udgående opkald.
\end{exercise}

\begin{exercise}
	Lav en activity der sender en \JavaInline|Broadcast|, og en \JavaInline|BroadCastReciever| der modtager den givne \JavaInline|BroadCast|.
\end{exercise}

