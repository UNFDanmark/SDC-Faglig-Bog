% !TeX spellcheck = da_DK

\chapter{Animation}
Animation handler om at få ting til at bevæge sig og ændre udseende. Mere specifikt lærer I at få Views til at flytte sig på skærmen, rotere, ændre størrelse og på andre måder ændre udseende.
\section{Grundlæggende teori om animation}
Før vi går igang med at lave animationer i Android skal vi lige have på plads hvordan animationer virker.
Animationer og video generelt er en række af billeder som skifter så hurtigt at det ligner en flydende bevægelse. \\
%TODO Illustration af still frames i række
Når man laver en animation foregår det altså ved at man laver små ændringer af billedet så hurtigt efter hinanden at det bliver "flydende". En anden måde at sige det på er at billedet ændrer sig i små hak, og man gør de hak så små at det ikke kan ses at der arbejdes i hak. Det vi skal lave er altså noget der virker ligsom stopmotion. \\
Grunden til at man arbejder i hak er at det tager en hel del arbejde for en computer \marginnote{En mobil er også en computer} at udregne hvordan den skal vise et billede, så det gælder om at finde en balance mellem udseende og hvor meget af computerens regnekraft animationen tager. Det animation vi arbejder med har nu et fastsat tempo for hvor hurtigt billedet ændrer sig, så I skal ikke overveje denne balance.
\subsection{Animationens rammer}
En animation starter et sted og slutter på et andet sted. Det kan både være at den ændrer sin position på skærmen, at starter med at være lille og slutter med at være stor eller at det roterer en omgang.
Animationen tager også et stykke tid. Det kunne være et halvt sekund eller ti sekunder det tager for den at nå fra start til slut. Her gælder det om at lave en balance mellem at animationen tager tid nok så brugeren kan nå at se hvad der sker, og at brugeren ikke sidder og venter på at animationen bliver færdig. Hvis der ikke er stor forskel på start og slut i din animation kan det være at det slet ikke er nødvendigt at sætte en animation op, men bare at sætte dit View til slutpunktet med det samme. \marginnote{Hvis du bare vil have en oversigt over mulighederne for et view så se sektionen "Hvad kan vi ændre"} \\
Ud fra disse rammer kan vi så regne hvor langt vi skal være nået på et givet tidspunkt i animationen. 
\begin{example}
	Vi skal rykke et View fra 30 til 50 pixels horizontalt på 2 sekunder. Vi skal vide hvilken pixel vi er kommet til efter 0,5 sekunder, 1 sekund, 1,23 sekunder etc. Det vi kan gøre er at plotte en funktion, hvilket er meget simpelt hvis den er lineær:
	\begin{equation}
	pixel=10\cdot tid+30
	\end{equation}
	Hvor 10 er antallet af pixels Viewet bevæger sig i sekundet og 30 er startværdien
\end{example}
Vi skal nu ikke til at udregene en hel masse funktioner for animationen. Det gør Android for os.
\section{En helt simpel animation i Android}
For at lave en animation bruger vi den klasse der hedder ValueAnimator. Du kan tænke på et objekt af ValueAnimator som en animation. Du opretter en ValueAnimator som vist i \autoref{animation1}
\begin{JavaCode}{Oprettelse af ValueAnimator}{animation1}
	ValueAnimator minAnimation = ValueAnimator.ofInt(startInt, slutInt);
\end{JavaCode}
\marginnote{Noter at du ikke skal skrive new. Den tekniske grund hertil er at ofInt() er en metode som returnerer en ny ValueAnimator.}
ofInt() tager to integers som er din startværdi og din slutværdi. Det kunne f.eks. være 0 og 360 hvis man ville lave en animere et View til at rotere 360 grader. \\
Det næste din animation har brug for er hvor lang tid den skal køre. Den sættes som vist i \autoref{animation2}
\begin{JavaCode}{Sæt duration}{animation2}
	minAnimation.setDuration(1000);	
\end{JavaCode}
Tiden er i millisekunder. Altså betyder 1000 et sekund. Fem sekunder ville være 5000 og et kvart sekund ville være 250. Værdien skal være af typen long. Altså et heltal.
\subsection{At lave et billede}
Nu har vi fortalt Android hvor den skal starte og slutte, og hvor lang tid animationen skal tage. Nu skal vi så fortælle Android om den skal flytte noget, rotere noget eller gøre noget helt tredje. Dette gør vi ved at skrive en funktion som bliver kaldt hver gang billedet skal animeres endnu et hak. Altså skal vi programmere hvad der sker hver gang det er tid til et nyt billede. Koden ser ud som i \autoref{animation3}
\begin{JavaCode}{Opsæt metode til at lave et billede}{animation3}
	minAnimation.addUpdateListener(new ValueAnimator.AnimatorUpdateListener() {
		@Override
		public void onAnimationUpdate(ValueAnimator animation) {
			int vaerdiTilBillede = (int) animation.getAnimatedValue();
			
			mitView.setRotation(vaerdiTilBillede);
		}
	});
\end{JavaCode}
Her bruges noget advanceret kode som I ikke behøver at forstå. Den korte version er at vi giver Android en funktionen onAnimationUpdate() og fortæller ValueAnimator at den skal kalde funktionen når den vil have et nyt billede.\\
\marginnote{Hvis I vil have den længere udgave så spørg os eller internettet om anonyme klasser}
Den anden ting der sker er at vi får en integer fra animationen, som repræsenterer hvor langt vi er i animationen. Det er den variabel der hedder "vaerdiTilBillede" som man får fra getAnimatedValue(). Vi udfører så ændringen på vores view med denne værdi, i eksemplet sætter vi rotationen med setRotation, men I kan bruge denne værdi til lige hvad I vil. De mulige måder at ændre views på kommer om lidt.\\
Til sidst skal du kalde .start() på din ValueAnimator og så kører animationen. Et fuldt eksempel på en animation kan ses i \autoref{animation4}
\begin{JavaCode}{420 Rotate it}{animation4}
	ValueAnimator minAnimation = ValueAnimator.ofInt(0, 420);
	minAnimation.setDuration(1337);
	minAnimation.addUpdateListener(new ValueAnimator.AnimatorUpdateListener() {
		@Override
		public void onAnimationUpdate(ValueAnimator animation) {
			int vaerdiTilBillede = (int) animation.getAnimatedValue();
			mitView.setRotation(vaerdiTilBillede);
		}
	});
	minAnimation.start();
\end{JavaCode}
I kan nu lave animationer i Android. Huzzah!

\section{Flere grundteknikker}
Nu hvor det helt basale er på plads kan vi udvide med nogle flere teknikker
%TODO Introduktion af Floats hvis de ikke allerede er introduceret
\subsection{Hvad kan vi ændre}
Her er der så en liste over hvad man kan lave med Views:\\
\begin{itemize}
	\item Rotation \\
	I kan sætte et views rotation som i \autoref{animationroter}
	\begin{JavaCode}{Roter et view}{animationroter}
		mitView.setRotation(jeresRotation);
	\end{JavaCode}
	Rotationen er i grader. 0 er normal rotation og 180 grader er på hovedet. Rotationen er en float. 
	\\
	\item Position \\
	Her sættes vertikal og horizontal position seperat. Koden ses i \autoref{animationflyt}
	\begin{JavaCode}{Flyt et view}{animationflyt}
		mitView.setTranslationX(jeresPositionX);
		mitView.setTranslationY(jeresPositionY);
	\end{JavaCode}
	Her skal det noteres at det er \textbf{øverste} venstre hjørne der er position (0,0). Ligeledes er det viewets øverste venstre hjørne der placeres. Hvis I vil placere et view op af skærmens højre side skal i altså sætte X til skærmbredden minus viewets bredde. I kan få bredden og højden af jeres view som i \autoref{animationbredde}
	\begin{JavaCode}{Få bredden af jeres view}{animationbredde}
		mitView.getWidth();
		mitView.getHeight();
	\end{JavaCode}
	\item Størrelse \\
	I kan sætte størrelsen som i \autoref{animationscale} 
	\begin{JavaCode}{Sæt størrelsen af jeres view}{animationscale}
		mitView.setScaleX(jeresStoerlseX);
		mitView.setScaleY(jeresStoerlseY);
	\end{JavaCode}
	Her er 1 den størrelse som viewet starter med. Skal størrelsen fordobles skal I give slutværdien 2. Skal den halveres skal I give slutværdi 0.5 etc. Hvis i vil sætte størrelsen til et præcist antal pixels skal i lave en udregning ud fra viewets nuværende størrelse. \\
	\item Gennemsigtighed \\
	Et view kan gøres gennemsigtigt (Man kan stadig interagere med det) med denne metoden set i \autoref{animationgennemsigtig}
	\begin{JavaCode}{Sæt et views gennemsigtighed}{animationgennemsigtig}
		mitView.setAlpha(jeresGennemsigtighed);
	\end{JavaCode}
	Her er 0 usynlig og 1 er fuldt synlig. Værdien som tages er selvfølgelig en float.
\end{itemize}
\subsection{Størrelsen af skærmen}
Der findes mange forskellige Android mobiler, og de har mange forskellige skærmstørrelser, især hvis man også arbejder med tablets. 
\begin{example}
	Den mobil I arbejder med har en skærmbredde på 480 pixels. I har et view som skal krydse skærmen og sætter det til at starte i 0 og bevæge sig 480 pixels. Og det virker lige som det skal på jeres test mobil. I prøver så at køre appen på en tablet med en skærmbredde på 1080 pixels. Jeres view når ikke engang halvvejs over skærmen. 
\end{example}
Det gælder derfor om altid at arbejde ud fra skærmens størrelse. Jeres view i eksemplet skal bevæge sig en hel skærmbredde. En figur som står i baggrunden og hopper skal hoppe 1/3 af skærmenhøjden. Etc.\\

Og siden skærmstørrelsen er så kritisk skulle man tro at den var lettere at få fat på. Men nej, koden for at få skærmstørrelsen ser ud som i \autoref{animation5}
\begin{JavaCode}{Kode til at få skærmstørrelse}{animation5}
	final View layout = findViewById(R.id.layout);
	ViewTreeObserver observer = layout.getViewTreeObserver();
	observer.addOnGlobalLayoutListener(new ViewTreeObserver.OnGlobalLayoutListener() {
		@Override
		public void onGlobalLayout() {
			screenHeight = layout.getHeight();
			screenWidth = layout.getWidth();
			//kode der bruger height og width her
			boardLayout.getViewTreeObserver().removeOnGlobalLayoutListener(this);
		}
	});
\end{JavaCode}

Dette skal stå inde i onCreate(). I skal også give jeres yderste layout et ID og hente det med findViewByID(). Og så skal i have screenHeight og screenWidth som feltvariabler. Og hvis I skal bruge størrelsen af skærmen i onCreate() skal den kode også stå inde i onGlobalLayout().
Grunden hertil er at Android ikke har udregnet hvordan grafikken placeres når OnCreate bliver kaldt. Vi får den så til at kalde onGlobalLayout når den har fundet størrelsen af layoutet.
\subsection{Interpolatorer}
Det vi har defineret for Android er egentlig kun startpunkt, slutpunkt og tid imellem dem. Hvordan funktionen mellem de to punkter ser ud kan også defineres. Det kan være at den bare er lineær, men den kunne også være en exponentionel funktion hvor hastigheden starter med at være er langsom og så derefter bliver hurtigere. 
Hvis du ikke sætter en interpolator så vil Android bruge AccelerateDecelerateInterpolator. Altså vil hastigheden for animationen være langsommere i starten og slutningen og så være hurtig i midten. Du sætter en interpolator som i \autoref{animation6:}
\begin{JavaCode}{Kode til at sætte interpolatoren}{animation6}
	animator.setInterpolator(new LinearInterpolator());
\end{JavaCode}

Du kan sætte interpolatoren mellem at du opretter animationen og starter den.
De helt basale muligheder er LinearInterpolator, AccelerateInterpolator, DecelerateInterpolator og så selvfølgelig AccelerateDecelerateInterpolator. Der findes dog flere som man kan finde online.
%TODO Illustrationer til interpolators
\subsection{Kontrolmetoder}
Animator har yderligere en række metoder som går den lettere at arbejde med. Du kan sætte animationen til først at starte noget tid efter at du kalder .start() ved at kalde koden i \autoref{animation7} \\
\begin{JavaCode}{Kode til at sætte et startDelay}{animation7}
	minAnimation.setStartDelay(minForsinkelse);
\end{JavaCode}
Forsinkelsen er igen i millisekunder, præcist som setDuration. 
Yderligere er der metoderne .pause(), .resume(), .cancel(),  .end() og .reverse() som meget vel giver sig selv. .end() sætter animationen til slutværdien mens de andre efterlader animationen ved den værdi den er på når de bliver kaldt.
\section{Mere advancerede teknikker}
Her kommer vi omkring nogle lidt mere advancerede teknikker
\subsection{Flere animationer samlet}
Vi kan samle flere animationer i en, så vi kan flytte vores view diagonalt, eller få det til at ændre farve og rotere samtidig. Til dette bruger vi AnimatorSet. Det virker som set i \autoref{animation8}:
\begin{JavaCode}{Kode til at køre flere animationer samtidigt}{animation8}
AnimatorSet mitAnimationSet = new AnimatorSet();
mitAnimatorSet.playTogether(minAnimation, minAndenAnimation);
mitAnimatorSet.start();
\end{JavaCode}
Vi laver altså et AnimatorSet og så kalder vi .playTogether() med vores animationer som argumenter. Derefter starter vi vores sæt af animationer. Her skal det noteres at AnimatorSet opfører sig meget ligesom ValueAnimator og man kan stadig pause den og give den et startDelay m.m. 
Desuden kan man lave et AnimatorSet af andre AnimatorSets. Altså give et AnimatorSet som parameter til .playTogether(). Man kan desuden give playTogether lige så mange animationer som man har lyst til.
%Muligvis kortere navne til argumenter så linjen kan være i bogen
\begin{JavaCode}{Køre mange animationer samlet}{animation9}
	mitAnimatorSet.playTogether(minAnimation, minAndenAnimation, rotation, hop);
\end{JavaCode}

\subsection{Animationer i rækkefølge}
Vi kan også spille animationer efter hinanden. Her bruger vi igen AnimatorSet som ses i \autoref{animation10}
\begin{JavaCode}{Kode til at køre flere animationer efter hinanden}{animation10}
	AnimatorSet mitAnimationSet = new AnimatorSet();
	mitAnimatorSet.playSequentially(minAnimation, minAndenAnimation);
	mitAnimatorSet.start();
\end{JavaCode}
Så kører minAnimation først og derefter kører minAndenAnimation. Igen kan man give lige så mange animationer som man vil. 
%igen, muligvis forkort argument navne for at passe koden til siden
\begin{JavaCode}{Kør mange animationer efter hinanden}{animationer11}
	mitAnimatorSet.playSequentially(minAnimation, minAndenAnimation, minTrejdeAnimation);
\end{JavaCode}
\subsection{Gentagelser}
Vi kan også sætte vores animationer til at gentage. Det er meget simpelt, se \autoref{animation12}
\begin{JavaCode}{Kode til at lave gentagelser}{animation12}
	minAnimation.setRepeatCount(antalGentagelser);
\end{JavaCode}
Her er det værd at notere at hvis man sætter antalGentagelser til 1 så afspiller animationen \textbf{2 gange} i alt.
Man kan også sætte gentagelsen som i \autoref{animation13}
\begin{JavaCode}{Sæt animationen til at køre tilbage}{animation13}
	minAnimation.setRepeatMode(ValueAnimator.REVERSE);
\end{JavaCode}
Hvilket gør at den på gentagelsen ikke går tilbage og kører forfra, men går fra slutværdien til startværdien. Den vil så igen gå fra start til slut på anden gentagelse etc.

%TODO Flere kodeeksempler?

\section{Hurtig guide}
Her er en lille 4 skridts guide til at lave en animation.
\begin{enumerate}
	\item Opret en ValueAnimator med enten ofInt() eller ofFloat() \autoref{animationguide1}
	\begin{JavaCode}{Opret animator}{animationguide1}
	ValueAnimator minAnimation = ValueAnimator.ofFloat(startFloat, slutFloat);
	\end{JavaCode}
	\item Sæt en varighed i millisekunder \autoref{animationguide2}
	\begin{JavaCode}{Sæt varighed}{animationguide2}
	minAnimation.setDuration(varighed);
	\end{JavaCode}
	\item Lav onAnimationUpdate og få den til at lave din ændring \autoref{animationsguide3}
	\begin{JavaCode}{Lav et billede}{animationsguide3}
		minAnimation.addUpdateListener(new ValueAnimator.AnimatorUpdateListener() {
			@Override
			public void onAnimationUpdate(ValueAnimator animation) {
				int vaerdiTilBillede = (int) animation.getAnimatedValue();
				//Manipuler dit view her
				mitView.setScale(vaerdiTilBillede);
			}
		});
	\end{JavaCode}
	\item Start animationen \autoref{animationguide4}
	\begin{JavaCode}{Start animationen}{animationguide4}
		minAnimation.start();
	\end{JavaCode}
\end{enumerate}

\subsubsection{Animationsopgaver}
Der er ingen tid afsat til specifikt at lave opgaver i animation, men programmering er et praktisk fag så det gælder om at øve sig. Her er nogle forslag til ting at øve sig med, men du kan også selv finde på udfordringer at øve dig på, inklusiv bare at gå direkte i gang med projektet. Hvis det ikke går godt så husk at tage små skridt og lave en ting af gangen.
\begin{exercise}
	Start helt fra bunden. Få en knap til at rotere 360 grader når du trykker på den.
\end{exercise}
\begin{exercise}
	Giv rotationen et startdelay så der går et øjeblik før den begynder at rotere.
\end{exercise}
\begin{exercise}
	Sæt din knap til at starte i venstre side af skærmen og så bevæge sig til midt på skærmen når du trykker på den.
\end{exercise}
\begin{exercise}
	Sæt bevægelsens interpolator til at være lineær
\end{exercise}
\begin{exercise}
	Kombiner de to første animationer så knappen først bevæger sig sidelæns og så derefter roterer
\end{exercise}
\begin{exercise}
	Sæt nu knappen til at starte i øverste højre hjørne. Få den til at bevæge sig lodret og vandret samtidigt til midt på skærmen, og derefter rotere en omgang. Noter hvordan at den lodrette bevægelse starter og slutter langsomt fordi den bruger den normale interpolator mens den horizontale bevægelse er lineær
\end{exercise}
\begin{exercise}
	Få rotationen til at gentage sig selv 2 gange. Sæt derefter dens repeatMode til at være reverse.
\end{exercise}
\begin{exercise}
	Lav en knap som bevæger sig yderligere 10 pixels til højre hver gang man trykker på den. Hint: Her ændrer startInt sig for hver animation.
\end{exercise}
\begin{exercise}
	Få helt styr på at arbejde med skærmens størrelse. Placer en knap i øverste venstre hjørne, og få den til at tage en fuld runde langs telefonens kant når du trykker på den. 
\end{exercise}